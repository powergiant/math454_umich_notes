\section{PDEs in cylindrical coordinates}

\subsection{Laplace equation in cylindrical coordinates}

In this section, we will discuss how to solve the Laplace equation $\Delta u = 0$ in cylindrical coordinates. The operator $\Delta$, often refered to as Laplacian, is defined by the following
\begin{equation}\label{eq.Laplacian}
    \Delta = \partial_{xx} + \partial_{yy} \quad \textrm{(in 2D)},\qquad\qquad \Delta = \partial_{xx} + \partial_{yy} + \partial_{zz} \quad \textrm{(in 3D)}
\end{equation}

We consider the cylindrical domain $\{(x, y, z): R_1^2\le x^2 + y^2 \le R_2^2,\,, z\in [-L, L]\}$, for which cylindrical coordinates $(\rho, \varphi, z)$ are particularly useful. The transformation from cartesian coordinates $(x, y, z)$ to cylindrical coordinates can simplify the domain to $\{(x, y, z): R_1\le r \le R_2,\,\varphi\in [0, 2\pi],\,z\in [-L, L]\}$.

The relationship between Cartesian coordinates $(x, y, z)$ and cylindrical coordinates $(\rho, \varphi, z)$ is given by
\begin{equation}\label{eq.from_cartesian_to_cylindrical}
    x = \rho \cos(\varphi), \quad y = \rho \sin(\varphi), \quad z = z.
\end{equation}

Additionally, we have the inverse relationships,
\begin{equation}\label{eq.from_cylindrical_to_cartesian}
    \rho = \sqrt{x^2 + y^2}, \quad \varphi = \arctan\left(\frac{y}{x}\right), \quad z = z.
\end{equation}

To convert PDEs from Cartesian to cylindrical coordinates, we need to express the partial derivatives with respect to $x$ and $y$ in terms of $\rho$, $\varphi$, and $z$.

\begin{proposition}[]
The first-order partial derivatives transform as,
\begin{equation}\label{eq.Laplace_1st_order}
    \begin{gathered}
        \partial_x = \cos(\varphi) \partial_\rho - \frac{\sin(\varphi)}{\rho} \partial_\varphi
        \\
        \partial_y = \sin(\varphi) \partial_\rho + \frac{\cos(\varphi)}{\rho} \partial_\varphi
        % \\
        % \frac{\partial f}{\partial z} = \frac{\partial f}{\partial z}
    \end{gathered}
\end{equation}
\end{proposition}
\begin{proof} Using the chain rule, we have
\begin{equation}\label{eq.proof_Laplace_1st_order_1}
    \begin{gathered}
        f_x = f_\rho \rho_x - f_\varphi \varphi_x
        \\
        f_y = f_\rho \rho_y + f_\varphi \varphi_y
    \end{gathered}
\end{equation}
By \eqref{eq.from_cylindrical_to_cartesian}, we can compute that
\begin{equation}
    \rho_x = \cos(\varphi), \quad \rho_y = \sin(\varphi), \quad \varphi_x = - \frac{\sin(\varphi)}{\rho}, \quad \rho_x = \frac{\cos(\varphi)}{\rho}
\end{equation}
Inserting into \eqref{eq.proof_Laplace_1st_order_1}, we get
\begin{equation}
    \begin{gathered}
        f_x = \cos(\varphi) f_\rho - \frac{\sin(\varphi)}{\rho} f_\varphi
        \\
        f_y = \sin(\varphi) f_\rho + \frac{\cos(\varphi)}{\rho} f_\varphi
        \\
        \frac{\partial f}{\partial z} = \frac{\partial f}{\partial z}
    \end{gathered}
\end{equation}
This implies \eqref{eq.Laplace_1st_order}.
\end{proof}


\begin{proposition}[]
The Laplacian transforms as
\begin{equation}\label{eq.Laplace_2nd_order}
    \Delta f = f_{xx} + f_{yy} = \frac{1}{\rho} ( \rho f_\rho)_\rho + \frac{1}{\rho^2} f_{\varphi\varphi}.
\end{equation}
\end{proposition}
\begin{proof} By \eqref{eq.Laplace_1st_order}, we get
We can compute that
\begin{equation}
\begin{split}
    f_{xx}=&\left(\cos(\varphi) \partial_\rho - \frac{\sin(\varphi)}{\rho} \partial_\varphi\right)\left(\cos(\varphi) \partial_\rho - \frac{\sin(\varphi)}{\rho} \partial_\varphi\right)f,
    \\
    =&\cos ^2 \varphi\, f_{\rho\rho}+\frac{2 \cos \varphi \sin \varphi}{\rho^2} f_{\varphi}-\frac{2 \sin \varphi \cos \varphi}{\rho} f_{\rho\varphi}+\frac{\sin ^2 \varphi}{\rho} f_{\rho}+\frac{\sin ^2 \varphi}{\rho^2} f_{\varphi\varphi}. 
\end{split}
\end{equation}
\begin{equation}
\begin{split}
    f_{yy}=&\left(\cos(\varphi) \partial_\rho - \frac{\sin(\varphi)}{\rho} \partial_\varphi\right)\left(\cos(\varphi) \partial_\rho - \frac{\sin(\varphi)}{\rho} \partial_\varphi\right)f,
    \\
    =&\sin ^2 \varphi\, f_{\rho\rho}-\frac{2 \sin \varphi \cos \varphi}{\rho^2} f_{\varphi}+\frac{2 \sin \varphi \cos \varphi}{\rho} f_{\rho\varphi}+\frac{\cos ^2 \varphi}{\rho} f_{\rho}+\frac{\cos ^2 \varphi}{\rho^2} f_{\varphi\varphi}.
\end{split}
\end{equation}

Summing the above equations, we get
\begin{equation}
    \Delta f = \partial_{xx} f + \partial_{yy} f = \frac{1}{\rho} ( \rho f_\rho)_\rho + \frac{1}{\rho^2} f_{\varphi\varphi}.
\end{equation}
This completes the proof.
\end{proof}


This expression is crucial for solving PDEs such as the Laplace equation or the wave equation in cylindrical coordinates.

\subsection{Separation of Variables for the 2D Laplace equation}
Separation of variables is a common technique used to solve PDEs, especially in systems with boundary conditions. Let's consider solving the equation on a cylindrical domain
\begin{equation} 
    \Delta u = 0, \quad R_1 \leq \sqrt{x^2+y^2} \leq R_2
\end{equation}
where $\Delta$ is the Laplacian operator.

\subsubsection{The case of $R_1 = 0$ and $R_2 = R$}
Let us find separated solutions of Laplace's equation $\Delta u = 0$ on the domain $0 \leq \sqrt{x^2+y^2} \leq R$. In terms of cylindrical coordinates, the domain becomes $0\le \rho \le R$, $-\pi \leq \varphi \leq \pi$ and the Laplacian becomes $\frac{1}{\rho} ( \rho u_\rho)_\rho + \frac{1}{\rho^2} u_{\varphi\varphi}$. Assume that $u$ is smooth and is independent of $z$. The boundary value problem should be written as
\begin{equation}\label{eq.boundary_value_cylindrical_disk}
\left\{\begin{aligned}
    &\Delta u = \frac{1}{\rho} ( \rho u_\rho)_\rho + \frac{1}{\rho^2} u_{\varphi\varphi} = 0, \quad && 0\leq \rho \leq R, \quad -\pi \leq \varphi \leq \pi
    \\
    &u(R, \varphi)=u_2(\varphi), && -\pi \leq \varphi \leq \pi. 
\end{aligned}\right.
\end{equation}

By plugging $u(\rho, \varphi)=R(\rho) \Phi(\varphi)$ into $\Delta u=0$, we obtain
$$
0=\frac{1}{\rho} ( \rho u_\rho)_\rho+\frac{1}{\rho^2} u_{\varphi \varphi}=\frac{1}{\rho}(\rho R^{\prime})^{\prime} \Phi+\frac{1}{\rho^2} R \Phi^{\prime \prime}
$$

Dividing by $R \Phi$ and multiplying by $\rho^2$, we have
$$
\rho \frac{(\rho R^{\prime})^{\prime}}{R} = - \frac{\Phi^{\prime \prime}}{\Phi}
$$


By introducing the separation constant $\lambda$, we have
\begin{equation}\label{eq.Laplace_separation_var}
\left\{\begin{aligned}
    &\Phi^{\prime \prime}+\lambda \Phi=0, \quad \Phi(-\pi)=\Phi(\pi), \quad \Phi^{\prime}(-\pi)=\Phi^{\prime}(\pi), 
    \\
    &\rho (\rho R^{\prime})^{\prime} - \lambda R = 0.
\end{aligned}
\right.
\end{equation}

Solve the eigenvalue and eigenfunction $\lambda$ and $\Phi$. We get that $\lambda=m^2$ and $\Phi_m(\varphi)=A_m \cos m \varphi+B_m \sin m \varphi$ ($m=0,1,2, \ldots$). 

Given the expression $\lambda=m^2$ for $\lambda$, we can also solve $R(\rho)$. First inserting $\lambda=m^2$ and the change of variable $\rho = e^s$, the second equation becomes $R''(s) - m^2 R(s) = 0$. The general solution is given by $R(s) = Ae^{ms} + Be^{-ms}$. Transforming back to $\rho$ variable, we get that $R_m(\rho) = C_m\rho^m + D_m\rho^{-m}$. The case of $m = 0$ is special, where the solution can be easily solved as $R_0(\rho) = C_0+D_0 \ln \rho$.

Therefore, the separated solutions are obtained as
\begin{equation}
u(\rho, \varphi)=\left\{
\begin{aligned}
    &\left(C_m\rho^m + D_m\rho^{-m}\right)\left(A_m \cos m \varphi+B_m \sin m \varphi\right), && m=1,2, \ldots, 
    \\
    &C_0+D_0 \ln \rho, && m=0 
\end{aligned}
\right.
\end{equation}


If $D_m\neq 0$ for some $m\ge 0$, we have $|u| \rightarrow \infty$ as $\rho \rightarrow 0$ and $u$ is not smooth. Therefore,
$$
u(\rho, \varphi)=\rho^m\left(A_m \cos m \varphi+B_m \sin m \varphi\right), \quad m=0,1,2, \ldots
$$
where we take $C_m = 1$ as we can merge $C_m$ with $A_m$ and $B_m$. For the case of $m = 0$, $u(\rho, \varphi)$ is constant according to the above expression.

Therefore, we can write the general solution
$$
u(\rho, \varphi)=\sum_{m = 0}^\infty\rho^m\left(A_m \cos m \varphi+B_m \sin m \varphi\right), \quad m=0,1,2, \ldots
$$

Finally, to decide the coefficients $A_m$ and $B_m$, we use the boundary condition in \eqref{eq.boundary_value_cylindrical_disk}.
$$
u_2(\varphi) = u(R, \varphi)=\sum_{m = 0}^\infty R^m\left(A_m \cos m \varphi+B_m \sin m \varphi\right), \quad m=0,1,2, \ldots
$$

Therefore, $A_mR^m$ and $B_mR^m$ are Fourier coefficients of the function $u_2(\varphi)$, which can be computed using the formula of coefficients \eqref{eq.fourier_coefficient}. 

\subsubsection{The general case}
Let us find separated solutions of Laplace's equation $\Delta u = 0$ on the domain $R_1 \leq \sqrt{x^2+y^2} \leq R_2$. In terms of cylindrical coordinates, the domain becomes $R_1\le \rho \le R_2$, $-\pi \leq \varphi \leq \pi$ and the Laplacian becomes $\frac{1}{\rho} ( \rho u_\rho)_\rho + \frac{1}{\rho^2} u_{\varphi\varphi}$. Assume that $u$ is smooth and is independent of $z$. The boundary value problem should be written as
\begin{equation}\label{eq.boundary_value_cylindrical_annulus}
\left\{\begin{aligned}
    &\Delta u = \frac{1}{\rho} ( \rho u_\rho)_\rho + \frac{1}{\rho^2} u_{\varphi\varphi} = 0, \quad && R_1\le \rho \le R_2, \quad -\pi \leq \varphi \leq \pi
    \\
    &u(R_2, \varphi)=u_2(\varphi),\quad u(R_2, \varphi)=u_2(\varphi),\qquad && -\pi \leq \varphi \leq \pi. 
\end{aligned}\right.
\end{equation}

By plugging $u(\rho, \varphi)=R(\rho) \Phi(\varphi)$ into $\Delta u=0$, we obtain
$$
0=\frac{1}{\rho} ( \rho u_\rho)_\rho+\frac{1}{\rho^2} u_{\varphi \varphi}=\frac{1}{\rho}(\rho R^{\prime})^{\prime} \Phi+\frac{1}{\rho^2} R \Phi^{\prime \prime}
$$

Dividing by $R \Phi$ and multiplying by $\rho^2$, we have
$$
\rho \frac{(\rho R^{\prime})^{\prime}}{R} = - \frac{\Phi^{\prime \prime}}{\Phi}
$$


By introducing the separation constant $\lambda$, we have
\begin{equation}\label{eq.Laplace_separation_var_general}
\left\{\begin{aligned}
    &\Phi^{\prime \prime}+\lambda \Phi=0, \quad \Phi(-\pi)=\Phi(\pi), \quad \Phi^{\prime}(-\pi)=\Phi^{\prime}(\pi), 
    \\
    &\rho (\rho R^{\prime})^{\prime} - \lambda R = 0.
\end{aligned}
\right.
\end{equation}

Solve the eigenvalue and eigenfunction $\lambda$ and $\Phi$. We get that $\lambda=m^2$ and $\Phi_m(\varphi)=A_m \cos m \varphi+B_m \sin m \varphi$ ($m=0,1,2, \ldots$). 

Given the expression $\lambda=m^2$ for $\lambda$, we can also solve $R(\rho)$. First inserting $\lambda=m^2$ and the change of variable $\rho = e^s$, the second equation becomes $R''(s) - m^2 R(s) = 0$. The general solution is given by $R(s) = Ae^{ms} + Be^{-ms}$. Transforming back to $\rho$ variable, we get that $R_m(\rho) = C_m\rho^m + D_m\rho^{-m}$. The case of $m = 0$ is special, where the solution can be easily solved as $R_0(\rho) = C_0+D_0 \ln \rho$.

Therefore, the separated solutions are obtained as
\begin{equation}
u(\rho, \varphi)=\left\{
\begin{aligned}
    &\left(C_m\rho^m + D_m\rho^{-m}\right)\left(A_m \cos m \varphi+B_m \sin m \varphi\right), && m=1,2, \ldots, 
    \\
    &C_0+D_0 \ln \rho, && m=0 
\end{aligned}
\right.
\end{equation}


Everything up to this point is the same as the case of $R_1 = 0$, $R_2 = R$. For general annulus domain, we do not have the condition that $|u| < \infty$ as $\rho \rightarrow 0$. Therefore, $D_m$ are not necessarily vanishing and the general solution can be written as
$$
\begin{aligned}
    u(\rho, \varphi)=&\sum_{m = 0}^\infty\left(C_m\rho^m + D_m\rho^{-m}\right)\left(A_m \cos m \varphi+B_m \sin m \varphi\right)
    \\
    =& \sum_{m = 0}^\infty\left(A^{(1)}_m\rho^m + A^{(2)}_m\rho^{-m}\right)\cos m \varphi + \left(B^{(1)}_m\rho^m + B^{(2)}_m\rho^{-m}\right)\sin m \varphi
\end{aligned}
$$
where in the second line, we introduced the new constants $A^{(1)}_m = C_mA_m$, $A^{(2)}_m = D_mA_m$, $B^{(1)}_m = C_mB_m$ and $B^{(2)}_m = D_mB_m$.

Finally, to decide the coefficients $A^{(1)}_m$, $B^{(1)}_m$, $A^{(2)}_m$ and $B^{(2)}_m$, we use the boundary condition in \eqref{eq.boundary_value_cylindrical_disk}.
$$
u_1(\varphi) = u(R_1, \varphi)=\sum_{m = 0}^\infty\left(A^{(1)}_mR_1^m + A^{(2)}_mR_1^{-m}\right)\cos m \varphi + \left(B^{(1)}_mR_1^m + B^{(2)}_mR_1^{-m}\right)\sin m \varphi
$$
$$
u_2(\varphi) = u(R_2, \varphi)=\sum_{m = 0}^\infty\left(A^{(1)}_mR_2^m + A^{(2)}_mR_2^{-m}\right)\cos m \varphi + \left(B^{(1)}_mR_2^m + B^{(2)}_mR_2^{-m}\right)\sin m \varphi
$$

Therefore, $\alpha^{(1)}_m = A^{(1)}_mR_1^m + A^{(2)}_mR_1^{-m}$ and $\beta^{(1)}_m = B^{(1)}_mR_1^m + B^{(2)}_mR_1^{-m}$ are Fourier coefficients of the function $u_1(\varphi)$; $\alpha^{(2)}_m = A^{(1)}_mR_2^m + A^{(2)}_mR_2^{-m}$ and $\beta^{(2)}_m = B^{(1)}_mR_2^m + B^{(2)}_mR_2^{-m}$ are Fourier coefficients of the function $u_2(\varphi)$, which can be computed using the formula of coefficients \eqref{eq.fourier_coefficient}. After solving $\alpha^{(1)}_m$, $\beta^{(1)}_m$, $\alpha^{(2)}_m$, $\beta^{(2)}_m$, we can solve $A^{(1)}_m$, $B^{(1)}_m$, $A^{(2)}_m$ and $B^{(2)}_m$ from them.

\subsection{The Bessel function}

When applying separation of variables to more general PDEs involving the Laplacian operator, a new class of functions, known as Bessel functions, will emerge.

\subsubsection{Eigenfunctions of the Laplacian}
Let us find separated solutions more general PDE on a disk $-\Delta u = \lambda u$ on the domain $0 \leq \sqrt{x^2+y^2} \leq R$. In terms of cylindrical coordinates, the domain becomes $0\le \rho \le R$, $-\pi \leq \varphi \leq \pi$ and the Laplacian becomes $\frac{1}{\rho} ( \rho u_\rho)_\rho + \frac{1}{\rho^2} u_{\varphi\varphi}$. Assume that $u$ is smooth and is independent of $z$. The boundary value problem should be written as
\begin{equation}\label{eq.boundary_value_cylindrical_helmholtz_disk}
\left\{\begin{aligned}
    &\Delta u = \frac{1}{\rho} ( \rho u_\rho)_\rho + \frac{1}{\rho^2} u_{\varphi\varphi} = -\lambda u, \quad && 0\leq \rho \leq R, \quad -\pi \leq \varphi \leq \pi
    \\
    &u(R, \varphi)=u_2(\varphi), && -\pi \leq \varphi \leq \pi. 
\end{aligned}\right.
\end{equation}

By plugging $u(\rho, \varphi)=R(\rho)\, \Phi(\varphi)$ into $\Delta u = -\lambda u$, we obtain
$$
-\lambda R(\rho)\, \Phi(\varphi)=\frac{1}{\rho}(\rho R^{\prime})^{\prime} \Phi+\frac{1}{\rho^2} R \Phi^{\prime \prime}
$$

Dividing by $R \Phi$ and multiplying by $\rho^2$, we have
$$
\frac{\rho(\rho R^{\prime})^{\prime} + \lambda \rho^2 R}{R} = - \frac{\Phi^{\prime \prime}}{\Phi} = m^2
$$
where we introduced the separation constant $m^2$, we have
\begin{equation}\label{eq.helmholtz_separation_var}
\left\{\begin{aligned}
    &\Phi^{\prime \prime}+m^2 \Phi=0, \quad \Phi(-\pi)=\Phi(\pi), \quad \Phi^{\prime}(-\pi)=\Phi^{\prime}(\pi), 
    \\
    &\rho(\rho R^{\prime})^{\prime} + (\lambda \rho^2 - m^2) R = 0.
\end{aligned}
\right.
\end{equation}

Solve the eigenvalue and eigenfunction $\lambda$ and $\Phi$. We get that $\lambda=m^2$ and $\Phi_m(\varphi)=A_m \cos m \varphi+B_m \sin m \varphi$ ($m=0,1,2, \ldots$). 

There is no elementary expression for the solution of the equation of $R(\rho)$. However, if we take $R(\rho) = J_m(\sqrt{\lambda}\rho)$, then the equation can be simplified to 
\begin{equation}
    \rho(\rho J_m^{\prime})^{\prime} + (\rho^2 - m^2) J_m = 0.
\end{equation}
The solution $J_m(\rho)$ to the above equation satisfying $J_m(0) < \infty$ is refered to as the Bessel function.

Therefore, the separated solutions are obtained as
\begin{equation}
u(\rho, \varphi)=J_m(\sqrt{\lambda}\rho)\left(A_m \cos m \varphi+B_m \sin m \varphi\right),\quad m=1,2, \ldots.
\end{equation}

The general solution can be written as
\begin{equation}\label{eq.helmholtz_general_sol}
    u(\rho, \varphi)=\sum_{m = 0}^\infty J_m(\sqrt{\lambda}\rho)\left(A_m \cos m \varphi+B_m \sin m \varphi\right)
\end{equation}

Finally, to decide the coefficients $A_m$ and $B_m$, we use the boundary condition in \eqref{eq.boundary_value_cylindrical_disk}.
\begin{equation}
    u_2(\varphi) = u(R, \varphi)=\sum_{m = 0}^\infty J_m(\sqrt{\lambda}R)\left(A_m \cos m \varphi+B_m \sin m \varphi\right), \quad m=0,1,2, \ldots
\end{equation}

Therefore, $A_m J_m(\sqrt{\lambda}R)$ and $B_m J_m(\sqrt{\lambda}R)$ are Fourier coefficients of the function $u_2(\varphi)$, which can be computed using the formula of coefficients \eqref{eq.fourier_coefficient}. 

\subsubsection{The Bessel function}

\begin{definition}[The Bessel function] The following ODE is refered to as the \underline{Bessel equation}
\begin{equation}\label{eq.Bessel_ODE}
    x(x J_m^{\prime}(x))^{\prime} + (x^2 - m^2) J_m(x) = 0.
\end{equation}
The solution $J_m(x)$ to the above equation satisfying $J_m(0) < \infty$ is refered to as the \underline{Bessel function}. 
\end{definition}
\begin{remark}[]
    The Bessel function defined in this manner has an ambiguity in constants, as any solution of the form $C J_m(x)$ also satisfies the above equation. This ambiguity is resolved by defining $J_m(x)$ using \eqref{eq.Bessel_int_formula} below.
\end{remark}

\begin{lemma}[]
The Bessel equation is equivalent to the following equation.
$$
    J_m^{\prime \prime}(x)+\frac{1}{x} J_m^{\prime}(x)+\left(1-\frac{m^2}{x^2}\right) J_m(x)=0 .
$$
\end{lemma}
\begin{proof}
    This follows from a simple computation.
\end{proof}

\begin{proposition}[Integral formula for the Bessel function] We have (up to a constant)
\begin{equation}\label{eq.Bessel_int_formula}
    J_m(x)=\frac{1}{2 \pi i^m} \int_{-\pi}^\pi e^{i x \cos \theta} e^{-i m \theta} d \theta, \quad m=0,1,2, \ldots .
\end{equation}
\end{proposition}
\begin{proof}
Since $e^{iy} = e^{i\rho \cos \varphi}$ is a solution to the equation $-\Delta u = u$, by the expression for the general solution \eqref{eq.helmholtz_general_sol} (taking $\lambda = 1$), we know that 
\begin{equation}
    e^{i\rho \cos \varphi} = \sum_{m = 0}^\infty J_m(\rho)\left(A_m \cos m \varphi+B_m \sin m \varphi\right)
\end{equation}
Transforming to complex Fourier series, we get 
\begin{equation}
    e^{i\rho \cos \varphi} = \sum_{m = 0}^\infty J_m(\rho)\left(\alpha_m e^{i m \varphi}+\beta_m e^{-i m \varphi}\right)
\end{equation}
Therefore, $\alpha_m J_m(\rho)$ is the complex Fourier coefficients of the function $e^{i\rho \cos \varphi}$. Using the formula of coefficients, we know that \eqref{eq.fourier_complex_coefficients}
\begin{equation}
    \alpha_m J_m(\rho) = \frac{1}{2 \pi} \int_{-\pi}^\pi e^{i\rho \cos \varphi} e^{- i m \varphi} d\varphi
\end{equation}
Therefore, $J_m(\rho)$ is proportional to $\frac{1}{2 \pi} \int_{-\pi}^\pi e^{i\rho \cos \varphi} e^{- i m \varphi} d\varphi$. If we choose the constant $\alpha_m = i^{m}$, then $J_m(\rho)$ is a real value function \textbf{TODO: add a proof to this}. Under this choice of constants, we prove \eqref{eq.Bessel_int_formula}. 
\end{proof}

\textbf{TODO: add a picture}

\begin{lemma}[] We have
\begin{equation}\label{eq.Bessel_value_at_zero}
    J_m(0) = \left\{\begin{aligned}
    &1,\quad &&\textit{if } m = 0,
    \\
    &0, &&\textit{if } m \neq 0.
\end{aligned}
\right.
\end{equation}
% \begin{equation}\label{eq.derivative_at_zero}
%     J'_m(0) = \left\{\begin{aligned}
%     &\frac{1}{2},\quad &&\textit{if } m = 1,
%     \\
%     &0, &&\textit{if } m \neq 0.
% \end{aligned}
% \right.
% \end{equation}
\end{lemma}
\begin{proof}
Let $x = 0$ in \eqref{eq.Bessel_int_formula}. 
\begin{equation}\label{eq.proof_Bessel_value_at_zero_1}
    \begin{split}
    J_m(\omega) =& \frac{1}{2\pi} \int_{-\pi}^{\pi} e^{i0 \cdot \cos \varphi} e^{-im \varphi} \, d\varphi
    \\
    =& \frac{1}{2\pi} \int_{-\pi}^{\pi} e^{-im \varphi} \, d\varphi
    \end{split}
\end{equation}

Evaluating this integral gives
\begin{equation}\label{eq.proof_Bessel_value_at_zero_2}
    J_m(\omega)=\frac{1}{2\pi} \int_{-\pi}^{\pi} e^{-im \varphi} \, d\varphi= 
    \begin{cases} 
    1 & \text{if } m = 0 \\ 
    0 & \text{if } m \neq 0 
    \end{cases}
\end{equation}
This proof uses the orthogonality of Fourier series \eqref{eq.Fourier_complex_orthogonality} by taking $n = 0$ and $L = 2\pi$.
\end{proof}

\subsubsection{The recurrence formula}
\begin{proposition}[]
The following recurrence formula holds.
\begin{equation}\label{eq.Bessel_recursion_1}
    J_m(x)=\frac{x}{2 m}\left[J_{m-1}(x)+J_{m+1}(x)\right], \quad m=1,2, \ldots .
\end{equation}
\begin{equation}\label{eq.Bessel_recursion_2}
    J_m^{\prime}(x)=\frac{1}{2}\left[J_{m-1}(x)-J_{m+1}(x)\right], \quad m=0,1,2, \ldots .
\end{equation}
\begin{equation}\label{eq.Bessel_recursion_3}
    \frac{d}{d x}\left[x^m J_m(x)\right]=x^m J_{m-1}(x), \quad m=1,2, \cdots
\end{equation}
\begin{equation}\label{eq.Bessel_recursion_4}
    \frac{d}{d x}\left[x^{-m} J_m(x)\right]=-x^{-m} J_{m+1}(x), \quad m=0,1,2, \cdots
\end{equation}
\end{proposition}
\begin{proof} Let us start with the proof of \eqref{eq.Bessel_recursion_1}
\begin{equation}\label{eq.proof_Bessel_recursion_1}
\begin{split}
    J_{m-1}(x) + J_{m+1}(x) =& \frac{1}{2\pi i^m} \left(i\int_{-\pi}^{\pi} e^{ix \cos \varphi} e^{-i(m-1) \varphi} d\varphi - i\int_{-\pi}^{\pi} e^{ix \cos \varphi} e^{-i(m+1) \varphi} d\varphi\right)
    \\
    =& \frac{1}{2\pi i^m} \int_{-\pi}^{\pi} e^{ix \cos \varphi} e^{-im \varphi} \underbrace{\left(ie^{i \varphi} - ie^{-i \varphi}\right)}_{2(\cos \varphi)'}d\varphi
    \\
    =& \frac{2}{2\pi i^m} \int_{-\pi}^{\pi} \underbrace{e^{ix \cos \varphi}(\cos \varphi)'}_{\frac{1}{ix}\left(e^{ix \cos \varphi}\right)'} e^{-im \varphi} d\varphi
    \\
    =& \frac{2}{ix}\frac{1}{2\pi i^m} \int_{-\pi}^{\pi} \left(e^{ix \cos \varphi}\right)' e^{-im \varphi} d\varphi.
\end{split}
\end{equation}
Apply the integration by parts,
\begin{equation}\label{eq.proof_Bessel_recursion_2}
\begin{split}
    J_{m-1}(x) + J_{m+1}(x) =& \frac{2}{ix}\frac{1}{2\pi i^m}\bigg(\underbrace{\left[\left(e^{ix \cos \varphi} e^{-im \varphi} \right)\right]_{-\pi}^{\pi}}_{= 0\ \textrm{by periodicity}} - \int_{-\pi}^{\pi} e^{ix \cos \varphi} \left(e^{-im \varphi}\right)' d\varphi \bigg)
    \\
    =& \frac{2m}{x}\frac{1}{2\pi i^m} \int_{-\pi}^{\pi} e^{ix \cos \varphi} e^{-im \varphi} d\varphi = \frac{2m}{x}J_m(x).
\end{split}
\end{equation}

Here is the proof of \eqref{eq.Bessel_recursion_2}
\begin{equation}\label{eq.proof_Bessel_recursion_3}
\begin{split}
    J_m'(x) =& \frac{1}{2\pi i^m} \int_{-\pi}^{\pi} \frac{d}{dx} \left( e^{ix \cos \varphi} \right) e^{-im \varphi} d\varphi
    \\
    =& \frac{1}{2\pi i^m} \int_{-\pi}^{\pi} e^{ix \cos \varphi} i\cos \varphi \, e^{-im \varphi} d\varphi
    \\
    =& \frac{1}{2\pi i^m} \int_{-\pi}^{\pi} e^{ix \cos \varphi} i\frac{e^{i\varphi} + e^{-i\varphi}}{2} \, e^{-im \varphi} d\varphi
    \\
    =& \frac{1}{2} \left( \frac{1}{2\pi i^{m-1}} \int_{-\pi}^{\pi} e^{ix \cos \varphi} e^{-i(m-1)\varphi} d\varphi - \frac{1}{2\pi i^{m+1}} \int_{-\pi}^{\pi} e^{ix \cos \varphi} e^{-i(m+1)\varphi} d\varphi \right)
    \\
    =& \frac{1}{2} (J_{m-1}(x) - J_{m+1}(x)).
\end{split}
\end{equation}

Here is the proof of \eqref{eq.Bessel_recursion_3}. Consider $\eqref{eq.Bessel_recursion_1}\times mx^{m-1} + \eqref{eq.Bessel_recursion_2}\times x^m$,
\begin{equation}\label{eq.proof_Bessel_recursion_4}
\begin{split}
( x^m J_m )' =& m x^{m-1} J_m + x^m J_m'
\\
=& m x^{m-1} \frac{x}{2m} ( J_{m-1} + J_{m+1} ) + \frac{x^m}{2} ( J_{m-1} - J_{m+1} )
\\
=& \frac{x^m}{2} (J_{m-1} + J_{m+1} + J_{m-1} - J_{m+1})
\\
=& x^m J_{m-1}.
\end{split}
\end{equation}

Here is the proof of \eqref{eq.Bessel_recursion_4}. Consider $-\eqref{eq.Bessel_recursion_1}\times mx^{-m-1} + \eqref{eq.Bessel_recursion_2}\times x^{-m}$,
\begin{equation}\label{eq.proof_Bessel_recursion_5}
\begin{split}
(x^{-m} J_m)' =& -m x^{-m-1} J_m + x^{-m} J_m' 
\\
=& -m x^{-m-1} \frac{x}{2m} ( J_{m-1} + J_{m+1} ) + \frac{x^{-m}}{2} ( J_{m-1} - J_{m+1} )
\\
=& \frac{x^{-m}}{2} (-J_{m-1} - J_{m+1} + J_{m-1} - J_{m+1} )
\\
=& - x^{-m} J_{m+1}.
\end{split}
\end{equation}
Therefore, the proof has been finished.
\end{proof}

\subsubsection{The orthogonality relation}
Consider the following Sturm-Liouville eigenvalues problem,
\begin{equation}\label{eq.Bessel_SL_problem}
\left\{
\begin{aligned}
    &x(x R')' + (\lambda x^2 - m^2)R = 0,
    \\
    &R(0) < \infty,\ R(1)\cos\beta + R'(1)\sin\beta = 0.
\end{aligned}
\right.
\end{equation}
We know that $R(x) = J_m(\sqrt{\lambda}x)$ is a solution to the above equation. Using the boundary condition, we know that $R(1)\cos\beta + R'(1)\sin\beta = 0$ $\Rightarrow$ $J_m(\sqrt{\lambda})\cos\beta + \sqrt{\lambda}J_m'(\sqrt{\lambda})\sin\beta = 0$. Therefore, if $\left\{x_n^{(m)}\right\}$ are the nonnegative solutions of the equation
\begin{equation}
    J_m\left(x_n^{(m)}\right) \cos \beta+x_n^{(m)} J_m^{\prime}\left(x_n^{(m)}\right) \sin \beta=0.
\end{equation}
Then $\lambda_n = \left(x_n^{(m)}\right)^2$. Applying the Sturm-Liouville theorem Theorem \ref{th.SL_1}, we get the following proposition.

\begin{proposition}[]
Let $\left\{x_n^{(m)}\right\}$ be the nonnegative solutions of the equation
\begin{equation}\label{eq.Bessel_roots}
    J_m\left(x_n^{(m)}\right) \cos \beta+x_n^{(m)} J_m^{\prime}\left(x_n^{(m)}\right) \sin \beta=0,
\end{equation}
where $m \geq 0$ and $0 \leq \beta \leq \pi / 2$.

Then we have the following orthogonality relations.
\begin{equation}\label{eq.Bessel_orthogonality_1}
    \int_0^1 J_m\left(x x_{n_1}^{(m)}\right) J_m\left(x x_{n_2}^{(m)}\right) x d x=0,\ n_1 \neq n_2
\end{equation}
\begin{equation}\label{eq.Bessel_orthogonality_2}
    \left\{\begin{aligned}
    &\int_0^1 J_m\left(x x_n^{(m)}\right)^2 x d x=\frac{1}{2} J_{m+1}\left(x_n^{(m)}\right)^2, && \text {if } \beta=0, 
    \\
    &\int_0^1 J_m\left(x x_n^{(m)}\right)^2 x d x=\frac{x_n^2-m^2+\cot ^2 \beta}{2 x_n^2} J_m\left(x_n^{(m)}\right)^2, && \text{if } 0<\beta \leq \frac{\pi}{2} .
    \end{aligned}\right.
\end{equation}
\end{proposition}
\begin{proof} For the ease of notation, let us drop the superscript in $x_n^{(m)}$. We know that $J_m\left(x x_{n}\right)$ are solutions of the following equation. 
\begin{equation}\label{eq.proof_Bessel_orthogonality_1}
    x(x R^{\prime})^{\prime} + (x_n^2 x^2 - m^2) R = 0.
\end{equation}
If we view $m$ as fixed, then the above equation is a Sturm-Liouville eigenvalue problem with eigenvalues $\lambda=x_n^2$, $s(x) = \rho(x) = x$ and $q(x) = m^2$. From the Sturm-Liouville theory (Theorem \ref{th.SL_1}), the first equation \eqref{eq.Bessel_orthogonality_1} (orthogonality) holds. For the second and third equations \eqref{eq.Bessel_orthogonality_2}, we multiply \eqref{eq.proof_Bessel_orthogonality_1} by $2 R^{\prime}$.
$$
2 x R^{\prime}\, (x R^{\prime})^{\prime} +\left(x_n^2 x^2-m^2\right) 2 R R^{\prime}=0
$$

We can rewrite this as
$$
\left[\left(x R^{\prime}\right)^2\right]^{\prime}+\left(x_n^2 x^2-m^2\right)\left(R^2\right)^{\prime}=0
$$

By integrating both sides and using integration by parts, we get
$$
\left.\left(x R^{\prime}\right)^2\right|_{x=1}-\left.\left(x R^{\prime}\right)^2\right|_{x=0}+\left.\left(x_n^2 x^2-m^2\right) R^2\right|_0 ^1-\int_0^1 2 x_n^2 x R^2 d x=0 .
$$

Note that $R(x)=J_m\left(x x_n\right)$ and $R^{\prime}(x)=x_m J_m^{\prime}(x x_n)$. Hence $R(0)=J_m(0)=0$ $(m=1,2, \cdots)$ by \eqref{eq.Bessel_value_at_zero}. We obtain
$$
\left[x_n J_m^{\prime}(x_n)\right]^2+\left(x_n^2-m^2\right) J_m(x_n)^2-2 x_n^2 \int_0^1 J_m\left(x x_n\right)^2 x d x=0 .
$$


Therefore, when $\beta=0$ (in other words, $J_m(x_n)=0$), we have
$$
\int_0^1 J_m\left(x x_n\right)^2 x d x=\frac{x_n^2\left[J_m^{\prime}(x_n)\right]^2}{2 x_n^2}=\frac{\left[\frac{m}{x_n} J_m(x_n)-J_{m+1}(x_n)\right]^2}{2}=\frac{J_{m+1}(x_n)^2}{2}
$$
where we applied the recurrence formula \eqref{eq.Bessel_recursion_4}. (Notice that $\frac{d}{d x}\left[x^{-m} J_m(x)\right]=-x^{-m} J_{m+1}(x)$ $\Leftrightarrow$ $J_m^{\prime}(x) = \frac{m}{x} J_m(x)-J_{m+1}(x)$).

When $0<\beta \leq \pi / 2$, or in other words, $J_m(x_n) \cos \beta+x_n J_m^{\prime}(x_n)\sin \beta=0$ (which implies that $J_m^{\prime}(x_n) = -\frac{1}{x_n} J_m(x_n) \cot \beta$), we have

$$
\begin{aligned}
\int_0^1 J_m\left(x x_n\right)^2 x d x & =\frac{x_n^2\left[J_m^{\prime}(x_n)\right]^2+\left(x_n^2-m^2\right) J_m(x_n)^2}{2 x_n^2} 
\\
& =\frac{x_n^2\left[\frac{-1}{x_n} J_m(x_n) \cot \beta\right]^2+\left(x_n^2-m^2\right) J_m(x_n)^2}{2 x_n^2} 
\\
& =\frac{\left(x_n^2-m^2+\cot ^2 \beta\right) J_m(x_n)^2}{2 x_n^2}
\end{aligned}
$$
This completes the proof of \eqref{eq.Bessel_orthogonality_2}.
\end{proof}

Now we introduce the notion of Fourier--Bessel series

\begin{definition}[Fourier-Bessel series]
    Let us consider the expansion of a piecewise smooth function $f(x)$, $0<x<1$, in a series of the form
$$
f(x)=\sum_{n=1}^{\infty} A_n J_m\left(x x_n^{(m)}\right), \quad 0<x<1,
$$
where $\left\{x_n\right\}$ are the nonnegative solutions of $J_m(x) \cos \beta+x J_m^{\prime}(x) \sin \beta=0$. This is called a \underline{Fourier--Bessel series}. 
\end{definition}

We have the following theorem.

\begin{theorem}[]
Let $m \geq 0$, $0 \leq \beta \leq \pi / 2$, and let $\left\{x_n: n \geq 1\right\}$ be the nonnegative solutions of \eqref{eq.Bessel_roots}. If $f(x)$, defined on $0<x<1$, is a piecewise smooth function, then $f(x)$ can be expanded in terms of Fourier--Bessel series
$$
    f(x)=\sum_{n=1}^{\infty} A_n J_m\left(x x_n^{(m)}\right), \quad 0<x<1,
$$
where the coefficients $A_n$ can be computed by
$$
    A_n=\frac{\int_0^1 f(x) J_m\left(x x_n^{(m)}\right) x d x}{\int_0^1 J_m\left(x x_n^{(m)}\right)^2 x d x}, \quad n=1,2, \cdots
$$
Moreover, the series $\sum_{n=1}^{\infty} A_n J_m\left(x x_n^{(m)}\right)$ converges for each $x \in[0,1]$ to $\frac{1}{2}[f(x+0)+f(x-0)]$ for $0<x<1$.
\end{theorem}
\begin{proof}
    \textbf{TODO: } By multiplying (3.10) by $J_m\left(x x_n^{(m)}\right)$ and integrating both sides, we obtain
\end{proof}

\begin{example}[]\label{ex.Fourier_Bessel_1}
Let us compute the Fourier--Bessel series of the function $f(x)=1$, $0<x<1$, where $m=0$ and $\beta=0$. We have $1=\sum_{n=1}^{\infty} A_n J_0\left(x x_n^{(m)}\right)$, where $J_0(x_n)=0$ and
$$
A_n=\frac{\int_0^1 J_m\left(x x_n^{(m)}\right) x d x}{\int_0^1 J_m\left(x x_n^{(m)}\right)^2 x d x}, \quad n=1,2, \cdots
$$

First applying a change of variable $t = xx_n$, then applying \eqref{eq.Bessel_recursion_3} and \eqref{eq.Bessel_orthogonality_2}, we obtain
$$
A_n=\frac{\frac{1}{x_n^2} \int_0^{x_n} t J_0(t) d t}{\int_0^1 J_0\left(x x_n\right)^2 x d x}\overset{\eqref{eq.Bessel_recursion_3}}{=}\frac{\left.\frac{1}{x_1^2} t J_1(t)\right|_0 ^{x_n}}{\int_0^1 J_0\left(x x_n\right)^2 x d x}\overset{\eqref{eq.Bessel_orthogonality_2}}{=}\frac{\frac{1}{x_n} J_1(x_n)}{\frac{1}{2} J_1(x_n)^2}=\frac{2}{x_n J_1(x_n)} .
$$
where we drop the superscript in $x_n^{(m)}$.

Therefore,
$$
1=2 \sum_{n=1}^{\infty} \frac{J_0\left(x x_n^{(m)}\right)}{x_n^{(m)} J_1\left(x_n^{(m)}\right)}.
$$
\end{example}


\subsection{The Taylor series}

Finally, we find the Taylor series solution of the Bessel's equation $x(x J_m^{\prime}(x))^{\prime} + (x^2 - m^2) J_m(x) = 0$ \eqref{eq.Bessel_ODE}. Let $J_m(x)=\sum_{n=0}^{\infty} a_n x^{n+\gamma}$ ($a_0 \neq 0$, $\gamma \geq 0$) be a solution to \eqref{eq.Bessel_ODE}. Inserting this expansion into \eqref{eq.Bessel_ODE}, we obtain
$$
\left(\gamma^2-m^2\right) a_0 x^\gamma+\left((1+\gamma)^2-m^2\right) a_1 x^{\gamma+1}+\sum_{n=2}^{\infty}\left[\left((n+\gamma)^2-m^2\right) a_n+a_{n-2}\right] x^{n+\gamma}=0 .
$$
Hence,
$$
\gamma=m, \quad a_1=0, \quad a_n=\frac{-a_{n-2}}{n(n+2 m)}(n \geq 2)
$$
From this, we obtain
$$
J_m(x)=a_0 x^m\left[1+\sum_{n=1}^{\infty} \frac{(-1)^n x^{2 n}}{2(2+2 m) 4(4+2 m) \cdots 2 n(2 n+2 m)}\right]
$$

Let us choose $a_0=2^m / m!$. Then, we have
$$
J_m(x)=\sum_{n=0}^{\infty} \frac{(-1)^n x^{2 n+m}}{2^{m+2 n}(m+n)!n!}
$$


\subsection{The vibrating drumhead}

Let us consider small transverse vibrations of a circular membrane.
\begin{equation}\label{eq.vibrating_drumhead}
\left\{\begin{aligned}
    &u_{tt}=c^2 \Delta u=c^2\left(u_{\rho \rho}+\frac{1}{\rho} u_\rho+\frac{1}{\rho^2} u_{\varphi \varphi}\right), && 0 \leq \rho<a, \quad t>0, 
    \\
    &u(a, \varphi, t)=0, && t>0, 
    \\
    &u(\rho, \varphi, 0)=1, \quad u_t(\rho, \varphi, 0)=0, && 0 \leq \rho<a .
\end{aligned}\right.
\end{equation}
We solve the equation in terms of Bessel's function.

We first look for separated solutions in the form
$$
u(\rho, \varphi, t)=R(\rho) \Phi(\varphi) T(t).
$$
Inserting into \eqref{eq.vibrating_drumhead}, we get
\[
    \frac{1}{c^2}\frac{T''}{T} = \frac{R''+\frac{1}{\rho} R'}{R} + \frac{1}{\rho^2}\frac{\Phi_{\varphi \varphi}}{\Phi}.
\]

The left (resp. right) hand side is a function of $t$ (resp. $\rho$, $\varphi$). Therefore, they must be a constant independent of all of three variables.
\[
    \frac{1}{c^2}\frac{T''}{T} = -\lambda,\quad \frac{R''+\frac{1}{\rho} R'}{R} + \frac{1}{\rho^2}\frac{\Phi_{\varphi \varphi}}{\Phi} = \lambda.
\]
The second equation can be transformed to 
\[
    \frac{\rho^2 R'' + \rho R'}{R} - \lambda \rho^2 = - \frac{\Phi_{\varphi \varphi}}{\Phi}
\]
Again, the left (resp. right) hand side is a function of $\rho$ (resp. $\varphi$). By introducing the separation constant $\mu$, we have
\[
    \frac{1}{c^2}\frac{T''}{T} = -\lambda,\quad \frac{\rho^2 R'' + \rho R'}{R} - \lambda \rho^2 = \mu, \quad - \frac{\Phi_{\varphi \varphi}}{\Phi} = \mu.
\]

From $u(a, \varphi, t)=0$ and $u_t(\rho, \varphi, 0)=0$, we know that $R(a) = 0$ respectively. Then we obtain
\begin{equation}\label{eq.solve_drumhead_1}
\begin{array}{l}
    \Phi^{\prime \prime}(\varphi)+\mu \Phi(\varphi)=0, \quad \Phi(-\pi)=\Phi(\pi), \quad \Phi^{\prime}(-\pi)=\Phi^{\prime}(\pi) \\
    R^{\prime \prime}(\rho)+\frac{1}{\rho} R^{\prime}(\rho)+\left(\lambda-\frac{\mu}{\rho^2}\right) R(\rho)=0, \quad R(a)=0 \\
    T^{\prime \prime}(t)+\lambda c^2 T(t)=0.
\end{array}
\end{equation}


In the first equation of \eqref{eq.solve_drumhead_1}, nontrivial solutions are obtained when $\sqrt{\mu} = m = 1,2, \cdots$,
$$
\Phi(\varphi)=A \cos m \varphi+B \sin m \varphi, \quad m=0,1,2, \cdots
$$


With $\mu=m^2$ ($m=0,1,2, \cdots$) in the second equation of \eqref{eq.solve_drumhead_1}, we obtain $R(\rho)=J_m(\rho \sqrt{\lambda})$. For $R(a)=0$, we obtain $\sqrt{\lambda}=x_n^{(m)} / a$ where $x_n^{(m)}$ are the nonnegative roots of $J_m(x)=0$. With $\lambda = \left(x_n^{(m)} / a\right)^2$ in the third equation of \eqref{eq.solve_drumhead_1}, we get the solution $T(t) = \left(\bar{A} \cos \frac{c t x_n^{(m)}}{a}+\bar{B} \sin \frac{c t x_n^{(m)}}{a}\right)$. The separated solutions are obtained as
\begin{equation}\label{eq.solve_drumhead_2}
    u(\rho, \varphi, t)=J_m\left(\frac{\rho x_n^{(m)}}{a}\right)(A \cos m \varphi+B \sin m \varphi)\left(\bar{A} \cos \frac{c t x_n^{(m)}}{a}+\bar{B} \sin \frac{c t x_n^{(m)}}{a}\right) .
\end{equation}


We will now take the initial conditions into account. The general solution is given as a linear combination (superposition) of \eqref{eq.solve_drumhead_2}. To satisfy $u_t(\rho, \varphi, 0)=0$, we set $\bar{B}_{}=0$. Now the general solution is written as
\begin{equation}\label{eq.solve_drumhead_3}
    u(\rho, \varphi, t)=\sum_{m=0}^{\infty} \sum_{n=1}^{\infty}\left\{\left[A_{m n} J_m\left(\frac{\rho x_n^{(m)}}{a}\right)\right] \cos m \varphi+\left[B_{m m} J_m\left(\frac{\rho x_n^{(m)}}{a}\right)\right] \sin m \varphi\right\} \cos \frac{c t x_n^{(m)}}{a}
\end{equation}

After matching the general solution with the last boundary condition $u(\rho, \varphi, 0)=1$, we obtain that 
\begin{equation}\label{eq.solve_drumhead_4}
    1 = \sum_{m=0}^{\infty} \sum_{n=1}^{\infty}\left(A_{m n} J_m\left(\frac{\rho x_n^{(m)}}{a}\right) \cos m \varphi+B_{m m} J_m\left(\frac{\rho x_n^{(m)}}{a}\right) \sin m \varphi\right)
\end{equation}

To compute the coefficients $A_{m n}$ (resp. $B_{m n})$, we multiply both sides by $J_m\left(\frac{\rho x_n^{(m)}}{a}\right) \cos m \varphi$ (resp. $J_m\left(\frac{\rho x_n^{(m)}}{a}\right) \sin m \varphi$) and then take integral. By $\frac{1}{\pi} \int_{-\pi}^\pi \cos (m x) d x = \frac{1}{\pi} \int_{-\pi}^\pi \sin (m x) d x = 0$ (if $m\neq 0$), we know that $A_{m n} = B_{m n} = 0$ if $m\neq 0$. Therefore, we take $m = 0$ in \eqref{eq.solve_drumhead_3} and \eqref{eq.solve_drumhead_4},
$$
u(\rho, \varphi, t)=\sum_{n=1}^{\infty} A_{0 n} J_0\left(\frac{\rho x_n^{(0)}}{a}\right) \cos \frac{c t x_n^{(0)}}{a} .
$$
\[
    1 = \sum_{n=1}^{\infty} A_{0 n} J_0\left(\frac{\rho x_n^{(0)}}{a}\right)
\]


In Example \ref{ex.Fourier_Bessel_1}, we calculated the Fourier-Bessel expansion $1=2 \sum_{n=1}^{\infty} \frac{J_0\left(x x_n^{(0)}\right)}{x_n^{(0)} J_1\left(x_n^{(0)}\right)}$. By comparison, we obtain $A_{0n} = \frac{2}{x_n^{(0)} J_1\left(x_n^{(0)}\right)}$, which implies that
$$
u(\rho, \varphi, t)=\sum_{n=1}^{\infty} \frac{2}{x_n^{(0)} J_1\left(x_n^{(0)}\right)} J_0\left(\frac{\rho x_n^{(0)}}{a}\right) \cos \frac{c t x_n^{(0)}}{a} .
$$

\subsection{Heat flow in the infinite cylinder}
Let us consider the heat transfer in the infinite cylinder $0 \leq \rho<a$. We will solve the heat equation in polar coordinates

$$
\left\{\begin{aligned}
&u_t=K \Delta u, && t>0, \quad 0 \leq \rho<a, \quad-\pi \leq \varphi \leq \pi, 
\\
&u\left(a, \varphi, t\right)=T_1, && t>0, \quad-\pi \leq \varphi \leq \pi, 
\\
&u(\rho, \varphi, 0)=T_2, && 0 \leq \rho<a, \quad-\pi \leq \varphi \leq \pi,
\end{aligned}\right.
$$
where $T_1$ and $T_2$ are positive constants.

We first the steady-state solution. Let us try $U(\rho)$ because the boundary conditions is independent of $\varphi$.
$$
K \Delta U=K\left(\frac{\partial^2 U}{\partial \rho^2}+\frac{1}{\rho} \frac{\partial U}{\partial \rho}+\frac{1}{\rho^2} \frac{\partial^2 U}{\partial \varphi^2}\right)=K\left(U^{\prime \prime}+\frac{1}{\rho} U^{\prime}\right)=0 .
$$
The general solution is obtained as
$$
U(\rho)=A+B \ln \rho .
$$
Let us exclude the second term and set $B=0$ (otherwise $U(0)$ diverges). To satisfy $U\left(a\right)=T_1$, we choose $A=T_1$. We thus obtain
$$
U(\rho)=T_1 .
$$

Define $v(\rho, \varphi, t)=u(\rho, \varphi, t)-U(\rho)$. We have
$$
\left\{\begin{aligned}
&v_t=K \Delta v, && t>0, \quad 0 \leq \rho<a, \quad-\pi \leq \varphi \leq \pi, 
\\
&v\left(a, \varphi, t\right)=0, && t>0, \quad-\pi \leq \varphi \leq \pi, 
\\
&v(\rho, \varphi, 0)=T_2-T_1, && 0 \leq \rho<a, \quad-\pi \leq \varphi \leq \pi,
\end{aligned}\right.
$$

Using separation of variables with $u(\rho, \varphi, t)=R(\rho) \Phi(\varphi) T(t)$, we obtain
Inserting into \eqref{eq.vibrating_drumhead}, we get
\[
    \frac{1}{K}\frac{T'}{T} = \frac{R''+\frac{1}{\rho} R'}{R} + \frac{1}{\rho^2}\frac{\Phi_{\varphi \varphi}}{\Phi}.
\]

The left (resp. right) hand side is a function of $t$ (resp. $\rho$, $\varphi$). Therefore, they must be a constant independent of all of three variables.
\[
    \frac{1}{K}\frac{T'}{T} = -\lambda,\quad \frac{R''+\frac{1}{\rho} R'}{R} + \frac{1}{\rho^2}\frac{\Phi_{\varphi \varphi}}{\Phi} = \lambda.
\]
The second equation can be transformed to 
\[
    \frac{\rho^2 R'' + \rho R'}{R} - \lambda \rho^2 = - \frac{\Phi_{\varphi \varphi}}{\Phi}
\]
Again, the left (resp. right) hand side is a function of $\rho$ (resp. $\varphi$). By introducing the separation constant $\mu$, we have
\[
    \frac{1}{K}\frac{T'}{T} = -\lambda,\quad \frac{\rho^2 R'' + \rho R'}{R} - \lambda \rho^2 = \mu, \quad - \frac{\Phi_{\varphi \varphi}}{\Phi} = \mu.
\]

From $u(a, \varphi, t)=0$ and $u_t(\rho, \varphi, 0)=0$, we know that $R(a) = 0$ respectively. Then we obtain
\begin{equation}\label{eq.solve_heat_disk_1}
\begin{array}{l}
    \Phi^{\prime \prime}(\varphi)+\mu \Phi(\varphi)=0, \quad \Phi(-\pi)=\Phi(\pi), \quad \Phi^{\prime}(-\pi)=\Phi^{\prime}(\pi) \\
    R^{\prime \prime}(\rho)+\frac{1}{\rho} R^{\prime}(\rho)+\left(\lambda-\frac{\mu}{\rho^2}\right) R(\rho)=0, \quad R(a)=0 \\
    T'(t)+\lambda K T(t)=0.
\end{array}
\end{equation}


In the first equation of \eqref{eq.solve_drumhead_1}, nontrivial solutions are obtained when $\sqrt{\mu} = m = 1,2, \cdots$,
$$
\Phi(\varphi)=A \cos m \varphi+B \sin m \varphi, \quad m=0,1,2, \cdots
$$


With $\mu=m^2$ ($m=0,1,2, \cdots$) in the second equation of \eqref{eq.solve_drumhead_1}, we obtain $R(\rho)=J_m(\rho \sqrt{\lambda})$. For $R(a)=0$, we obtain $\sqrt{\lambda}=x_n^{(m)} / a$ where $x_n^{(m)}$ are the nonnegative roots of $J_m(x)=0$. With $\lambda = \left(x_n^{(m)} / a\right)^2$ in the third equation of \eqref{eq.solve_drumhead_1}, we get the solution $T(t) = \exp \left[-\frac{\left(x_n^{(m)}\right)^2 K t}{a^2}\right]$. The separated solutions are obtained as
\begin{equation}\label{eq.solve_heat_disk_2}
    v(\rho, \varphi, t)=J_m\left(\frac{\rho x_n^{(m)}}{a}\right)(A \cos m \varphi+B \sin m \varphi)\exp \Bigg(-\frac{\left(x_n^{(m)}\right)^2 K t}{a^2}\Bigg).
\end{equation}

The general solutions are obtained as
\begin{equation}\label{eq.solve_heat_disk_3}
    v(\rho, \varphi, t)=\sum_{m=0}^{\infty} \sum_{n=1}^{\infty} J_m\left(\frac{\rho x_n^{(m)}}{a}\right)\left(A_{m n} \cos m \varphi+B_{m n} \sin m \varphi\right) \exp \Bigg(-\frac{\left(x_n^{(m)}\right)^2 K t}{a^2}\Bigg).
\end{equation}

After matching the general solution with the last boundary condition $u(\rho, \varphi, 0)=T_2-T_1$, we obtain that 
\begin{equation}\label{eq.solve_heat_disk_4}
    T_2-T_1 = \sum_{m=0}^{\infty} \sum_{n=1}^{\infty}\left(A_{m n} J_m\left(\frac{\rho x_n^{(m)}}{a}\right) \cos m \varphi+B_{m m} J_m\left(\frac{\rho x_n^{(m)}}{a}\right) \sin m \varphi\right)
\end{equation}

To compute the coefficients $A_{m n}$ (resp. $B_{m n})$, we multiply both sides by $J_m\left(\frac{\rho x_n^{(m)}}{a}\right) \cos m \varphi$ (resp. $J_m\left(\frac{\rho x_n^{(m)}}{a}\right) \sin m \varphi$) and then take integral. By $\frac{1}{\pi} \int_{-\pi}^\pi \cos (m x) d x = \frac{1}{\pi} \int_{-\pi}^\pi \sin (m x) d x = 0$ (if $m\neq 0$), we know that $A_{m n} = B_{m n} = 0$ if $m\neq 0$. Therefore, we take $m = 0$ in \eqref{eq.solve_heat_disk_3} and \eqref{eq.solve_heat_disk_4},
$$
u(\rho, \varphi, t)=\sum_{n=1}^{\infty} A_{0 n} J_0\left(\frac{\rho x_n^{(0)}}{a}\right) \exp \Bigg(-\frac{\left(x_n^{(0)}\right)^2 K t}{a^2}\Bigg).
$$
\[
    T_2-T_1 = \sum_{n=1}^{\infty} A_{0 n} J_0\left(\frac{\rho x_n^{(0)}}{a}\right)
\]

Noting that $1=2 \sum_{n=1}^{\infty} \frac{J_0\left(x x_n^{(m)}\right)}{x_n J_1\left(x_n\right)}$ $(0<x<1, J_0\left(x_n^{(0)}\right)=0)$, we get $A_{0n} = \frac{2(T_2-T_1)}{x_n^{(0)} J_1\left(x_n^{(0)}\right)}$, which implies that 
$$
u(\rho, \varphi, t)=T_1+\sum_{n=1}^{\infty} \frac{2(T_2-T_1)}{x_n^{(0)} J_1\left(x_n^{(0)}\right)} J_0\left(\frac{\rho x_n^{(0)}}{a}\right) \exp \Bigg(-\frac{\left(x_n^{(0)}\right)^2 K t}{a^2}\Bigg).
$$