\section{PDEs in cylindrical coordinates}

\subsection{Laplace equation in cylindrical coordinates}

In this section, we will discuss how to solve the Laplace equation $\Delta u = 0$ in cylindrical coordinates. The operator $\Delta$, often refered to as Laplacian, is defined by the following
\begin{equation}\label{eq.Laplacian}
    \Delta = \partial_{xx} + \partial_{yy} \quad \textrm{(in 2D)},\qquad\qquad \Delta = \partial_{xx} + \partial_{yy} + \partial_{zz} \quad \textrm{(in 3D)}
\end{equation}

We consider the cylindrical domain $\{(x, y, z): R_1^2\le x^2 + y^2 \le R_2^2,\,, z\in [-L, L]\}$, for which cylindrical coordinates $(\rho, \varphi, z)$ are particularly useful. The transformation from cartesian coordinates $(x, y, z)$ to cylindrical coordinates can simplify the domain to $\{(x, y, z): R_1\le r \le R_2,\,\varphi\in [0, 2\pi],\,z\in [-L, L]\}$.

The relationship between Cartesian coordinates $(x, y, z)$ and cylindrical coordinates $(\rho, \varphi, z)$ is given by
\begin{equation}\label{eq.from_cartesian_to_cylindrical}
    x = \rho \cos(\varphi), \quad y = \rho \sin(\varphi), \quad z = z.
\end{equation}

Additionally, we have the inverse relationships,
\begin{equation}\label{eq.from_cylindrical_to_cartesian}
    \rho = \sqrt{x^2 + y^2}, \quad \varphi = \arctan\left(\frac{y}{x}\right), \quad z = z.
\end{equation}

To convert PDEs from Cartesian to cylindrical coordinates, we need to express the partial derivatives with respect to $x$ and $y$ in terms of $\rho$, $\varphi$, and $z$.

\begin{proposition}[]
The first-order partial derivatives transform as,
\begin{equation}\label{eq.Laplace_1st_order}
    \begin{gathered}
        \partial_x = \cos(\varphi) \partial_\rho - \frac{\sin(\varphi)}{\rho} \partial_\varphi
        \\
        \partial_y = \sin(\varphi) \partial_\rho + \frac{\cos(\varphi)}{\rho} \partial_\varphi
        % \\
        % \frac{\partial f}{\partial z} = \frac{\partial f}{\partial z}
    \end{gathered}
\end{equation}
\end{proposition}
\begin{proof} Using the chain rule, we have
\begin{equation}\label{eq.proof_Laplace_1st_order_1}
    \begin{gathered}
        f_x = f_\rho \rho_x - f_\varphi \varphi_x
        \\
        f_y = f_\rho \rho_y + f_\varphi \varphi_y
    \end{gathered}
\end{equation}
By \eqref{eq.from_cylindrical_to_cartesian}, we can compute that
\begin{equation}
    \rho_x = \cos(\varphi), \quad \rho_y = \sin(\varphi), \quad \varphi_x = - \frac{\sin(\varphi)}{\rho}, \quad \rho_x = \frac{\cos(\varphi)}{\rho}
\end{equation}
Inserting into \eqref{eq.proof_Laplace_1st_order_1}, we get
\begin{equation}
    \begin{gathered}
        f_x = \cos(\varphi) f_\rho - \frac{\sin(\varphi)}{\rho} f_\varphi
        \\
        f_y = \sin(\varphi) f_\rho + \frac{\cos(\varphi)}{\rho} f_\varphi
        \\
        \frac{\partial f}{\partial z} = \frac{\partial f}{\partial z}
    \end{gathered}
\end{equation}
This implies \eqref{eq.Laplace_1st_order}.
\end{proof}


\begin{proposition}[]
The Laplacian transforms as
\begin{equation}\label{eq.Laplace_2nd_order}
    \Delta f = f_{xx} + f_{yy} = \frac{1}{\rho} ( \rho f_\rho)_\rho + \frac{1}{\rho^2} f_{\varphi\varphi}.
\end{equation}
\end{proposition}
\begin{proof} By \eqref{eq.Laplace_1st_order}, we get
We can compute that
\begin{equation}
\begin{split}
    f_{xx}=&\left(\cos(\varphi) \partial_\rho - \frac{\sin(\varphi)}{\rho} \partial_\varphi\right)\left(\cos(\varphi) \partial_\rho - \frac{\sin(\varphi)}{\rho} \partial_\varphi\right)f,
    \\
    =&\cos ^2 \varphi\, f_{\rho\rho}+\frac{2 \cos \varphi \sin \varphi}{\rho^2} f_{\varphi}-\frac{2 \sin \varphi \cos \varphi}{\rho} f_{\rho\varphi}+\frac{\sin ^2 \varphi}{\rho} f_{\rho}+\frac{\sin ^2 \varphi}{\rho^2} f_{\varphi\varphi}. 
\end{split}
\end{equation}
\begin{equation}
\begin{split}
    f_{yy}=&\left(\cos(\varphi) \partial_\rho - \frac{\sin(\varphi)}{\rho} \partial_\varphi\right)\left(\cos(\varphi) \partial_\rho - \frac{\sin(\varphi)}{\rho} \partial_\varphi\right)f,
    \\
    =&\sin ^2 \varphi\, f_{\rho\rho}-\frac{2 \sin \varphi \cos \varphi}{\rho^2} f_{\varphi}+\frac{2 \sin \varphi \cos \varphi}{\rho} f_{\rho\varphi}+\frac{\cos ^2 \varphi}{\rho} f_{\rho}+\frac{\cos ^2 \varphi}{\rho^2} f_{\varphi\varphi}.
\end{split}
\end{equation}

Summing the above equations, we get
\begin{equation}
    \Delta f = \partial_{xx} f + \partial_{yy} f = \frac{1}{\rho} ( \rho f_\rho)_\rho + \frac{1}{\rho^2} f_{\varphi\varphi}.
\end{equation}
This completes the proof.
\end{proof}


This expression is crucial for solving PDEs such as the Laplace equation or the wave equation in cylindrical coordinates.

\subsection{Separation of Variables for the 2D Laplace equation}
Separation of variables is a common technique used to solve PDEs, especially in systems with boundary conditions. Let's consider solving the equation
\begin{equation} 
    \Delta u = 0, \quad R_1 \leq \sqrt{x^2+y^2} \leq R_2
\end{equation}
where $\Delta$ is the Laplacian operator in cylindrical coordinates.

\subsubsection{The case of $R_1 = 0$ and $R_2 = R$}
Let us find separated solutions of Laplace's equation $\Delta u = 0$ on the domain $0 \leq \sqrt{x^2+y^2} \leq R$. In terms of cylindrical coordinates, the domain becomes $0\le \rho \le R$, $-\pi \leq \varphi \leq \pi$ and the Laplacian becomes $\frac{1}{\rho} ( \rho u_\rho)_\rho + \frac{1}{\rho^2} u_{\varphi\varphi}$. Assume that $u$ is smooth and is independent of $z$. The boundary value problem should be written as
\begin{equation}\label{eq.boundary_value_cylindrical_disk}
\left\{\begin{aligned}
    &\Delta u = \frac{1}{\rho} ( \rho u_\rho)_\rho + \frac{1}{\rho^2} u_{\varphi\varphi} = 0, \quad && 0\leq \rho \leq R, \quad -\pi \leq \varphi \leq \pi
    \\
    &u(R, \varphi)=u_2(\varphi), && -\pi \leq \varphi \leq \pi. 
\end{aligned}\right.
\end{equation}

By plugging $u(\rho, \varphi)=R(\rho) \Phi(\varphi)$ into $\Delta u=0$, we obtain
$$
0=\frac{1}{\rho} ( \rho u_\rho)_\rho+\frac{1}{\rho^2} u_{\varphi \varphi}=\frac{1}{\rho}(\rho R^{\prime})^{\prime} \Phi+\frac{1}{\rho^2} R \Phi^{\prime \prime}
$$

Dividing by $R \Phi$ and multiplying by $\rho^2$, we have
$$
\rho \frac{(\rho R^{\prime})^{\prime}}{R} = - \frac{\Phi^{\prime \prime}}{\Phi}
$$


By introducing the separation constant $\lambda$, we have
\begin{equation}\label{eq.Laplace_separation_var}
\left\{\begin{aligned}
    &\Phi^{\prime \prime}+\lambda \Phi=0, \quad \Phi(-\pi)=\Phi(\pi), \quad \Phi^{\prime}(-\pi)=\Phi^{\prime}(\pi), 
    \\
    &\rho (\rho R^{\prime})^{\prime} - \lambda R = 0.
\end{aligned}
\right.
\end{equation}

Solve the eigenvalue and eigenfunction $\lambda$ and $\Phi$. We get that $\lambda=m^2$ and $\Phi_m(\varphi)=A_m \cos m \varphi+B_m \sin m \varphi$ ($m=0,1,2, \ldots$). 

Given the expression $\lambda=m^2$ for $\lambda$, we can also solve $R(\rho)$. First inserting $\lambda=m^2$ and the change of variable $\rho = e^s$, the second equation becomes $R''(s) - m^2 R(s) = 0$. The general solution is given by $R(s) = Ae^{ms} + Be^{-ms}$. Transforming back to $\rho$ variable, we get that $R_m(\rho) = C_m\rho^m + D_m\rho^{-m}$. The case of $m = 0$ is special, where the solution can be easily solved as $R_0(\rho) = C_0+D_0 \ln \rho$.

Therefore, the separated solutions are obtained as
\begin{equation}
u(\rho, \varphi)=\left\{
\begin{aligned}
    &\left(C_m\rho^m + D_m\rho^{-m}\right)\left(A_m \cos m \varphi+B_m \sin m \varphi\right), && m=1,2, \ldots, 
    \\
    &C_0+D_0 \ln \rho, && m=0 
\end{aligned}
\right.
\end{equation}


If $D_m\neq 0$ for some $m\ge 0$, we have $|u| \rightarrow \infty$ as $\rho \rightarrow 0$ and $u$ is not smooth. Therefore,
$$
u(\rho, \varphi)=\rho^m\left(A_m \cos m \varphi+B_m \sin m \varphi\right), \quad m=0,1,2, \ldots
$$
where we take $C_m = 1$ as we can merge $C_m$ with $A_m$ and $B_m$. For the case of $m = 0$, $u(\rho, \varphi)$ is constant according to the above expression.

Therefore, we can write the general solution
$$
u(\rho, \varphi)=\sum_{m = 0}^\infty\rho^m\left(A_m \cos m \varphi+B_m \sin m \varphi\right), \quad m=0,1,2, \ldots
$$

Finally, to decide the coefficients $A_m$ and $B_m$, we use the boundary condition in \eqref{eq.boundary_value_cylindrical_disk}.
$$
u_2(\varphi) = u(R, \varphi)=\sum_{m = 0}^\infty R^m\left(A_m \cos m \varphi+B_m \sin m \varphi\right), \quad m=0,1,2, \ldots
$$

Therefore, $A_mR^m$ and $B_mR^m$ are Fourier coefficients of the function $u_2(\varphi)$, which can be computed using the formula of coefficients \eqref{eq.fourier_coefficient}. 

\subsubsection{The general case}
Let us find separated solutions of Laplace's equation $\Delta u = 0$ on the domain $R_1 \leq \sqrt{x^2+y^2} \leq R_2$. In terms of cylindrical coordinates, the domain becomes $R_1\le \rho \le R_2$, $-\pi \leq \varphi \leq \pi$ and the Laplacian becomes $\frac{1}{\rho} ( \rho u_\rho)_\rho + \frac{1}{\rho^2} u_{\varphi\varphi}$. Assume that $u$ is smooth and is independent of $z$. The boundary value problem should be written as
\begin{equation}\label{eq.boundary_value_cylindrical_annulus}
\left\{\begin{aligned}
    &\Delta u = \frac{1}{\rho} ( \rho u_\rho)_\rho + \frac{1}{\rho^2} u_{\varphi\varphi} = 0, \quad && R_1\le \rho \le R_2, \quad -\pi \leq \varphi \leq \pi
    \\
    &u(R_2, \varphi)=u_2(\varphi),\quad u(R_2, \varphi)=u_2(\varphi),\qquad && -\pi \leq \varphi \leq \pi. 
\end{aligned}\right.
\end{equation}

By plugging $u(\rho, \varphi)=R(\rho) \Phi(\varphi)$ into $\Delta u=0$, we obtain
$$
0=\frac{1}{\rho} ( \rho u_\rho)_\rho+\frac{1}{\rho^2} u_{\varphi \varphi}=\frac{1}{\rho}(\rho R^{\prime})^{\prime} \Phi+\frac{1}{\rho^2} R \Phi^{\prime \prime}
$$

Dividing by $R \Phi$ and multiplying by $\rho^2$, we have
$$
\rho \frac{(\rho R^{\prime})^{\prime}}{R} = - \frac{\Phi^{\prime \prime}}{\Phi}
$$


By introducing the separation constant $\lambda$, we have
\begin{equation}\label{eq.Laplace_separation_var}
\left\{\begin{aligned}
    &\Phi^{\prime \prime}+\lambda \Phi=0, \quad \Phi(-\pi)=\Phi(\pi), \quad \Phi^{\prime}(-\pi)=\Phi^{\prime}(\pi), 
    \\
    &\rho (\rho R^{\prime})^{\prime} - \lambda R = 0.
\end{aligned}
\right.
\end{equation}

Solve the eigenvalue and eigenfunction $\lambda$ and $\Phi$. We get that $\lambda=m^2$ and $\Phi_m(\varphi)=A_m \cos m \varphi+B_m \sin m \varphi$ ($m=0,1,2, \ldots$). 

Given the expression $\lambda=m^2$ for $\lambda$, we can also solve $R(\rho)$. First inserting $\lambda=m^2$ and the change of variable $\rho = e^s$, the second equation becomes $R''(s) - m^2 R(s) = 0$. The general solution is given by $R(s) = Ae^{ms} + Be^{-ms}$. Transforming back to $\rho$ variable, we get that $R_m(\rho) = C_m\rho^m + D_m\rho^{-m}$. The case of $m = 0$ is special, where the solution can be easily solved as $R_0(\rho) = C_0+D_0 \ln \rho$.

Therefore, the separated solutions are obtained as
\begin{equation}
u(\rho, \varphi)=\left\{
\begin{aligned}
    &\left(C_m\rho^m + D_m\rho^{-m}\right)\left(A_m \cos m \varphi+B_m \sin m \varphi\right), && m=1,2, \ldots, 
    \\
    &C_0+D_0 \ln \rho, && m=0 
\end{aligned}
\right.
\end{equation}


Everything up to this point is the same as the case of $R_1 = 0$, $R_2 = R$. For general annulus domain, we do not have the condition that $|u| < \infty$ as $\rho \rightarrow 0$. Therefore, $D_m$ are not necessarily vanishing and the general solution can be written as
$$
u(\rho, \varphi)=\sum_{m = 0}^\infty\left(C_m\rho^m + D_m\rho^{-m}\right)\left(A_m \cos m \varphi+B_m \sin m \varphi\right), \quad m=0,1,2, \ldots
$$

Finally, to decide the coefficients $A_m$, $B_m$, $C_m$ and $D_m$, we use the boundary condition in \eqref{eq.boundary_value_cylindrical_disk}.
$$
u_1(\varphi) = u(R_1, \varphi)=\sum_{m = 0}^\infty \left(C_mR_1^m + D_mR_1^{-m}\right)\left(A_m \cos m \varphi+B_m \sin m \varphi\right), \quad m=0,1,2, \ldots
$$
$$
u_2(\varphi) = u(R_2, \varphi)=\sum_{m = 0}^\infty \left(C_mR_2^m + D_mR_2^{-m}\right)\left(A_m \cos m \varphi+B_m \sin m \varphi\right), \quad m=0,1,2, \ldots
$$

Therefore, $\alpha^1_m = A_m(C_mR_1^m + D_mR_1^{-m})$ and $\beta^1_m = B_m(C_mR_1^m + D_mR_1^{-m})$ are Fourier coefficients of the function $u_1(\varphi)$; $\alpha^2_m = A_m(C_mR_2^m + D_mR_2^{-m})$ and $\beta^2_m = B_m(C_mR_2^m + D_mR_2^{-m})$ are Fourier coefficients of the function $u_2(\varphi)$, which can be computed using the formula of coefficients \eqref{eq.fourier_coefficient}. After solving $\alpha^1_m$, $\beta^1_m$, $\alpha^2_m$, $\beta^2_m$, we can solve $A_m$, $B_m$, $C_m$ and $D_m$ from them.

\subsection{Eigenfunctions of Laplacian and the Bessel's function}

6. Bessel Functions

When solving PDEs in cylindrical coordinates, Bessel functions often arise. Consider the differential equation:

\[
y'' + \frac{1}{x} y' + \left( 1 - \frac{m^2}{x^2} \right) y = 0
\]

This is Bessel's equation, and its solutions are known as Bessel functions, denoted by $ J_m(x) $.

The general solution to the radial part of many problems can be written as:

\[
R(\rho) = J_m(\lambda \rho)
\]

where $ J_m(\lambda \rho) $ is the Bessel function of the first kind.

---

7. Conclusion

Cylindrical coordinates provide a powerful framework for solving PDEs, particularly in systems with radial symmetry. The method of separation of variables, along with Bessel functions, allows for elegant solutions to complex problems. By transforming Cartesian coordinates to cylindrical, we simplify the equations and exploit the symmetry of the problem.

\subsection{The vibrating drumhead}

\subsection{Heat flow in the infinite cylinder}