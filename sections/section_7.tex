\section{Fourier transform and PDEs on unbounded domains}

\subsection{The Fourier transform}

\subsubsection{Motivation: from Fourier series to Fourier transform}

Recall the complex form of the Fourier series on $[-L,L]$:
\[
  f(x) = \sum_{n=-\infty}^{\infty} c_n\,e^{in\pi x/L},
  \qquad
  c_n = \frac{1}{2L}\int_{-L}^{L} f(x')\,e^{-in\pi x'/L}\,dx'.
\]
Substituting $c_n$ back into the expansion gives
\[
  f(x)
  = \sum_{n=-\infty}^{\infty}
    \frac{1}{2L}\int_{-L}^{L} f(x')\,e^{-in\pi x'/L}\,dx'
    \;\cdot\; e^{in\pi x/L}.
\]
Define the discrete frequencies
\[
  p_n = \frac{n\pi}{L}, \qquad \Delta p = p_{n+1}-p_n = \frac{\pi}{L}.
\]
Then we can rewrite the Fourier series as a Riemann sum:
\[
  f(x)
  = \sum_{n=-\infty}^{\infty}
    \frac{1}{2\pi}
    \underbrace{\int_{-L}^{L} f(x')\,e^{-ip_n x'}\,dx'}_{\displaystyle\to\;\hat{f}(p_n)}
    \;\cdot\; e^{ip_n x}\;\Delta p.
\]
As $L\to\infty$, $\Delta p\to 0$ and the discrete variable $p_n$ becomes a continuous variable $p$. The Riemann sum converges to an integral:
\begin{equation}\label{eq.ft_limit}
  f(x)
  = \frac{1}{2\pi}\int_{-\infty}^{\infty}
    \!\left(\int_{-\infty}^{\infty} f(x')\,e^{-ipx'}\,dx'\right)
    e^{ipx}\,dp.
\end{equation}
This motivates the following definition.

\subsubsection{Definition and properties}

\begin{definition}[Fourier transform]\label{def.fourier_transform}
Let $f(x)$ be a sufficiently fast-decaying function defined on $\mathbb{R}$. The \underline{Fourier transform} of $f$ is defined by
\begin{equation}\label{eq.fourier_transform}
  \hat{f}(p) = \int_{-\infty}^{\infty} f(x)\,e^{-ipx}\,dx.
\end{equation}
The \underline{inverse Fourier transform} is given by
\begin{equation}\label{eq.inverse_fourier_transform}
  f(x) = \frac{1}{2\pi}\int_{-\infty}^{\infty} \hat{f}(p)\,e^{ipx}\,dp.
\end{equation}
\end{definition}

\begin{theorem}[Properties of the Fourier transform]\label{th.fourier_transform_properties}
Let $f(x)$ be a function with Fourier transform $\hat{f}(p)$. The following properties hold.
\begin{enumerate}
  \item \textbf{Derivative property:} $\widehat{f'}(p) = ip\,\hat{f}(p)$, provided $f(x) \to 0$ as $|x| \to \infty$.
  \item \textbf{Multiplication by $x$:} $\widehat{xf}(p) = i\,\hat{f}'(p)$.
  \item \textbf{Parseval's identity:}
    $\displaystyle\int_{-\infty}^{\infty} |f(x)|^2\,dx
      = \frac{1}{2\pi}\int_{-\infty}^{\infty} |\hat{f}(p)|^2\,dp.$
  \item \textbf{Generalized Parseval's identity:}
    $\displaystyle\int_{-\infty}^{\infty}f(x)\overline{g(x)}\,dx
        = \frac{1}{2\pi}\int_{-\infty}^{\infty}\hat{f}(p)\overline{\hat{g}(p)}\,dp.$
\end{enumerate}
\end{theorem}

\begin{proof}[Proof of (1)]
We compute the Fourier transform of $f'$ by integration by parts:
\begin{equation}\label{eq.proof_ft_derivative}
  \widehat{f'}(p)
  = \int_{-\infty}^{\infty} f'(x)\,e^{-ipx}\,dx
  = \bigl[f(x)\,e^{-ipx}\bigr]_{-\infty}^{\infty}
    + ip\int_{-\infty}^{\infty} f(x)\,e^{-ipx}\,dx
  = ip\,\hat{f}(p),
\end{equation}
where the boundary term vanishes since $f(x)\to 0$ as $|x|\to\infty$.
\end{proof}

\begin{proof}[Proof of (2)]
Differentiating $\hat{f}(p) = \int_{-\infty}^{\infty} f(x)\,e^{-ipx}\,dx$ with respect to $p$ under the integral sign gives
\begin{equation}\label{eq.proof_ft_multiplication}
  \hat{f}'(p)
  = \frac{d}{dp}\int_{-\infty}^{\infty} f(x)\,e^{-ipx}\,dx
  = \int_{-\infty}^{\infty} f(x)\,(-ix)\,e^{-ipx}\,dx
  = -i\,\widehat{xf}(p).
\end{equation}
Therefore $\widehat{xf}(p) = i\,\hat{f}'(p)$.
\end{proof}

An important consequence of property (1) is the following.

\begin{corollary}\label{cor.second_derivative_ft}
If $f(x) \to 0$ and $f'(x) \to 0$ as $|x| \to \infty$, then $\widehat{f''}(p) = (ip)^2\hat{f}(p) = -p^2\hat{f}(p)$.
\end{corollary}

We also record the following property of real-valued functions.

\begin{proposition}\label{prop.ft_real}
If $f(x)$ is real-valued, then $\hat{f}(p) = \overline{\hat{f}(-p)}$.
\end{proposition}

\subsection{Heat equation on the full line}

Consider the heat equation on $\mathbb{R}$:
\begin{equation}\label{eq.heat_eq_R}
  u_t = k\,u_{xx},\quad x\in\mathbb{R},\quad t>0,
  \qquad u(x,0) = f(x).
\end{equation}

\subsubsection{Solution by Fourier transform}

We apply the Fourier transform in $x$ to both sides of \eqref{eq.heat_eq_R}. By Corollary~\ref{cor.second_derivative_ft}, the right-hand side transforms as $k\,\widehat{u_{xx}} = -kp^2\,\hat{u}$. Since the Fourier transform does not act on $t$, we obtain the ODE
\begin{equation}\label{eq.heat_eq_R_ft}
  \hat{u}_t = -kp^2\,\hat{u}.
\end{equation}

This is a separable first-order ODE in $t$ (with $p$ as a parameter). Solving by separation of variables,
\begin{equation}\label{eq.heat_eq_R_ft_solve}
  \frac{d\hat{u}}{dt} = -kp^2\,\hat{u}
  \quad\Longrightarrow\quad
  \ln\hat{u} = -kp^2\,t + C
  \quad\Longrightarrow\quad
  \hat{u}(p,t) = \hat{u}(p,0)\,e^{-kp^2 t}.
\end{equation}
The initial condition gives $\hat{u}(p,0) = \hat{f}(p)$, so
\begin{equation}\label{eq.heat_eq_R_ft_sol}
  \hat{u}(p,t) = \hat{f}(p)\,e^{-kp^2 t}.
\end{equation}

Applying the inverse Fourier transform \eqref{eq.inverse_fourier_transform},
\begin{equation}\label{eq.heat_eq_R_sol_ft}
  u(x,t) = \frac{1}{2\pi}\int_{-\infty}^{\infty}
    \hat{f}(p)\,e^{-kp^2 t}\,e^{ipx}\,dp.
\end{equation}

\subsubsection{The heat kernel}\label{sec.heat_kernel}

We can simplify \eqref{eq.heat_eq_R_sol_ft} into a convolution form. Substituting the definition $\hat{f}(p) = \int_{-\infty}^{\infty}f(x')\,e^{-ipx'}\,dx'$ and interchanging the order of integration,
\begin{equation}\label{eq.heat_kernel_derivation}
  u(x,t) = \frac{1}{2\pi}\int_{-\infty}^{\infty}f(x')
    \int_{-\infty}^{\infty}e^{-kp^2 t + ip(x-x')}\,dp\;dx'.
\end{equation}

To evaluate the inner integral, we use the following Gaussian integral formula.

\begin{theorem}[Gaussian integral]\label{th.gaussian_integral}
\begin{equation}\label{eq.gaussian_integral}
  \int_{-\infty}^{\infty} e^{-\alpha p^2+\beta p}\,dp
  = \sqrt{\frac{\pi}{\alpha}}\;e^{\beta^2/(4\alpha)},
  \qquad \operatorname{Re}(\alpha)>0.
\end{equation}
\end{theorem}

We apply Theorem~\ref{th.gaussian_integral} with $\alpha = kt$ and $\beta = i(x-x')$, so that $\beta^2 = -(x-x')^2$:
\begin{equation}\label{eq.heat_kernel_gaussian}
  \int_{-\infty}^{\infty}e^{-ktp^2+i(x-x')p}\,dp
  = \sqrt{\frac{\pi}{kt}}\;e^{-(x-x')^2/(4kt)}.
\end{equation}

Substituting back into \eqref{eq.heat_kernel_derivation},
\begin{align}
  u(x,t)
  &= \frac{1}{2\pi}\int_{-\infty}^{\infty}f(x')\;
     \sqrt{\frac{\pi}{kt}}\;e^{-(x-x')^2/(4kt)}\,dx' \nonumber\\
  &= \frac{1}{\sqrt{4\pi kt}}\int_{-\infty}^{\infty}
     f(x')\,e^{-(x-x')^2/(4kt)}\,dx'. \label{eq.heat_kernel_sol}
\end{align}

We define the \underline{heat kernel} (also called the \underline{Green's function} for the heat equation on $\mathbb{R}$):
\begin{equation}\label{eq.heat_kernel}
  \boxed{G(x,x';t) = \frac{1}{\sqrt{4\pi kt}}\,e^{-(x-x')^2/(4kt)}.}
\end{equation}
The solution to the heat equation \eqref{eq.heat_eq_R} is then
\begin{equation}\label{eq.heat_sol_green}
  u(x,t) = \int_{-\infty}^{\infty}G(x,x';t)\,f(x')\,dx'.
\end{equation}

\subsubsection{Connection to eigenfunction expansion}

Consider the heat equation $u_t = k\,u_{xx}$ on $x\in\mathbb{R}$. Define the operator $A = -\partial_{xx}$ acting on bounded functions $\phi(x)$ on $\mathbb{R}$. The eigenvalue problem $A\phi = \lambda\phi$, i.e.\ $-\phi'' = \lambda\phi$, has solutions
\begin{equation}\label{eq.eigenvalue_full_line}
  \phi(x) = e^{ipx},\qquad \lambda = p^2, \qquad p\in\mathbb{R}.
\end{equation}
The eigenfunctions $e^{ipx}$ are labeled by a continuous parameter $p$, and the eigenvalues satisfy $\lambda = p^2 \ge 0$. This is called the \underline{continuous spectrum}, in contrast to the discrete spectrum that arises on bounded domains. The ``eigenfunction expansion'' becomes
\begin{equation}\label{eq.eigenfunction_expansion_ft}
  u(x,t) = \int_{-\infty}^{\infty}\hat{u}(p,t)\,e^{ipx}\,\frac{dp}{2\pi},
\end{equation}
which is exactly the inverse Fourier transform.

\subsection{Fourier sine and cosine transforms}

The Fourier transform requires that $f(x)$ is defined on all of $\mathbb{R}$ and decays sufficiently fast as $|x| \to \infty$. In many applications, we need to solve PDEs on the half-line $x > 0$. In this case, we extend $f$ to the whole line and use the Fourier sine or cosine transform.

\subsubsection{Heat equation on the half-line}

Consider the heat equation on the half-line:
\begin{equation}\label{eq.heat_eq_half_line}
  u_t = k\,u_{xx},\quad x>0,\quad t>0,\qquad u(x,0) = f(x),
\end{equation}
together with one of the following boundary conditions:
\begin{itemize}
  \item \textbf{Dirichlet boundary condition:} $u(0,t)=0$.
  \item \textbf{Neumann boundary condition:} $u_x(0,t)=0$.
\end{itemize}

Since the Fourier transform requires $u(x,t)$ to be defined on $x\in(-\infty,\infty)$, we need to extend $f$ from $x > 0$ to all of $\mathbb{R}$.

\subsubsection{Odd and even extensions}

The \underline{odd extension} of $f(x)$ (for Dirichlet boundary conditions) is defined by
\begin{equation}\label{eq.odd_extension_ft}
  f_o(x) =
  \begin{cases}
    f(x),   & x>0,\\
    -f(-x), & x<0.
  \end{cases}
\end{equation}

The \underline{even extension} of $f(x)$ (for Neumann boundary conditions) is defined by
\begin{equation}\label{eq.even_extension_ft}
  f_e(x) =
  \begin{cases}
    f(x),  & x>0,\\
    f(-x), & x<0.
  \end{cases}
\end{equation}

The key observation is:
\begin{itemize}
  \item If $f$ is odd, then $\hat{f}(p)$ is purely imaginary and odd.
  \item If $f$ is even, then $\hat{f}(p)$ is real and even.
\end{itemize}
Therefore, for Dirichlet boundary conditions ($f_o$ is odd), we use the \underline{Fourier sine transform}, and for Neumann boundary conditions ($f_e$ is even), we use the \underline{Fourier cosine transform}.

\begin{definition}[Fourier sine transform]\label{def.fourier_sine_transform}
The \underline{Fourier sine transform} and its inverse are defined by
\begin{equation}\label{eq.fourier_sine_transform}
  \hat{f}_s(p) = \int_0^{\infty}f(x)\sin(px)\,dx,
  \qquad
  f(x) = \frac{2}{\pi}\int_0^{\infty}\hat{f}_s(p)\sin(px)\,dp.
\end{equation}
\end{definition}

\begin{definition}[Fourier cosine transform]\label{def.fourier_cosine_transform}
The \underline{Fourier cosine transform} and its inverse are defined by
\begin{equation}\label{eq.fourier_cosine_transform}
  \hat{f}_c(p) = \int_0^{\infty}f(x)\cos(px)\,dx,
  \qquad
  f(x) = \frac{2}{\pi}\int_0^{\infty}\hat{f}_c(p)\cos(px)\,dp.
\end{equation}
\end{definition}

\subsubsection{Dirichlet case: solution via sine transform}

We use the odd extension $f_o$ to extend $f$ to $\mathbb{R}$, then apply the full-line solution.

Taking the Fourier sine transform of $u_t = k\,u_{xx}$ gives the ODE
\begin{equation}\label{eq.dirichlet_sine_ode}
  (\hat{u}_s)_t = -kp^2\,\hat{u}_s.
\end{equation}
With the initial condition $\hat{u}_s(p,0) = \hat{f}_s(p)$, the solution is
\begin{equation}\label{eq.dirichlet_sine_sol}
  \hat{u}_s(p,t) = \hat{f}_s(p)\,e^{-kp^2 t}.
\end{equation}
Applying the inverse sine transform,
\begin{equation}\label{eq.dirichlet_sine_invert}
  u(x,t) = \frac{2}{\pi}\int_0^{\infty}\hat{f}_s(p)\,\sin(px)\,e^{-kp^2 t}\,dp.
\end{equation}

Substituting $\hat{f}_s(p) = \int_0^{\infty}f(x')\sin(px')\,dx'$ and using the full-line result with the odd extension, we obtain
\begin{equation}\label{eq.dirichlet_half_line_sol}
  u(x,t) = \frac{1}{\sqrt{4\pi kt}}\int_0^{\infty}f(x')
    \left[e^{-(x-x')^2/(4kt)} - e^{-(x+x')^2/(4kt)}\right]\,dx'.
\end{equation}

We can write this as $u(x,t) = \int_0^{\infty} G^D(x,x';t)\,f(x')\,dx'$, where the \underline{Dirichlet Green's function} on the half-line is
\begin{equation}\label{eq.dirichlet_green}
  G^D(x,x';t) = \frac{1}{\sqrt{4\pi kt}}
  \left[e^{-(x-x')^2/(4kt)} - e^{-(x+x')^2/(4kt)}\right].
\end{equation}

\subsubsection{Neumann case: solution via cosine transform}

Using the even extension $f_e$ and the cosine transform, we get the same type of ODE:
\begin{equation}\label{eq.neumann_cosine_ode}
  (\hat{u}_c)_t = -kp^2\,\hat{u}_c,\qquad
  \hat{u}_c(p,0) = \hat{f}_c(p).
\end{equation}
The solution is $\hat{u}_c(p,t) = \hat{f}_c(p)\,e^{-kp^2 t}$. Applying the inverse cosine transform,
\begin{equation}\label{eq.neumann_cosine_invert}
  u(x,t) = \frac{2}{\pi}\int_0^{\infty}\hat{f}_c(p)\,\cos(px)\,e^{-kp^2 t}\,dp.
\end{equation}
Substituting $\hat{f}_c$ and computing as before, we obtain
\begin{equation}\label{eq.neumann_half_line_sol}
  u(x,t) = \frac{1}{\sqrt{4\pi kt}}\int_0^{\infty}f(x')
    \left[e^{-(x-x')^2/(4kt)} + e^{-(x+x')^2/(4kt)}\right]\,dx'.
\end{equation}

The \underline{Neumann Green's function} on the half-line is
\begin{equation}\label{eq.neumann_green}
  \boxed{G^N(x,x';t) = \frac{1}{\sqrt{4\pi kt}}
  \left[e^{-(x-x')^2/(4kt)} + e^{-(x+x')^2/(4kt)}\right].}
\end{equation}

Note that the Dirichlet Green's function uses a minus sign (odd extension), while the Neumann Green's function uses a plus sign (even extension).

\subsection{Wave equation on $\mathbb{R}$}

Consider the wave equation on the real line:
\begin{equation}\label{eq.wave_eq_R}
  u_{tt} = c^2\,u_{xx},\quad x\in\mathbb{R},\quad t>0,
\end{equation}
with initial conditions $u(x,0) = f(x)$ and $u_t(x,0) = g(x)$.

\subsubsection{Solution by Fourier transform}

Taking the Fourier transform in $x$ and using Corollary~\ref{cor.second_derivative_ft}, we obtain the ODE
\begin{equation}\label{eq.wave_eq_R_ft}
  \hat{u}_{tt} + c^2 p^2\,\hat{u} = 0.
\end{equation}
This is a constant-coefficient second-order ODE in $t$. The characteristic equation is $r^2 + c^2p^2 = 0$, giving $r = \pm icp$. By Theorem~\ref{th.2nd_ODE}, the general solution is
\begin{equation}\label{eq.wave_eq_R_ft_sol}
  \hat{u}(p,t) = A(p)\,e^{icpt} + B(p)\,e^{-icpt}.
\end{equation}

Applying the initial conditions $\hat{u}(p,0) = \hat{f}(p)$ and $\hat{u}_t(p,0) = \hat{g}(p)$, we get the system
\begin{equation}\label{eq.wave_eq_R_ic}
  A(p)+B(p) = \hat{f}(p),
  \qquad
  icp\bigl[A(p)-B(p)\bigr] = \hat{g}(p).
\end{equation}
Solving for $A(p)$ and $B(p)$,
\begin{equation}\label{eq.wave_eq_R_AB}
  A(p) = \frac{1}{2}\hat{f}(p) + \frac{\hat{g}(p)}{2icp},
  \qquad
  B(p) = \frac{1}{2}\hat{f}(p) - \frac{\hat{g}(p)}{2icp}.
\end{equation}

Substituting back into \eqref{eq.wave_eq_R_ft_sol} and using $\frac{e^{icpt}+e^{-icpt}}{2} = \cos(cpt)$ and $\frac{e^{icpt}-e^{-icpt}}{2} = i\sin(cpt)$, we get
\begin{equation}\label{eq.wave_eq_R_ft_combined}
  \hat{u}(p,t) = \hat{f}(p)\cos(cpt) + \frac{\hat{g}(p)}{icp}\sin(cpt).
\end{equation}

\subsubsection{D'Alembert's formula}

Applying the inverse Fourier transform to \eqref{eq.wave_eq_R_ft_combined}, the solution splits into two terms $u(x,t) = I_1 + I_2$.

For the first term, using $\cos(cpt) = \frac{1}{2}(e^{icpt} + e^{-icpt})$,
\begin{equation}\label{eq.dalembert_I1}
\begin{aligned}
  I_1 &= \frac{1}{2\pi}\int_{-\infty}^{\infty}\hat{f}(p)\,e^{ipx}\cos(cpt)\,dp \\
  &= \frac{1}{2}\!\left[
       \frac{1}{2\pi}\int_{-\infty}^{\infty}\hat{f}(p)\,e^{ip(x+ct)}\,dp
     + \frac{1}{2\pi}\int_{-\infty}^{\infty}\hat{f}(p)\,e^{ip(x-ct)}\,dp
     \right] \\
  &= \frac{1}{2}\bigl[f(x+ct)+f(x-ct)\bigr],
\end{aligned}
\end{equation}
where the last step follows from the inverse Fourier transform formula \eqref{eq.inverse_fourier_transform}.

For the second term,
\begin{equation}\label{eq.dalembert_I2_start}
  I_2 = \frac{1}{2\pi}\int_{-\infty}^{\infty}
    \frac{\hat{g}(p)}{icp}\,e^{ipx}\,\sin(cpt)\,dp.
\end{equation}
We use the identity $\frac{\sin(cpt)}{cp} = \int_0^t \cos(cps)\,ds$ to write
\begin{equation}\label{eq.dalembert_I2}
\begin{aligned}
  I_2 &= \frac{1}{2\pi}\int_0^t\int_{-\infty}^{\infty}
    \frac{\hat{g}(p)}{i}\,e^{ipx}\,\cos(cps)\,dp\,ds
  = \frac{1}{2c}\int_{x-ct}^{x+ct}g(s)\,ds.
\end{aligned}
\end{equation}

Combining $I_1$ and $I_2$, we obtain the following classical result.

\begin{theorem}[D'Alembert's formula]\label{th.dalembert}
The solution to the wave equation \eqref{eq.wave_eq_R} with initial conditions $u(x,0)=f(x)$ and $u_t(x,0)=g(x)$ is
\begin{equation}\label{eq.dalembert}
  \boxed{u(x,t) = \frac{1}{2}\bigl[f(x+ct)+f(x-ct)\bigr]
    + \frac{1}{2c}\int_{x-ct}^{x+ct}g(s)\,ds.}
\end{equation}
\end{theorem}

\begin{example}\label{ex.dalembert}
Consider $u_{tt} = c^2\,u_{xx}$ with $u(x,0) = e^{-x^2}$ and $u_t(x,0) = 0$. Since $g = 0$, D'Alembert's formula gives
\begin{equation}\label{eq.dalembert_example}
  u(x,t) = \frac{1}{2}\bigl[e^{-(x+ct)^2}+e^{-(x-ct)^2}\bigr].
\end{equation}
The initial profile $e^{-x^2}$ splits into two bumps traveling in opposite directions, each with speed $c$ and half the original amplitude.
\end{example}

\subsection{Nonhomogeneous heat equation}

Consider the nonhomogeneous heat equation
\begin{equation}\label{eq.nonhom_heat}
  u_t - k\,u_{xx} = h(x,t),\qquad u(x,0) = f(x).
\end{equation}

We decompose $u = v + w$, where $v$ solves the homogeneous problem with initial data, and $w$ solves the nonhomogeneous problem with zero initial data:
\begin{equation}\label{eq.nonhom_heat_decomp}
\begin{aligned}
  &v_t = k\,v_{xx},\qquad v(x,0) = f(x), \\
  &w_t - k\,w_{xx} = h(x,t),\qquad w(x,0) = 0.
\end{aligned}
\end{equation}

The solution for $v$ is already known from section~\ref{sec.heat_kernel}: $\hat{v}(p,t) = \hat{f}(p)\,e^{-kp^2 t}$.

For $w$, we take the Fourier transform in $x$ to obtain the first-order linear ODE
\begin{equation}\label{eq.nonhom_heat_w_ft}
  \hat{w}_t + kp^2\,\hat{w} = \hat{h}(p,t),
  \qquad \hat{w}(p,0) = 0.
\end{equation}
We solve this by the integrating factor method (Theorem~\ref{th.linear_ODE}). Multiplying by the integrating factor $e^{kp^2 t}$,
\begin{equation}\label{eq.nonhom_heat_w_integrating}
  \frac{d}{dt}\bigl[e^{kp^2 t}\,\hat{w}\bigr] = e^{kp^2 t}\,\hat{h}(p,t).
\end{equation}
Integrating from $0$ to $t$ and using $\hat{w}(p,0) = 0$,
\begin{equation}\label{eq.nonhom_heat_w_ft_sol}
  e^{kp^2 t}\,\hat{w}(p,t)
  = \int_0^t e^{kp^2 s}\,\hat{h}(p,s)\,ds
  \quad\Longrightarrow\quad
  \hat{w}(p,t) = \int_0^t \hat{h}(p,s)\,e^{-kp^2(t-s)}\,ds.
\end{equation}

Applying the inverse Fourier transform and substituting back $\hat{f}$ and $\hat{h}$, and using the Gaussian integral (Theorem~\ref{th.gaussian_integral}), we obtain the full solution:
\begin{equation}\label{eq.nonhom_heat_full}
\begin{aligned}
  u(x,t)
  &= \int_{-\infty}^{\infty}
     \frac{1}{\sqrt{4\pi kt}}\,e^{-(x-x')^2/(4kt)}\,f(x')\,dx' \\
  &\quad+ \int_0^t\int_{-\infty}^{\infty}
     \frac{1}{\sqrt{4\pi k(t-s)}}\,e^{-(x-x')^2/[4k(t-s)]}\,h(x',s)\,dx'\,ds.
\end{aligned}
\end{equation}
In terms of the Green's function \eqref{eq.heat_kernel},
\begin{equation}\label{eq.nonhom_heat_green}
  \boxed{u(x,t) = \int_{-\infty}^{\infty}G(x,x';t)\,f(x')\,dx'
  + \int_0^t\!\int_{-\infty}^{\infty}G(x,x';t-s)\,h(x',s)\,dx'\,ds.}
\end{equation}

\subsection{The Dirac delta function}

\begin{definition}[Dirac delta function]\label{def.dirac_delta}
The Dirac delta ``function'' $\delta(x)$ is defined by the two properties
\begin{equation}\label{eq.dirac_delta_def}
  \delta(x) =
  \begin{cases}
    \infty, & x=0, \\
    0,      & x\ne 0,
  \end{cases}
  \qquad\text{and}\qquad
  \int_{-\infty}^{\infty}\delta(x)\,dx = 1.
\end{equation}
It represents a ``point charge'' (or point source).
\end{definition}

For any domain $\Omega \subset \mathbb{R}$,
\begin{equation}\label{eq.dirac_delta_domain}
  \int_\Omega \delta(x)\,dx =
  \begin{cases}
    1, & 0\in\Omega, \\
    0, & 0\notin\Omega.
  \end{cases}
\end{equation}

Note that the heat kernel \eqref{eq.heat_kernel} approaches a delta function as $t \to 0^+$:
\begin{equation}\label{eq.heat_kernel_delta_limit}
  \frac{1}{\sqrt{4\pi kt}}\,e^{-(x-x')^2/(4kt)} \to \delta(x-x') \qquad \text{as } t \to 0^+.
\end{equation}

\subsubsection{Connection to Green's function}

The \underline{Green's function} $G$ for the heat equation satisfies
\begin{equation}\label{eq.green_heat_delta}
  \begin{cases}
    G_t = k\,G_{xx}, \\
    G\big|_{t=0} = \delta(x-x').
  \end{cases}
\end{equation}
That is, $G$ is the response to a point-source initial condition at $x=x'$.

\subsubsection{Properties of $\delta(x)$}

\begin{theorem}[Properties of $\delta(x)$]\label{th.delta_properties}
The Dirac delta function has the following properties.
\begin{enumerate}
  \item \textbf{Sifting property:}
    $\displaystyle\int_{-\infty}^{\infty}\delta(x-a)\,g(x)\,dx = g(a)$.
  \item $\delta(x)$ is even: $\delta(-x)=\delta(x)$.
  \item $\displaystyle\delta(cx) = \frac{1}{|c|}\,\delta(x)$, for $c\ne 0$.
  \item $x\,\delta(x) = 0$.
\end{enumerate}
\end{theorem}

\subsubsection{$\delta$ as a limit of sequences}

The delta function can be realized as a limit of ordinary functions:
\begin{equation}\label{eq.delta_limit}
  \delta(x) = \lim_{\varepsilon\to 0}\delta_\varepsilon(x),
\end{equation}
where $\delta_\varepsilon(x)$ is any family of functions satisfying $\int_{-\infty}^{\infty}\delta_\varepsilon(x)\,dx = 1$ and $\delta_\varepsilon(x)\to 0$ for $x\ne 0$ as $\varepsilon \to 0$. A standard example is the Gaussian approximation
\begin{equation}\label{eq.delta_gaussian}
  \delta_\varepsilon(x) = \frac{1}{\varepsilon\sqrt{\pi}}\,e^{-x^2/\varepsilon^2}.
\end{equation}

\subsubsection{Fourier transform of $\delta$}

By the sifting property with $g(x) = e^{-ipx}$ and $a = 0$,
\begin{equation}\label{eq.delta_ft}
  \hat{\delta}(p)
  = \int_{-\infty}^{\infty}\delta(x)\,e^{-ipx}\,dx = e^{0} = 1.
\end{equation}

Applying the inverse Fourier transform to $\hat{\delta}(p) = 1$, we obtain the \underline{Fourier representation of $\delta$}:
\begin{equation}\label{eq.delta_fourier_rep}
  \delta(x) = \frac{1}{2\pi}\int_{-\infty}^{\infty}e^{ipx}\,dp.
\end{equation}

\subsection{Green's function: general theory}

\subsubsection{Verification that $G$ is the Green's function}

We verify that the heat kernel $G(x,t) = \frac{1}{\sqrt{4\pi kt}}\,e^{-x^2/(4kt)}$ satisfies $G_t = k\,G_{xx}$ and $G\big|_{t\to 0^+} = \delta(x)$.

\begin{proof}
We work in Fourier space. The Fourier transform of $G$ is
\begin{equation}\label{eq.green_ft}
  \hat{G}(p,t) = \int_{-\infty}^{\infty}G(x,t)\,e^{-ipx}\,dx = e^{-kp^2 t}.
\end{equation}
We check both conditions:
\begin{enumerate}
  \item \textbf{PDE:} $\hat{G}_t = -kp^2\,e^{-kp^2 t} = -kp^2\,\hat{G}
    = k\,\widehat{G_{xx}}$, which is the Fourier transform of $G_t = k\,G_{xx}$.
  \item \textbf{Initial condition:} $\hat{G}(p,0) = 1 = \hat{\delta}(p)$,
    so $G\big|_{t=0} = \delta(x)$. \qedhere
\end{enumerate}
\end{proof}

\subsubsection{Convolution and the convolution theorem}

\begin{definition}[Convolution]\label{def.convolution}
The \underline{convolution} of two functions $f$ and $g$ is defined by
\begin{equation}\label{eq.convolution_def}
  (f\star g)(x) = \int_{-\infty}^{\infty}f(y)\,g(x-y)\,dy.
\end{equation}
\end{definition}

\begin{theorem}[Convolution theorem]\label{th.convolution}
The Fourier transform of a convolution is the product of the Fourier transforms:
\begin{equation}\label{eq.convolution_theorem}
  \widehat{f\star g}(p) = \hat{f}(p)\cdot\hat{g}(p).
\end{equation}
\end{theorem}

\begin{proof}
We compute directly from the definitions:
\begin{equation}\label{eq.proof_convolution}
\begin{aligned}
  \widehat{f\star g}(p)
  &= \int_{-\infty}^{\infty}
     \left(\int_{-\infty}^{\infty}f(y)\,g(x-y)\,dy\right)e^{-ipx}\,dx \\
  &= \int_{-\infty}^{\infty}f(y)\,e^{-ipy}
     \left(\int_{-\infty}^{\infty}g(x-y)\,e^{-ip(x-y)}\,dx\right)dy \\
  &= \hat{f}(p)\cdot\hat{g}(p),
\end{aligned}
\end{equation}
where in the second line we substituted $\mu = x - y$ in the inner integral.
\end{proof}

As a consistency check, the convolution theorem works for $\delta$: convolution with $\delta$ is the identity ($\delta\star g = g$), and in Fourier space $\hat{\delta}\cdot\hat{g} = 1\cdot\hat{g} = \hat{g}$.

\subsubsection{Superposition principle for Green's functions}

The sifting property of $\delta$ implies that any function $f(x)$ can be decomposed as
\begin{equation}\label{eq.f_delta_decomposition}
  f(x) = \int_{-\infty}^{\infty}f(x')\,\delta(x-x')\,dx'.
\end{equation}
This means that $f$ is a ``continuous linear combination'' of delta functions $\delta(x - x')$, with coefficients $f(x')$.

Since $G(x,x';t)$ solves the heat equation with initial data $\delta(x-x')$, by linearity (superposition), the solution for general initial data $f$ is
\begin{equation}\label{eq.green_superposition}
  u(x,t) = \int_{-\infty}^{\infty}G(x,x';t)\,f(x')\,dx'.
\end{equation}

\begin{lemma}\label{lem.green_solution}
If $G(x,x';t)$ satisfies \eqref{eq.green_heat_delta}, then $u(x,t) = \int_{-\infty}^{\infty}G(x,x';t)\,f(x')\,dx'$ is a solution to $u_t = k\,u_{xx}$, $u|_{t=0} = f(x)$.
\end{lemma}

\begin{proof}
We verify both conditions.

\textit{PDE:} Differentiating under the integral sign,
\begin{equation}\label{eq.green_pde_check}
  u_t = \int_{-\infty}^{\infty}G_t(x,x';t)\,f(x')\,dx'
  = \int_{-\infty}^{\infty}k\,G_{xx}(x,x';t)\,f(x')\,dx'
  = k\,u_{xx}.
\end{equation}

\textit{Initial condition:}
\begin{equation}\label{eq.green_ic_check}
  u\big|_{t=0}
  = \int_{-\infty}^{\infty}G(x,x';0)\,f(x')\,dx'
  = \int_{-\infty}^{\infty}\delta(x-x')\,f(x')\,dx'
  = f(x).
\end{equation}
\end{proof}

\subsubsection{Computing $G$ by Fourier transform}

We can also derive the heat kernel by directly solving the Green's function problem \eqref{eq.green_heat_delta}. Taking the Fourier transform in $x$,
\begin{equation}\label{eq.green_ft_ode}
  \hat{G}_t = -kp^2\,\hat{G}.
\end{equation}
The initial condition $G|_{t=0} = \delta(x-x')$ gives
\begin{equation}\label{eq.green_ft_ic}
  \hat{G}(p,0)
  = \int_{-\infty}^{\infty}\delta(x-x')\,e^{-ipx}\,dx = e^{-ipx'}.
\end{equation}
Solving the ODE gives $\hat{G}(p,t) = e^{-ipx'}\,e^{-kp^2 t}$. Applying the inverse Fourier transform and using the Gaussian integral (Theorem~\ref{th.gaussian_integral}),
\begin{equation}\label{eq.green_ft_result}
\begin{aligned}
  G(x,x';t) &= \frac{1}{2\pi}\int_{-\infty}^{\infty}e^{ipx}\,e^{-ipx'}\,e^{-kp^2 t}\,dp
  = \frac{1}{2\pi}\int_{-\infty}^{\infty}e^{-kp^2 t + ip(x-x')}\,dp \\
  &= \frac{1}{\sqrt{4\pi kt}}\,e^{-(x-x')^2/(4kt)}.
\end{aligned}
\end{equation}
This confirms the heat kernel \eqref{eq.heat_kernel}.

\subsection{Green's function for the Laplace equation}

\subsubsection{Poisson's equation in 2D}

Consider Poisson's equation in 2D:
\begin{equation}\label{eq.poisson_2d}
  u_{xx} + u_{yy} = f(x,y).
\end{equation}
The Green's function satisfies
\begin{equation}\label{eq.green_laplace_2d}
  G_{xx} + G_{yy} = \delta(x-x')\,\delta(y-y').
\end{equation}
For $(x,y)\ne(x',y')$, the right-hand side vanishes, so $G_{xx}+G_{yy}=0$.

\subsubsection{Free-space Green's function}

By symmetry, $G$ depends only on the distance $r = \sqrt{(x-x')^2+(y-y')^2}$ from the source point $(x',y')$. In polar coordinates centered at $(x',y')$, the Laplacian of a radially symmetric function gives
\begin{equation}\label{eq.laplacian_polar}
  0 = \frac{1}{r}\bigl(r\,G_r\bigr)_r \qquad\text{for } r>0.
\end{equation}
Integrating once gives $r\,G_r = B$, and integrating again gives
\begin{equation}\label{eq.green_laplace_radial}
  G(r) = A + B\ln r.
\end{equation}

To determine $B$, we integrate the equation $G_{xx}+G_{yy}=\delta(x-x')\delta(y-y')$ over a disk $D_\varepsilon$ of radius $\varepsilon$ centered at $(x',y')$ and apply the divergence theorem:
\begin{equation}\label{eq.green_laplace_normalization}
  1 = \iint_{D_\varepsilon} (G_{xx}+G_{yy})\,dA = \oint_{\partial D_\varepsilon} \frac{\partial G}{\partial r}\,ds
  = \frac{B}{r}\cdot 2\pi r\bigg|_{r=\varepsilon} = 2\pi B.
\end{equation}
Therefore $B = \frac{1}{2\pi}$, and we may take $A = 0$. The \underline{free-space Green's function} for Poisson's equation in 2D is
\begin{equation}\label{eq.green_laplace_2d_result}
  \boxed{G(x,y;x',y') = \frac{1}{2\pi}\ln r = \frac{1}{2\pi}\ln\sqrt{(x-x')^2+(y-y')^2}.}
\end{equation}

The solution to Poisson's equation is then
\begin{equation}\label{eq.poisson_sol}
  u(x,y) = \iint G(x,y;x',y')\,f(x',y')\,dx'\,dy'.
\end{equation}

\subsubsection{Method of mirror images}

The \underline{method of mirror images} allows us to construct Green's functions on half-spaces by placing ``mirror charges'' to enforce boundary conditions.

\begin{example}[Dirichlet on the half-line]\label{ex.mirror_dirichlet}
Consider the heat equation on $x > 0$ with Dirichlet boundary condition:
\begin{equation}\label{eq.mirror_dirichlet_problem}
  u_t = k\,u_{xx},\quad u\big|_{x=0}=0,\quad u\big|_{t=0}=f(x).
\end{equation}
The Green's function must satisfy
\begin{equation}\label{eq.mirror_dirichlet}
  \begin{cases}
    G_t = k\,G_{xx}, \\
    G\big|_{t=0} = \delta(x-x'), \\
    G\big|_{x=0} = 0.
  \end{cases}
\end{equation}

We place a mirror charge of opposite sign $-\delta(x+x')$ at $-x'$. The free-space Green's function is $G^{\text{free}} = \frac{1}{\sqrt{4\pi kt}}\,e^{-(x-x')^2/(4kt)}$ and the image contribution is $G^{\text{image}} = -\frac{1}{\sqrt{4\pi kt}}\,e^{-(x+x')^2/(4kt)}$. Therefore
\begin{equation}\label{eq.mirror_dirichlet_sol}
  G^D(x,x';t) = \frac{1}{\sqrt{4\pi kt}}
  \left[e^{-(x-x')^2/(4kt)} - e^{-(x+x')^2/(4kt)}\right].
\end{equation}

One can verify that the boundary condition is satisfied: $G^D(0,x';t) = \frac{1}{\sqrt{4\pi kt}}
\bigl[e^{-x'^2/(4kt)} - e^{-x'^2/(4kt)}\bigr] = 0$.
\end{example}

\begin{example}[Neumann on the half-line]\label{ex.mirror_neumann}
For the Neumann boundary condition $u_x(0,t) = 0$, the image has the \emph{same} sign (even reflection):
\begin{equation}\label{eq.mirror_neumann_sol}
  G^N(x,x';t) = \frac{1}{\sqrt{4\pi kt}}
  \left[e^{-(x-x')^2/(4kt)} + e^{-(x+x')^2/(4kt)}\right].
\end{equation}
The solution is then $u(x,t) = \int_0^{\infty}G^N(x,x';t)\,f(x')\,dx'$.
\end{example}

\begin{example}[2D Laplace equation on the half-plane]\label{ex.mirror_2d_laplace}
For the 2D Laplace equation on the half-plane $x>0$ with Dirichlet boundary condition $G|_{x=0}=0$, we place a mirror charge at $(-x', y')$:
\begin{equation}\label{eq.mirror_2d_laplace}
  G^D(x,y;x',y') = \frac{1}{4\pi}
  \ln\!\left[\frac{(x-x')^2+(y-y')^2}{(x+x')^2+(y-y')^2}\right].
\end{equation}
\end{example}
