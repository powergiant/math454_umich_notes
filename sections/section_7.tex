\section{Fourier transform and PDEs on unbounded domains}

\subsection{The Fourier transform}

\subsubsection{Motivation: from Fourier series to Fourier transform}\label{sec.motivation_fourier}

Recall the complex form of the Fourier series on $[-L,L]$,
\[
  f(x) = \sum_{n=-\infty}^{\infty} \alpha_n\,e^{in\pi x/L},
  \qquad
  \alpha_n = \frac{1}{2L}\int_{-L}^{L} f(x')\,e^{-in\pi x'/L}\,dx'.
\]
Substituting $\alpha_n$ back into the expansion gives
\[
  f(x)
  = \sum_{n=-\infty}^{\infty}
    \frac{1}{2L}\int_{-L}^{L} f(x')\,e^{-in\pi x'/L}\,dx'
    \;\cdot\; e^{in\pi x/L}.
\]
Define the discrete frequencies
\[
  \mu_n = \frac{n\pi}{L}, \qquad \Delta \mu = \mu_{n+1}-\mu_n = \frac{\pi}{L}.
\]
Then we can rewrite the Fourier series as a Riemann sum:
\[
  f(x)
  = \sum_{n=-\infty}^{\infty}
    \frac{1}{2\pi}
    \underbrace{\int_{-L}^{L} f(x')\,e^{-i\mu_n x'}\,dx'}_{\displaystyle\to\;\hat{f}(\mu_n)}
    \;\cdot\; e^{i\mu_n x}\;\Delta \mu.
\]
As $L\to\infty$, $\Delta \mu\to 0$ and the discrete variable $\mu_n$ becomes a continuous variable $\mu$. The Riemann sum converges to an integral:
\begin{equation}\label{eq.ft_limit}
  f(x)
  = \frac{1}{2\pi}\int_{-\infty}^{\infty}
    \!\left(\int_{-\infty}^{\infty} f(x')\,e^{-i\mu x'}\,dx'\right)
    e^{i\mu x}\,d\mu.
\end{equation}
This motivates the following definition.

\subsubsection{Definition and properties}

\begin{definition}[Fourier transform]\label{def.fourier_transform}
Let $f(x)$ be a sufficiently fast-decaying function defined on $\mathbb{R}$. The \underline{Fourier transform} of $f$ is defined by
\begin{equation}\label{eq.fourier_transform}
  \hat{f}(\mu) = \int_{-\infty}^{\infty} f(x)\,e^{-i\mu x}\,dx.
\end{equation}
% The \underline{inverse Fourier transform} is given by
% \begin{equation}\label{eq.inverse_fourier_transform}
%   f(x) = \frac{1}{2\pi}\int_{-\infty}^{\infty} \hat{f}(\mu)\,e^{i\mu x}\,d\mu.
% \end{equation}
\end{definition}

\begin{theorem}[Properties of the Fourier transform]\label{th.fourier_transform_properties}
Let $f(x)$ be a fast-decaying function defined on $\mathbb{R}$. Then the following properties hold.
\begin{enumerate}
  \item Inversion formula. $f(x) = \frac{1}{2\pi}\int_{-\infty}^{\infty} \hat{f}(\mu)\,e^{i\mu x}\,d\mu$.
  \item Derivative property. $\widehat{f'}(\mu) = i\mu\,\hat{f}(\mu)$.
  \item Parseval's identity.
    $\displaystyle\int_{-\infty}^{\infty} |f(x)|^2\,dx
      = \frac{1}{2\pi}\int_{-\infty}^{\infty} |\hat{f}(\mu)|^2\,d\mu.$
\end{enumerate}
\end{theorem}
\begin{proof} The first conclusion follows directly from the derivation in section \ref{sec.motivation_fourier}

The second conclusion can be proved as follows. We compute the Fourier transform of $f'$ by integration by parts:
\begin{equation}\label{eq.proof_ft_derivative}
  \widehat{f'}(\mu)
  = \int_{-\infty}^{\infty} f'(x)\,e^{-i\mu x}\,dx
  = \bigl[f(x)\,e^{-i\mu x}\bigr]_{-\infty}^{\infty}
    + i\mu\int_{-\infty}^{\infty} f(x)\,e^{-i\mu x}\,dx
  = i\mu\,\hat{f}(\mu),
\end{equation}
where the boundary term vanishes since $f(x)\to 0$ as $|x|\to\infty$.

The third conclusion can be proved as follows.
For each $L>0$, let $f_L$ be the $2L$-periodic extension of $f|_{[-L,L]}$. Its complex Fourier series is
\[
  f_L(x) = \sum_{n=-\infty}^{\infty} \alpha_n\,e^{in\pi x/L},
  \qquad
  \alpha_n = \frac{1}{2L}\int_{-L}^{L} f_L(x)\,e^{-in\pi x/L}\,dx.
\]
Introduce the truncated Fourier transform
\[
  \hat{f}_L(\mu) = \int_{-L}^{L} f_L(x)\,e^{-i\mu x}\,dx,
  \qquad \mu_n = \frac{n\pi}{L}.
\]
Then $\alpha_n = \frac{1}{2L}\hat{f}_L(\mu_n)$, so Parseval for Fourier series gives
\begin{equation}\label{eq.parseval_series_to_transform}
  \int_{-L}^{L} |f_L(x)|^2\,dx
  = 2L\sum_{n=-\infty}^{\infty} |\alpha_n|^2
  = \frac{1}{2L}\sum_{n=-\infty}^{\infty} |\hat{f}_L(\mu_n)|^2.
\end{equation}
Since $\Delta\mu = \mu_{n+1}-\mu_n = \pi/L$, we can rewrite
\[
  \frac{1}{2L}\sum_{n=-\infty}^{\infty} |\hat{f}_L(\mu_n)|^2
  = \frac{\Delta\mu}{2\pi}\sum_{n=-\infty}^{\infty} |\hat{f}_L(\mu_n)|^2.
\]
As $L\to\infty$, we have $f_L\to f$ in $L^2(\mathbb{R})$ and $\hat{f}_L(\mu)\to\hat{f}(\mu)$,
so the sum becomes a Riemann sum. Therefore
\[
  \int_{-\infty}^{\infty} |f(x)|^2\,dx
  = \frac{1}{2\pi}\int_{-\infty}^{\infty} |\hat{f}(\mu)|^2\,d\mu,
\]
which is the desired identity.
\end{proof}


\subsection{Solving heat equation by Fourier transform}\label{sec.heat_equation_fourier}

In this section, we solve the following heat equation using Fourier transform

\begin{equation}\label{eq.heat_eq_R}
\left\{\begin{aligned} 
    &u_t=K u_{xx}, \qquad x \in (-\infty, \infty), 
    \\
    &u(x, 0)=f(x).
\end{aligned}\right.
\end{equation}

We apply the Fourier transform in $x$ to both sides of \eqref{eq.heat_eq_R}. By Theorem \ref{th.fourier_transform_properties} 2, the right-hand side transforms as $K\,\widehat{u_{xx}} = i \mu K\,\widehat{u_x}  = -K\mu^2\,\hat{u}$. Since the Fourier transform does not act on $t$, we obtain the ODE
\begin{equation}\label{eq.heat_eq_R_ft}
  \hat{u}_t = -K\mu^2\,\hat{u}.
\end{equation}

\subsubsection{Applying the Fourier transform}\label{sec.apply_fourier}

This is a separable first-order ODE in $t$ (with $\mu$ as a parameter). Solving by separation of variables,
\begin{equation}\label{eq.heat_eq_R_ft_solve}
  \frac{d\hat{u}}{dt} = -K\mu^2\,\hat{u}
  \quad\Longrightarrow\quad
  \ln\hat{u} = -K\mu^2\,t + C
  \quad\Longrightarrow\quad
  \hat{u}(\mu,t) = \hat{u}(\mu,0)\,e^{-K\mu^2 t}.
\end{equation}
The initial condition gives $\hat{u}(\mu,0) = \hat{f}(\mu)$, so
\begin{equation}\label{eq.heat_eq_R_ft_sol}
  \hat{u}(\mu,t) = \hat{f}(\mu)\,e^{-K\mu^2 t}.
\end{equation}

Applying the inverse Fourier transform Theorem \ref{th.fourier_transform_properties} 1,
\begin{equation}\label{eq.heat_eq_R_sol_ft}
  u(x,t) = \frac{1}{2\pi}\int_{-\infty}^{\infty}
    \hat{f}(\mu)\,e^{-K\mu^2 t}\,e^{i\mu x}\,d\mu.
\end{equation}

We can simplify \eqref{eq.heat_eq_R_sol_ft} into a convolution form. Substituting $\hat{f}(\mu) = \int_{-\infty}^{\infty}f(x')\,e^{-i\mu x'}\,dx'$ and interchanging the order of integration,
\begin{equation}\label{eq.heat_kernel_derivation}
  u(x,t) = \frac{1}{2\pi}\int_{-\infty}^{\infty}f(x')
    \int_{-\infty}^{\infty}e^{-K\mu^2 t + i\mu(x-x')}\,d\mu\;dx'.
\end{equation}

\subsubsection{The Gaussian integral formula}

To evaluate the inner integral, we use the following Gaussian integral formula.

\begin{theorem}[Gaussian integral]\label{th.gaussian_integral_simple}
\begin{equation}\label{eq.gaussian_integral_simple}
  \int_{-\infty}^{\infty} e^{-\mu^2}\,d\mu
  = \sqrt{\pi}.
\end{equation}
\end{theorem}
\begin{proof}
    This is established in the calculus course.
\end{proof}

\begin{theorem}[Gaussian integral]\label{th.gaussian_integral}
\begin{equation}\label{eq.gaussian_integral}
  \int_{-\infty}^{\infty} e^{-\alpha \mu^2+\beta \mu}\,d\mu
  = \sqrt{\frac{\pi}{\alpha}}\;e^{\beta^2/(4\alpha)},
  \qquad \operatorname{Re}(\alpha)>0.
\end{equation}
\end{theorem}
\begin{proof}
Complete the square:
\[
  -\alpha\mu^2+\beta\mu
  = -\alpha\left(\mu-\frac{\beta}{2\alpha}\right)^2 + \frac{\beta^2}{4\alpha}.
\]
Hence
\[
  \int_{-\infty}^{\infty} e^{-\alpha \mu^2+\beta \mu}\,d\mu
  = e^{\beta^2/(4\alpha)}\int_{-\infty}^{\infty}
    \exp\!\left(-\alpha\left(\mu-\frac{\beta}{2\alpha}\right)^2\right)\,d\mu.
\]
By the change of variables $y=\sqrt{\alpha}\left(\mu-\beta/(2\alpha)\right)$, we obtain
\[
  \int_{-\infty}^{\infty}
    \exp\!\left(-\alpha\left(\mu-\frac{\beta}{2\alpha}\right)^2\right)\,d\mu
  = \frac{1}{\sqrt{\alpha}}\int_{-\infty}^{\infty} e^{-y^2}\,dy,
\]
The the conclusion follows from applying Theorem~\ref{th.gaussian_integral_simple}.
\end{proof}

We apply Theorem~\ref{th.gaussian_integral} with $\alpha = Kt$ and $\beta = i(x-x')$, so that $\beta^2 = -(x-x')^2$:
\begin{equation}\label{eq.heat_kernel_gaussian}
  \int_{-\infty}^{\infty}e^{-Kt\mu^2+i(x-x')\mu}\,d\mu
  = \sqrt{\frac{\pi}{Kt}}\;e^{-(x-x')^2/(4Kt)}.
\end{equation}

Substituting back into \eqref{eq.heat_kernel_derivation},
\begin{align}
  u(x,t)
  &= \frac{1}{2\pi}\int_{-\infty}^{\infty}f(x')\;
     \sqrt{\frac{\pi}{Kt}}\;e^{-(x-x')^2/(4Kt)}\,dx' \nonumber\\
  &= \frac{1}{\sqrt{4\pi Kt}}\int_{-\infty}^{\infty}
     f(x')\,e^{-(x-x')^2/(4Kt)}\,dx'. \label{eq.heat_kernel_sol}
\end{align}

We define the \underline{heat kernel} (also called the \underline{Green's function} for the heat equation on $\mathbb{R}$):
\begin{equation}\label{eq.heat_kernel}
  G(x,x';t) = \frac{1}{\sqrt{4\pi Kt}}\,e^{-(x-x')^2/(4Kt)}.
\end{equation}
The solution to the heat equation \eqref{eq.heat_eq_R} is then
\begin{equation}\label{eq.heat_sol_green}
  u(x,t) = \int_{-\infty}^{\infty}G(x,x';t)\,f(x')\,dx'.
\end{equation}

\subsubsection{Connection to eigenfunction expansion}

Consider the heat equation $u_t = K\,u_{xx}$ on $x\in\mathbb{R}$ and set
$A=-\partial_{xx}$ with domain $\textrm{Dom}(A)=\{\phi:\phi \text{ is bounded on }\mathbb{R}\}$.
The eigenvalue problem $A\phi=\lambda\phi$ is
\[
  -\phi''=\lambda\phi.
\]
Writing $\lambda=\mu^2$ (case A) gives $\phi''+\mu^2\phi=0$, hence
\[
  \phi(x)=e^{i\mu x},\qquad \mu\in\mathbb{R}.
\]
Writing $\lambda=-\mu^2$ (case B) gives $\phi''-\mu^2\phi=0$, so
$\phi(x)=Ae^{\mu x}+Be^{-\mu x}$, which is unbounded unless $A=B=0$.

Thus only the (case A) $\lambda=\mu^2\ge 0$ is acceptable, and the eigenfunction expansion is
\begin{equation}\label{eq.eigenfunction_expansion_ft}
  u(x,t) = \int_{-\infty}^{\infty}\hat{u}(\mu,t)\,e^{i\mu x}\,\frac{d\mu}{2\pi},
\end{equation}
which is exactly the inverse Fourier transform.

Since $u_t=K\,u_{xx}=-K A u$, differentiating under the integral gives
\[
  u_t(x,t) = \int_{-\infty}^{\infty}\frac{d\hat{u}}{dt}(\mu,t)\,e^{i\mu x}\,\frac{d\mu}{2\pi},
  \qquad
  -K A u = -K\int_{-\infty}^{\infty}\mu^2\hat{u}(\mu,t)\,e^{i\mu x}\,\frac{d\mu}{2\pi}.
\]
Equating coefficients yields the ODE
\begin{equation}\label{eq.eigenfunction_ode}
  \frac{d\hat{u}}{dt}(\mu,t) = -K\mu^2\,\hat{u}(\mu,t).
\end{equation}



\subsection{Boundary value problem on the half line}

The Fourier transform requires a solution on the whole line. For problems on $(0,\infty)$ we extend the initial data to $\mathbb{R}$ by odd/even reflection, which leads naturally to the sine/cosine transforms.

\subsubsection{Heat equation on the half-line}

Consider the heat equation
\begin{equation}\label{eq.heat_eq_half_line}
  u_t = K\,u_{xx},\quad x>0,\quad t>0,\qquad u(x,0)=f(x),
\end{equation}
with one of the boundary conditions
\begin{itemize}
  \item Dirichlet. $u(0,t)=0$,
  \item Neumann. $u_x(0,t)=0$.
\end{itemize}

\subsubsection{Odd and even extensions}

Define the odd and even extensions of $f$ to $\mathbb{R}$ by
\begin{equation}\label{eq.odd_extension_ft}
  f_O(x)=
  \begin{cases}
    f(x), & x>0,\\
    0, & x=0,\\
    -f(-x), & x<0,
  \end{cases}
\end{equation}
\begin{equation}\label{eq.even_extension_ft}
  f_E(x)=
  \begin{cases}
    f(x), & x>0,\\
    0, & x=0,\\
    f(-x), & x<0.
  \end{cases}
\end{equation}
The odd extension leads to the sine transform (Dirichlet), and the even extension leads to the cosine transform (Neumann).

\begin{lemma}\label{lem.odd_even_extension_bc}
If $f$ is odd, then $f(0)=0$. If $f$ is even and differentiable, then $f'(0)=0$.
\end{lemma}
\begin{proof}
If $f$ is odd, $f(0)=-f(0)$ so $f(0)=0$. If $f$ is even, then $f'(x)=-f'(-x)$, so $f'(0)=-f'(0)$ and hence $f'(0)=0$.
\end{proof}

\begin{theorem}\label{thm.half_line_reflection}
Let $u_O$ solve the full-line heat equation with initial data $f_O$, and let $u_E$ solve the full-line heat equation with initial data $f_E$. Then the restrictions $u_O|_{x>0}$ and $u_E|_{x>0}$ solve the half-line heat equation with Dirichlet and Neumann boundary conditions, respectively.
\end{theorem}

\subsubsection{Dirichlet case: odd extension}

The full-line solution with initial data $f_O$ is
\begin{equation}\label{eq.dirichlet_full_line_sol}
  u_O(x,t)=\frac{1}{\sqrt{4\pi Kt}}\int_{-\infty}^{\infty} e^{-(x-x')^2/(4Kt)}\,f_O(x')\,dx'.
\end{equation}
For $x>0$, split the integral and substitute the odd extension:
\begin{align}
  u(x,t) &= \frac{1}{\sqrt{4\pi Kt}}
    \left(\int_{0}^{\infty} e^{-(x-x')^2/(4Kt)}\,f(x')\,dx'
    + \int_{-\infty}^{0} e^{-(x-x')^2/(4Kt)}\,(-f(-x'))\,dx'\right)\nonumber\\
  &= \frac{1}{\sqrt{4\pi Kt}}
    \left(\int_{0}^{\infty} e^{-(x-x')^2/(4Kt)}\,f(x')\,dx'
    - \int_{0}^{\infty} e^{-(x+x')^2/(4Kt)}\,f(x')\,dx'\right).\label{eq.dirichlet_split_integral}
\end{align}
Therefore
\begin{equation}\label{eq.dirichlet_half_line_sol}
  u(x,t)=\frac{1}{\sqrt{4\pi Kt}}\int_{0}^{\infty} f(x')
  \left[e^{-(x-x')^2/(4Kt)}-e^{-(x+x')^2/(4Kt)}\right]\,dx'.
\end{equation}
Equivalently, $u(x,t)=\int_{0}^{\infty} G^D(x,x';t)\,f(x')\,dx'$, where the Dirichlet Green's function is
\begin{equation}\label{eq.dirichlet_green}
  G^D(x,x';t)=\frac{1}{\sqrt{4\pi Kt}}
  \left[e^{-(x-x')^2/(4Kt)}-e^{-(x+x')^2/(4Kt)}\right].
\end{equation}

\subsubsection{Neumann case: even extension}

The full-line solution with initial data $f_E$ is
\begin{equation}\label{eq.neumann_full_line_sol}
  u_E(x,t)=\frac{1}{\sqrt{4\pi Kt}}\int_{-\infty}^{\infty} e^{-(x-x')^2/(4Kt)}\,f_E(x')\,dx'.
\end{equation}
For $x>0$ we use the even extension and the same splitting:
\begin{align}
  u(x,t) &= \frac{1}{\sqrt{4\pi Kt}}
    \left(\int_{0}^{\infty} e^{-(x-x')^2/(4Kt)}\,f(x')\,dx'
    + \int_{-\infty}^{0} e^{-(x-x')^2/(4Kt)}\,f(-x')\,dx'\right)\nonumber\\
  &= \frac{1}{\sqrt{4\pi Kt}}
    \left(\int_{0}^{\infty} e^{-(x-x')^2/(4Kt)}\,f(x')\,dx'
    + \int_{0}^{\infty} e^{-(x+x')^2/(4Kt)}\,f(x')\,dx'\right).\label{eq.neumann_split_integral}
\end{align}
Hence
\begin{equation}\label{eq.neumann_half_line_sol}
  u(x,t)=\frac{1}{\sqrt{4\pi Kt}}\int_{0}^{\infty} f(x')
  \left[e^{-(x-x')^2/(4Kt)}+e^{-(x+x')^2/(4Kt)}\right]\,dx'.
\end{equation}
The Neumann Green's function on the half-line is
\begin{equation}\label{eq.neumann_green}
  G^N(x,x';t)=\frac{1}{\sqrt{4\pi Kt}}
  \left[e^{-(x-x')^2/(4Kt)}+e^{-(x+x')^2/(4Kt)}\right].
\end{equation}
The Dirichlet kernel uses a minus sign (odd extension) and the Neumann kernel uses a plus sign (even extension).

\subsection{Wave equation on $\mathbb{R}$}

Consider the wave equation on the real line:
\begin{equation}\label{eq.wave_eq_R}
  u_{tt} = c^2\,u_{xx},\quad x\in\mathbb{R},\quad t>0,
\end{equation}
with initial conditions $u(x,0) = f(x)$ and $u_t(x,0) = g(x)$.

\subsubsection{Solution by Fourier transform}

Taking the Fourier transform in $x$ and using Theorem \ref{th.fourier_transform_properties} 2, we obtain the ODE
\begin{equation}\label{eq.wave_eq_R_ft}
  \hat{u}_{tt} + c^2 \mu^2\,\hat{u} = 0.
\end{equation}
This is a constant-coefficient second-order ODE in $t$. The characteristic equation is $r^2 + c^2\mu^2 = 0$, giving $r = \pm ic\mu$. By Theorem~\ref{th.2nd_ODE}, the general solution is
\begin{equation}\label{eq.wave_eq_R_ft_sol}
  \hat{u}(\mu,t) = A(\mu)\,\cos(c\mu t) + B(\mu)\,\sin(c\mu t)
\end{equation}

Applying the initial conditions $\hat{u}(\mu,0) = \hat{f}(\mu)$ and $\hat{u}_t(\mu,0) = \hat{g}(\mu)$, we get
\begin{equation}\label{eq.wave_eq_R_ic}
  A(\mu) = \hat{f}(\mu),
  \qquad
  ic\mu B(\mu) = \hat{g}(\mu).
\end{equation}
Solving for $A(\mu)$ and $B(\mu)$,
\begin{equation}\label{eq.wave_eq_R_AB}
  A(\mu) = \hat{f}(\mu),
  \qquad
  B(\mu) = \frac{\hat{g}(\mu)}{ic\mu}.
\end{equation}

Substituting back into \eqref{eq.wave_eq_R_ft_sol}, we get
\begin{equation}\label{eq.wave_eq_R_ft_combined}
  \hat{u}(\mu,t) = \hat{f}(\mu)\cos(c\mu t) + \frac{\hat{g}(\mu)}{ic\mu}\sin(c\mu t).
\end{equation}

\subsubsection{D'Alembert's formula}

Applying the inverse Fourier transform to \eqref{eq.wave_eq_R_ft_combined}, the solution splits into two terms $u(x,t) = I_1 + I_2$.

For the first term, using $\cos(c\mu t) = \frac{1}{2}(e^{ic\mu t} + e^{-ic\mu t})$,
\begin{equation}\label{eq.dalembert_I1}
\begin{aligned}
  I_1 &= \frac{1}{2\pi}\int_{-\infty}^{\infty}\hat{f}(\mu)\,e^{i\mu x}\cos(c\mu t)\,d\mu \\
  &= \frac{1}{2}\!\left[
       \frac{1}{2\pi}\int_{-\infty}^{\infty}\hat{f}(\mu)\,e^{i\mu(x+ct)}\,d\mu
     + \frac{1}{2\pi}\int_{-\infty}^{\infty}\hat{f}(\mu)\,e^{i\mu(x-ct)}\,d\mu
     \right] \\
  &= \frac{1}{2}\bigl[f(x+ct)+f(x-ct)\bigr],
\end{aligned}
\end{equation}
where the last step follows from the inverse Fourier transform formula \eqref{eq.inverse_fourier_transform}.

For the second term,
\begin{equation}\label{eq.dalembert_I2_start}
  I_2 = \frac{1}{2\pi}\int_{-\infty}^{\infty}
    \frac{\hat{g}(\mu)}{ic\mu}\,e^{i\mu x}\,\sin(c\mu t)\,d\mu.
\end{equation}
We use the identity $\frac{e^{i\mu x}\,\sin(c\mu t)}{i c\mu} = \frac{e^{i\mu (x - ct)} - e^{i\mu (x + ct)}}{2c\mu} = \frac{1}{2c}\int_{x-ct}^{x+ct} e^{i\mu x'} \mathrm{d} x'$ to write
\begin{equation}\label{eq.dalembert_I2}
\begin{aligned}
  I_2 &= \frac{1}{2c}\int_{x-ct}^{x+ct} \left(\frac{1}{2\pi} \int_{-\infty}^{\infty} \hat{g}(\mu) e^{i\mu x'} d\mu\right) \, \mathrm{d} x' = \frac{1}{2c}\int_{x-ct}^{x+ct} g(x') \, \mathrm{d} x'.
\end{aligned}
\end{equation}

Combining $I_1$ and $I_2$, we obtain the following classical result.

\begin{theorem}[D'Alembert's formula]\label{th.dalembert}
The solution to the wave equation \eqref{eq.wave_eq_R} with initial conditions $u(x,0)=f(x)$ and $u_t(x,0)=g(x)$ is
\begin{equation}\label{eq.dalembert}
  u(x,t) = \frac{1}{2}\bigl[f(x+ct)+f(x-ct)\bigr]
    + \frac{1}{2c}\int_{x-ct}^{x+ct}g(x')\,\mathrm{d}x'.
\end{equation}
\end{theorem}
\begin{proof}
    The proof follows from the explanation above.
\end{proof}

\begin{example}\label{ex.dalembert}
Consider $u_{tt} = c^2\,u_{xx}$ with $u(x,0) = e^{-x^2}$ and $u_t(x,0) = 0$. Since $g = 0$, D'Alembert's formula gives
\begin{equation}\label{eq.dalembert_example}
  u(x,t) = \frac{1}{2}\bigl[e^{-(x+ct)^2}+e^{-(x-ct)^2}\bigr].
\end{equation}
The initial profile $e^{-x^2}$ splits into two bumps traveling in opposite directions, each with speed $c$ and half the original amplitude.
\end{example}

\subsection{Nonhomogeneous heat equation}

Consider the nonhomogeneous heat equation
\begin{equation}\label{eq.nonhom_heat}
  u_t - K\,u_{xx} = h(x,t),\qquad u(x,0) = f(x).
\end{equation}

We decompose $u = v + w$, where $v$ solves the homogeneous problem with initial data, and $w$ solves the nonhomogeneous problem with zero initial data:
\begin{equation}\label{eq.nonhom_heat_decomp}
\begin{aligned}
  &v_t = K\,v_{xx},\qquad v(x,0) = f(x), \\
  &w_t - K\,w_{xx} = h(x,t),\qquad w(x,0) = 0.
\end{aligned}
\end{equation}

The solution for $v$ is already known from section~\ref{sec.apply_fourier}, $\hat{v}(\mu,t) = \hat{f}(\mu)\,e^{-K\mu^2 t}$.

For $w$, we take the Fourier transform in $x$ to obtain the first-order linear ODE
\begin{equation}\label{eq.nonhom_heat_w_ft}
  \hat{w}_t + K\mu^2\,\hat{w} = \hat{h}(\mu,t),
  \qquad \hat{w}(\mu,0) = 0.
\end{equation}
We solve this by the integrating factor method (Theorem~\ref{th.linear_ODE}). Multiplying by the integrating factor $e^{K\mu^2 t}$,
\begin{equation}\label{eq.nonhom_heat_w_integrating}
  \frac{d}{dt}\bigl[e^{K\mu^2 t}\,\hat{w}\bigr] = e^{K\mu^2 t}\,\hat{h}(\mu,t).
\end{equation}
Integrating from $0$ to $t$ and using $\hat{w}(\mu,0) = 0$,
\begin{equation}\label{eq.nonhom_heat_w_ft_sol}
  e^{K\mu^2 t}\,\hat{w}(\mu,t)
  = \int_0^t e^{K\mu^2 s}\,\hat{h}(\mu,s)\,ds
  \quad\Longrightarrow\quad
  \hat{w}(\mu,t) = \int_0^t \hat{h}(\mu,s)\,e^{-K\mu^2(t-s)}\,ds.
\end{equation}

Applying the inverse Fourier transform and substituting back $\hat{f}$ and $\hat{h}$, and using the Gaussian integral (Theorem~\ref{th.gaussian_integral}), we obtain the full solution:
\begin{equation}\label{eq.nonhom_heat_full}
\begin{aligned}
  u(x,t)
  &= \int_{-\infty}^{\infty}
     \frac{1}{\sqrt{4\pi Kt}}\,e^{-(x-x')^2/(4Kt)}\,f(x')\,dx' \\
  &\quad+ \int_0^t\int_{-\infty}^{\infty}
     \frac{1}{\sqrt{4\pi K(t-s)}}\,e^{-(x-x')^2/[4K(t-s)]}\,h(x',s)\,dx'\,ds.
\end{aligned}
\end{equation}
In terms of the Green's function \eqref{eq.heat_kernel},
\begin{equation}\label{eq.nonhom_heat_green}
  u(x,t) = \int_{-\infty}^{\infty}G(x,x';t)\,f(x')\,dx'
  + \int_0^t\!\int_{-\infty}^{\infty}G(x,x';t-s)\,h(x',s)\,dx'\,ds.
\end{equation}

\subsection{The Dirac delta function}

\begin{definition}[Dirac delta function]\label{def.dirac_delta}
The Dirac delta ``function'' $\delta(x)$ is defined by the two properties
\begin{equation}\label{eq.dirac_delta_def}
  \delta(x) =
  \begin{cases}
    \infty, & x=0, \\
    0,      & x\ne 0,
  \end{cases}
  \qquad\text{and}\qquad
  \int_{-\infty}^{\infty}\delta(x)\,dx = 1.
\end{equation}
It represents a ``point charge'' (or point source).
\end{definition}

For any domain $\Omega \subset \mathbb{R}$,
\begin{equation}\label{eq.dirac_delta_domain}
  \int_\Omega \delta(x)\,dx =
  \begin{cases}
    1, & 0\in\Omega, \\
    0, & 0\notin\Omega.
  \end{cases}
\end{equation}

Note that the heat kernel \eqref{eq.heat_kernel} approaches a delta function as $t \to 0^+$:
\begin{equation}\label{eq.heat_kernel_delta_limit}
  \frac{1}{\sqrt{4\pi Kt}}\,e^{-(x-x')^2/(4Kt)} \to \delta(x-x') \qquad \text{as } t \to 0^+.
\end{equation}

\subsubsection{Connection to Green's function}

The \underline{Green's function} $G$ for the heat equation satisfies
\begin{equation}\label{eq.green_heat_delta}
  \begin{cases}
    G_t = K\,G_{xx}, \\
    G\big|_{t=0} = \delta(x-x').
  \end{cases}
\end{equation}
That is, $G$ is the response to a point-source initial condition at $x=x'$.

\subsubsection{Properties of $\delta(x)$}

\begin{theorem}[Properties of $\delta(x)$]\label{th.delta_properties}
The Dirac delta function has the following properties.
\begin{enumerate}
  \item $\displaystyle\int_{-\infty}^{\infty}\delta(x-a)\,g(x)\,dx = g(a)$.
  \item $\delta(x)$ is even: $\delta(-x)=\delta(x)$.
  \item $\displaystyle\delta(cx) = \frac{1}{|c|}\,\delta(x)$, for $c\ne 0$.
  \item $x\,\delta(x) = 0$.
\end{enumerate}
\end{theorem}

\subsubsection{Fourier transform of $\delta$}

By the sifting property with $g(x) = e^{-i\mu x}$ and $a = 0$,
\begin{equation}\label{eq.delta_ft}
  \hat{\delta}(\mu)
  = \int_{-\infty}^{\infty}\delta(x)\,e^{-i\mu x}\,dx = e^{0} = 1.
\end{equation}

Applying the inverse Fourier transform to $\hat{\delta}(\mu) = 1$, we obtain the \underline{Fourier representation of $\delta$}:
\begin{equation}\label{eq.delta_fourier_rep}
  \delta(x) = \frac{1}{2\pi}\int_{-\infty}^{\infty}e^{i\mu x}\,d\mu.
\end{equation}

\subsection{Green's function: general theory}

\subsubsection{Verification that $G$ is the Green's function}

We verify that the heat kernel $G(x,t) = \frac{1}{\sqrt{4\pi Kt}}\,e^{-x^2/(4Kt)}$ satisfies $G_t = K\,G_{xx}$ and $G\big|_{t\to 0^+} = \delta(x)$.

\begin{proof}
We work in Fourier space. The Fourier transform of $G$ is
\begin{equation}\label{eq.green_ft}
  \hat{G}(\mu,t) = \int_{-\infty}^{\infty}G(x,t)\,e^{-i\mu x}\,dx = e^{-K\mu^2 t}.
\end{equation}
We check both conditions:
\begin{enumerate}
  \item \textbf{PDE:} $\hat{G}_t = -K\mu^2\,e^{-K\mu^2 t} = -K\mu^2\,\hat{G}
    = k\,\widehat{G_{xx}}$, which is the Fourier transform of $G_t = K\,G_{xx}$.
  \item \textbf{Initial condition:} $\hat{G}(\mu,0) = 1 = \hat{\delta}(\mu)$,
    so $G\big|_{t=0} = \delta(x)$. \qedhere
\end{enumerate}
\end{proof}

\subsubsection{Convolution and the convolution theorem}

\begin{definition}[Convolution]\label{def.convolution}
The \underline{convolution} of two functions $f$ and $g$ is defined by
\begin{equation}\label{eq.convolution_def}
  (f\star g)(x) = \int_{-\infty}^{\infty}f(y)\,g(x-y)\,dy.
\end{equation}
\end{definition}

\begin{theorem}[Convolution theorem]\label{th.convolution}
The Fourier transform of a convolution is the product of the Fourier transforms:
\begin{equation}\label{eq.convolution_theorem}
  \widehat{f\star g}(\mu) = \hat{f}(\mu)\cdot\hat{g}(\mu).
\end{equation}
\end{theorem}

\begin{proof}
We compute directly from the definitions:
\begin{equation}\label{eq.proof_convolution}
\begin{aligned}
  \widehat{f\star g}(\mu)
  &= \int_{-\infty}^{\infty}
     \left(\int_{-\infty}^{\infty}f(y)\,g(x-y)\,dy\right)e^{-i\mu x}\,dx \\
  &= \int_{-\infty}^{\infty}f(y)\,e^{-i\mu y}
     \left(\int_{-\infty}^{\infty}g(x-y)\,e^{-i\mu(x-y)}\,dx\right)dy \\
  &= \hat{f}(\mu)\cdot\hat{g}(\mu),
\end{aligned}
\end{equation}
where in the second line we substituted $\mu = x - y$ in the inner integral.
\end{proof}

As a consistency check, the convolution theorem works for $\delta$: convolution with $\delta$ is the identity ($\delta\star g = g$), and in Fourier space $\hat{\delta}\cdot\hat{g} = 1\cdot\hat{g} = \hat{g}$.

\subsubsection{Superposition principle for Green's functions}

The sifting property of $\delta$ implies that any function $f(x)$ can be decomposed as
\begin{equation}\label{eq.f_delta_decomposition}
  f(x) = \int_{-\infty}^{\infty}f(x')\,\delta(x-x')\,dx'.
\end{equation}
This means that $f$ is a ``continuous linear combination'' of delta functions $\delta(x - x')$, with coefficients $f(x')$.

Since $G(x,x';t)$ solves the heat equation with initial data $\delta(x-x')$, by linearity (superposition), the solution for general initial data $f$ is
\begin{equation}\label{eq.green_superposition}
  u(x,t) = \int_{-\infty}^{\infty}G(x,x';t)\,f(x')\,dx'.
\end{equation}

\begin{lemma}\label{lem.green_solution}
If $G(x,x';t)$ satisfies \eqref{eq.green_heat_delta}, then $u(x,t) = \int_{-\infty}^{\infty}G(x,x';t)\,f(x')\,dx'$ is a solution to $u_t = K\,u_{xx}$, $u|_{t=0} = f(x)$.
\end{lemma}

\begin{proof}
We verify both conditions.

\textit{PDE:} Differentiating under the integral sign,
\begin{equation}\label{eq.green_pde_check}
  u_t = \int_{-\infty}^{\infty}G_t(x,x';t)\,f(x')\,dx'
  = \int_{-\infty}^{\infty}K\,G_{xx}(x,x';t)\,f(x')\,dx'
  = K\,u_{xx}.
\end{equation}

\textit{Initial condition:}
\begin{equation}\label{eq.green_ic_check}
  u\big|_{t=0}
  = \int_{-\infty}^{\infty}G(x,x';0)\,f(x')\,dx'
  = \int_{-\infty}^{\infty}\delta(x-x')\,f(x')\,dx'
  = f(x).
\end{equation}
\end{proof}

\subsubsection{Computing $G$ by Fourier transform}

We can also derive the heat kernel by directly solving the Green's function problem \eqref{eq.green_heat_delta}. Taking the Fourier transform in $x$,
\begin{equation}\label{eq.green_ft_ode}
  \hat{G}_t = -K\mu^2\,\hat{G}.
\end{equation}
The initial condition $G|_{t=0} = \delta(x-x')$ gives
\begin{equation}\label{eq.green_ft_ic}
  \hat{G}(\mu,0)
  = \int_{-\infty}^{\infty}\delta(x-x')\,e^{-i\mu x}\,dx = e^{-i\mu x'}.
\end{equation}
Solving the ODE gives $\hat{G}(\mu,t) = e^{-i\mu x'}\,e^{-K\mu^2 t}$. Applying the inverse Fourier transform and using the Gaussian integral (Theorem~\ref{th.gaussian_integral}),
\begin{equation}\label{eq.green_ft_result}
\begin{aligned}
  G(x,x';t) &= \frac{1}{2\pi}\int_{-\infty}^{\infty}e^{i\mu x}\,e^{-i\mu x'}\,e^{-K\mu^2 t}\,d\mu
  = \frac{1}{2\pi}\int_{-\infty}^{\infty}e^{-K\mu^2 t + i\mu(x-x')}\,d\mu \\
  &= \frac{1}{\sqrt{4\pi Kt}}\,e^{-(x-x')^2/(4Kt)}.
\end{aligned}
\end{equation}
This confirms the heat kernel \eqref{eq.heat_kernel}.

\subsection{Green's function for the Laplace equation}

\subsubsection{Poisson's equation in 2D}

Consider Poisson's equation in 2D:
\begin{equation}\label{eq.poisson_2d}
  u_{xx} + u_{yy} = f(x,y).
\end{equation}
The Green's function satisfies
\begin{equation}\label{eq.green_laplace_2d}
  G_{xx} + G_{yy} = \delta(x-x')\,\delta(y-y').
\end{equation}
For $(x,y)\ne(x',y')$, the right-hand side vanishes, so $G_{xx}+G_{yy}=0$.

\subsubsection{Free-space Green's function}

By symmetry, $G$ depends only on the distance $r = \sqrt{(x-x')^2+(y-y')^2}$ from the source point $(x',y')$. In polar coordinates centered at $(x',y')$, the Laplacian of a radially symmetric function gives
\begin{equation}\label{eq.laplacian_polar}
  0 = \frac{1}{r}\bigl(r\,G_r\bigr)_r \qquad\text{for } r>0.
\end{equation}
Integrating once gives $r\,G_r = B$, and integrating again gives
\begin{equation}\label{eq.green_laplace_radial}
  G(r) = A + B\ln r.
\end{equation}

To determine $B$, we integrate the equation $G_{xx}+G_{yy}=\delta(x-x')\delta(y-y')$ over a disk $D_\varepsilon$ of radius $\varepsilon$ centered at $(x',y')$ and apply the divergence theorem:
\begin{equation}\label{eq.green_laplace_normalization}
  1 = \iint_{D_\varepsilon} (G_{xx}+G_{yy})\,dA = \oint_{\partial D_\varepsilon} \frac{\partial G}{\partial r}\,ds
  = \frac{B}{r}\cdot 2\pi r\bigg|_{r=\varepsilon} = 2\pi B.
\end{equation}
Therefore $B = \frac{1}{2\pi}$, and we may take $A = 0$. The \underline{free-space Green's function} for Poisson's equation in 2D is
\begin{equation}\label{eq.green_laplace_2d_result}
  G(x,y;x',y') = \frac{1}{2\pi}\ln r = \frac{1}{2\pi}\ln\sqrt{(x-x')^2+(y-y')^2}.
\end{equation}

The solution to Poisson's equation is then
\begin{equation}\label{eq.poisson_sol}
  u(x,y) = \iint G(x,y;x',y')\,f(x',y')\,dx'\,dy'.
\end{equation}

\subsubsection{Method of mirror images}

The \underline{method of mirror images} allows us to construct Green's functions on half-spaces by placing ``mirror charges'' to enforce boundary conditions.

\begin{example}[Dirichlet on the half-line]\label{ex.mirror_dirichlet}
Consider the heat equation on $x > 0$ with Dirichlet boundary condition:
\begin{equation}\label{eq.mirror_dirichlet_problem}
  u_t = K\,u_{xx},\quad u\big|_{x=0}=0,\quad u\big|_{t=0}=f(x).
\end{equation}
The Green's function must satisfy
\begin{equation}\label{eq.mirror_dirichlet}
  \begin{cases}
    G_t = K\,G_{xx}, \\
    G\big|_{t=0} = \delta(x-x'), \\
    G\big|_{x=0} = 0.
  \end{cases}
\end{equation}

We place a mirror charge of opposite sign $-\delta(x+x')$ at $-x'$. The free-space Green's function is $G^{\text{free}} = \frac{1}{\sqrt{4\pi Kt}}\,e^{-(x-x')^2/(4Kt)}$ and the image contribution is $G^{\text{image}} = -\frac{1}{\sqrt{4\pi Kt}}\,e^{-(x+x')^2/(4Kt)}$. Therefore
\begin{equation}\label{eq.mirror_dirichlet_sol}
  G^D(x,x';t) = \frac{1}{\sqrt{4\pi Kt}}
  \left[e^{-(x-x')^2/(4Kt)} - e^{-(x+x')^2/(4Kt)}\right].
\end{equation}

One can verify that the boundary condition is satisfied: $G^D(0,x';t) = \frac{1}{\sqrt{4\pi Kt}}
\bigl[e^{-x'^2/(4Kt)} - e^{-x'^2/(4Kt)}\bigr] = 0$.
\end{example}

\begin{example}[Neumann on the half-line]\label{ex.mirror_neumann}
For the Neumann boundary condition $u_x(0,t) = 0$, the image has the \emph{same} sign (even reflection):
\begin{equation}\label{eq.mirror_neumann_sol}
  G^N(x,x';t) = \frac{1}{\sqrt{4\pi Kt}}
  \left[e^{-(x-x')^2/(4Kt)} + e^{-(x+x')^2/(4Kt)}\right].
\end{equation}
The solution is then $u(x,t) = \int_0^{\infty}G^N(x,x';t)\,f(x')\,dx'$.
\end{example}

\begin{example}[2D Laplace equation on the half-plane]\label{ex.mirror_2d_laplace}
For the 2D Laplace equation on the half-plane $x>0$ with Dirichlet boundary condition $G|_{x=0}=0$, we place a mirror charge at $(-x', y')$:
\begin{equation}\label{eq.mirror_2d_laplace}
  G^D(x,y;x',y') = \frac{1}{4\pi}
  \ln\!\left[\frac{(x-x')^2+(y-y')^2}{(x+x')^2+(y-y')^2}\right].
\end{equation}
\end{example}
