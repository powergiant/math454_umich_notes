\section{Introduction and preliminaries}

\subsection{Definitions of partial differential equations}

\begin{definition}[Notations of partial derivatives]
For $f(x)$ with one variable $x$, we know $f^{\prime}(x)=\frac{d f}{d x}$. For $u(x, y)$, we introduce partial derivatives as
\begin{equation}\label{eq.1st_derivatives}
    \frac{\partial u}{\partial x}=\left.\frac{d u}{d x}\right|_{y \text { is fixed }}=\partial_x u=u_x .
\end{equation}
Similarly,
\begin{equation}\label{eq.2nd_derivatives}
    \frac{\partial^2 u}{\partial x^2}=\partial_x^2 u=\partial_{xx} u=u_{x x}
\end{equation}
\end{definition}

\begin{example}[]
For $u(x, y)=x y^2$, we have
\begin{equation*}
\frac{\partial u}{\partial x} = \partial_x u = u_x = y^2, \qquad \frac{\partial u}{\partial y} = \partial_y u = u_y = 2 x y,
\end{equation*}
\begin{equation*}
\frac{\partial^2 u}{\partial x^2}=\partial_x^2 u=\partial_{xx} u=u_{x x}=0, \qquad \frac{\partial^2 u}{\partial y^2}=\partial_y^2 u=\partial_{yy} u=u_{y y}=2 x
\end{equation*}

\end{example}

\begin{definition}[Definition of general PDEs]
    Given a function $u = u(x, y)$ of two variables, (similarly $u = u(x_1, \cdots, x_n)$ of $n$ variables) and an expression $F(u_{xx}, u_{xy}, u_{yy}, u_{x}, u_{y}, u, x, y)$ of partial derivatives of $u$, the following equation is a \underline{partial differential equation}, abbreviated as \underline{PDE}.
    \begin{equation}\label{eq.PDE_general}
        F(u_{xx}, u_{xy}, u_{yy}, u_{x}, u_{y}, u, x, y) = 0
    \end{equation}
    In the future, we may also use the notation $F[u]$ to represent $F(u_{xx}, u_{xy}, u_{yy}, u_{x}, u_{y}, u, x, y)$. And \eqref{eq.PDE_general} can be rewritten as 
    \begin{equation}\label{eq.PDE_general_short}
        F[u] = 0.
    \end{equation}
\end{definition}
\begin{remark}[]
    $F[u]$ may also involve derivatives of order $\ge 2$, but we do not discuss it in this course.
\end{remark}

\begin{example}[Examples of PDEs]
    Here are some examples of PDEs.
    \begin{equation}\label{eq.examples_PDE}
        \begin{split}
            &u_{x x}-u_y=0 \quad \text{(the heat equation)}
            \\
            &u_{x x}-u_{y y}=0 \quad \text{(the wave equation)}
            \\
            &u_{x x}+u_{y y}=0 \quad \text{(Laplace's equation)}
            \\
            &u_{x}+u_{y}=0 \quad \text{(the transport equation)}
            \\
            &u_{x}+uu_{y}=0 \quad \text{(the Burgers equation)}
        \end{split}
    \end{equation}
\end{example}

\begin{definition}[Order of PDEs]
    The \underline{order} of a PDE is the order of the highest-order derivative in the equation.
    In \eqref{eq.examples_PDE}, the first three PDEs are second order, and the last two are first order.
\end{definition}

\begin{definition}[Linear PDEs]
    Given a PDE $F[u] = 0$, if it satisfies 
    \begin{equation}\label{eq.linear_pde_def}
        F[u+v] = F[u] + F[v]\textit{ and }F[cu] = cF[u],
    \end{equation}
    then we say that $F[u] = 0$ is a \underline{linear PDE}. In \eqref{eq.examples_PDE}, the first four PDEs are linear, while the last one is not.
\end{definition}

We have the following proposition which characterizes all second order linear PDEs,

\begin{proposition}[]
The second-order linear PDEs can always be written as
\begin{equation}\label{eq.2nd_order_linear}
    a(x, y) u_{x x}+b(x, y) u_{x y}+c(x, y) u_{y y}+d(x, y) u_x+e(x, y) u_y+f(x, y) u=g(x, y),
\end{equation}
We assume $a^2+b^2+c^2 \neq 0$ for any $x$, $y$ (at least one of $a, b, c$ is nonzero). 
\end{proposition}
\begin{proof}
    This go beyond the scope of this course.
\end{proof}

\begin{definition}[]
    We call $a, b, c, d, e, f$ \underline{coefficients} and $g$ \underline{source term}. 
\end{definition}

\subsection{Classification of second-order PDEs}

In this course, we will mainly consider second-order linear PDEs.
\begin{equation}\label{eq.2nd_order_linear'}
    a(x, y) u_{x x}+b(x, y) u_{x y}+c(x, y) u_{y y}+d(x, y) u_x+e(x, y) u_y+f(x, y) u=g(x, y),
\end{equation}

These equations are classified as follows by the coefficients $a, b, c$.

\begin{definition}[Classification of PDEs] The second-order linear PDEs \eqref{eq.2nd_order_linear'} are classified as \underline{elliptic}, \underline{parabolic} and \underline{hyperbolic} by the following,
    \begin{equation}\label{eq.classification_PDE}
        \begin{cases}
            4 a c-b^2>0 & \text { elliptic } 
            \\ 
            4 a c-b^2=0 & \text { parabolic } 
            \\ 
            4 a c-b^2<0 & \text { hyperbolic }
        \end{cases}
    \end{equation}
where we note that $a, b, c$ are functions of $x, y$ and the inequalities in \eqref{eq.classification_PDE} is required to be true for any $x$, $y$.
\end{definition}


\subsection{Review of ODEs}

\begin{definition}[Separable ODEs]
    The following ODE is the \underline{separable} ODE
    \begin{equation}\label{eq.separable}
        y' + p(x)y = 0.
    \end{equation}
\end{definition}

\begin{theorem}[]\label{th.separable_ODE}
    Separable ODE can be solved in the following way
    \begin{equation*}
        y' + p(x)y = 0
        \quad\Rightarrow\quad \frac{dy}{dx} + p(x)y = 0 \quad\Rightarrow\quad \frac{dy}{y} = - p(x)dx
        \quad\Rightarrow\quad
        \int \frac{dy}{y} = - \int p(x)dx
    \end{equation*}
    and the solution is 
    \begin{equation}\label{eq.sol_separable}
        y(x) = Ce^{-\int p(x)dx}
    \end{equation}
\end{theorem}
\begin{proof}
    There is nothing to prove.
\end{proof}

\begin{example}[]
    Let us solve the following ODEs
    \begin{equation*}
        y^{\prime} = -6 x y
    \end{equation*}

    Apply the procedure in Theorem \ref{th.separable_ODE}
    \begin{equation*}
        y' = -6 x y
        \quad\Rightarrow\quad \frac{dy}{y} = -6 x dx
        \quad\Rightarrow\quad
        \int \frac{dy}{y} = - \int 6 x dx
        \quad\Rightarrow\quad
        \ln |y|=-3 x^2+C^{\prime}
    \end{equation*}
    Therefore, the solution is 
    \begin{equation*}
        y=C e^{-3 x^2}.
    \end{equation*}
    where $C(= \pm e^{C^{\prime}})$ is an arbitrary constant.
\end{example}

\begin{definition}[Linear ODEs]
    The following ODE is the \underline{linear} ODE
    \begin{equation}\label{eq.linear}
        y' + p(x)y = q(x).
    \end{equation}
\end{definition}

\begin{theorem}[]\label{th.linear_ODE}
    Linear ODE can be solved by the following procedure
    \begin{enumerate}
        \item Solve the corresponding separable equation $y' - p(x)y = 0$ to obtain a solution $\hat{y} = e^{\int p(x)dx}$.
        \item Multiply the linear ODE by $\hat{y}$ and rewrite the ODE
        \begin{equation*}
            y' + p(x)y = q(x)
            \quad\Rightarrow\quad 
            \hat{y}(y' + p(x)y) = \hat{y}q(x)
            \quad\Rightarrow\quad 
            (\hat{y}y)' = \hat{y}q(x)
        \end{equation*}
        \item Integrate the above equation 
        \begin{equation*}
            \begin{split}
                &\hat{y}y = \int \hat{y}q(x) dx + C
                \quad\Rightarrow\quad 
                y = \frac{1}{\hat{y}}\left(\int \hat{y}q(x) dx +  C\right)
                \\
                &
                \quad \Rightarrow \quad
                y = e^{-\int p(x)dx}\left(\int q(x)e^{\int p(x)dx} dx+C\right)
            \end{split}
        \end{equation*}
    \end{enumerate}
\end{theorem}
\begin{proof}
    There is nothing to prove.
\end{proof}
\begin{remark}[]
    In (2), we applied the following equation
    \begin{equation*}
        (\hat{y}y)' = \hat{y}(y' + p(x)y) 
    \end{equation*} 
    which is a corollary of the Leibniz rule.
    \begin{equation*}
        (\hat{y}y)' = \hat{y}'y + \hat{y}y' = \hat{y}y' + p(x) \hat{y} y = \hat{y}(y' + p(x)y) 
    \end{equation*}
\end{remark}

\begin{example}[]
    $(x^2+1) y^{\prime}+3 x y=6 x$, $y(0)=3$ is solved as $y(x)=2+(x^2+1)^{-3 / 2}$ by the procedure in Theorem \ref{th.linear_ODE}.

    To apply Theorem \ref{th.linear_ODE}, we divide both sides by $(x^2+1)$.
    \begin{equation*}
        (x^2+1) y^{\prime}+3 x y=6x 
        \quad\Rightarrow\quad
        y^{\prime} + \frac{3 x}{x^2+1} y = \frac{6x}{x^2+1}
    \end{equation*}
    then we apply the three steps in  Theorem \ref{th.linear_ODE}.
    \begin{enumerate}
        \item Solve the corresponding separable equation $y' - \frac{3 x}{x^2+1}y = 0$ to obtain a solution $(x^2+1)^{\frac{3}{2}}$.
        \item Multiply the linear ODE by $(x^2+1)^{\frac{3}{2}}$ and rewrite the ODE
        \begin{equation*}
            y^{\prime} + \frac{3 x}{x^2+1} y = \frac{6x}{x^2+1} 
            \quad\Rightarrow\quad
            ((x^2+1)^{\frac{3}{2}}y)' = 6x(x^2+1)^{\frac{1}{2}}
        \end{equation*}
        \item Integrate the above equation
        \begin{equation*}
            \begin{split}
                ((x^2+1)^{\frac{3}{2}}y)' = 6x(x^2+1)^{\frac{1}{2}}
                \quad&\Rightarrow\quad
                y = (x^2+1)^{-\frac{3}{2}} \left(\int 6x(x^2+1)^{\frac{1}{2}}+ C\right)
                \\
                &\Rightarrow\quad
                y = 2 + C(x^2+1)^{-\frac{3}{2}}
            \end{split} 
        \end{equation*}
    \end{enumerate}

We note that the solutions with undetermined constant $C$ are called \underline{general solutions}. Finally, we apply the initial condition $y(0)=3$ to obtain $C = 1$, so the solution of the initial value problem is 
\begin{equation*}
    y = 2 + (x^2+1)^{-\frac{3}{2}}.
\end{equation*}
\end{example}

\begin{definition}[Second order ODEs]
    The \underline{constant coefficient second order ODEs} are the following equations
    \begin{equation}\label{eq.2nd_ODE}
        a y^{\prime \prime}+b y^{\prime}+c y=0\ (a \neq 0) .
    \end{equation}
\end{definition}

\begin{theorem}[]\label{th.2nd_ODE}
    Constant coefficient second order ODEs can be solved by the following procedure
    \begin{enumerate}
        \item Solve the \underline{characteristic equation} $a\lambda^2 +  b\lambda + c = 0$ to get two solutions $\lambda_1$ and $\lambda_2$.
        \item If $\lambda_1\neq\lambda_2$, the general solution is 
        \begin{equation}\label{eq.sol_2nd_ode}
            y(x)=C_1 e^{\lambda_1 x}+C_2 e^{\lambda_2 x}
        \end{equation}
        \item If $\lambda_1=\lambda_2=\lambda$, the general solution is 
        \begin{equation}\label{eq.sol_2nd_ode_eqroot}
            y(x)=(C_1 + C_2x) e^{\lambda x}
        \end{equation}
        \item If $\lambda_1, \lambda_2$ are complex roots $\alpha \pm i\beta$, apply the Euler's formula to rewrite \eqref{eq.sol_2nd_ode}
        \begin{equation}\label{eq.sol_2nd_ode_complex}
            y(x)=C_1 e^{(\alpha+i \beta) x}+C_2 e^{(\alpha-i \beta) x}=e^{\alpha x}\left(c_1 \cos \beta x+c_2 \sin \beta x\right)
        \end{equation}
        where $c_1=C_1+C_2$, $c_2=i(C_1-C_2)$.
    \end{enumerate}
\end{theorem}
\begin{proof}
    This was explained in your ODE course.
\end{proof}

\begin{example}[]
    $y'' - 3y' + 2y = 0$ is solved as $y(x)=C_1e^{x}+C_2e^{2x}$ by the procedure in Theorem \ref{th.2nd_ODE}.
    \begin{enumerate}
        \item Solve the characteristic equation $\lambda^2 - 3\lambda + 2 = 0$ to get two solutions $\lambda_1 = 1$ and $\lambda_2 = 2$.
        \item Since $\lambda_1\neq\lambda_2$, by \eqref{eq.sol_2nd_ode}, the general solution is 
        \begin{equation}
            y(x)=C_1 e^{x}+C_2 e^{2x}
        \end{equation}
    \end{enumerate}
\end{example}

\begin{example}[]
    $y'' + y = 0$ is solved as $y(x)=C_1\cos(x)+C_2\sin(x)$ by the procedure in Theorem \ref{th.2nd_ODE}.
    \begin{enumerate}
        \item Solve the characteristic equation $\lambda^2 + 1 = 0$ to get two solutions $\lambda_1 = i$ and $\lambda_2 = -i$.
        \item Since $\lambda_1,\lambda_2$ are complex, by \eqref{eq.sol_2nd_ode_complex}, the general solution is 
        \begin{equation}
            y(x)=C_1 \cos x + C_2 \sin x
        \end{equation}
    \end{enumerate}
\end{example}

\begin{example}[]
    $y'' + 2y' + y = 0$ is solved as $y(x)=(C_1+C_2x)e^{-x}$ by the procedure in Theorem \ref{th.2nd_ODE}.
    \begin{enumerate}
        \item Solve the characteristic equation $\lambda^2 + 2\lambda + 1 = 0$ to get $\lambda_1 = \lambda_2 = -1$.
        \item Since $\lambda_1,\lambda_2$ are equal, by \eqref{eq.sol_2nd_ode_eqroot}, the general solution is 
        \begin{equation}
            y(x)=(C_1+C_2x)e^{-x}
        \end{equation}
    \end{enumerate}
\end{example}

% \subsection{Boundary value problem}
% \textbf{TODO: }

\subsection{Separation of variables}
Many linear PDEs can be reduced to linear ODEs with the method of separation of
variables, described below.

We take the Laplace's equation
\begin{equation}\label{eq.Laplace}
    u_{x x}+u_{y y}=0
\end{equation}
with boundary condition
\begin{equation}\label{eq.Laplace_boundary}
    u(0, y)=0, \quad u(L, y)=0, \quad u(x, 0)=0, \quad u(x, L)=\varphi(x).
\end{equation}
as an example.

We are looking for a \underline{separated solution}. Substitute into \eqref{eq.Laplace}, then we get 
\begin{equation*}
    X^{\prime \prime} Y+X Y^{\prime \prime}=0 \quad\Rightarrow\quad
    \frac{X^{\prime \prime}}{X}= -\frac{Y^{\prime \prime}}{Y}
\end{equation*}

The following lemma implies that $X''/X$ and $Y''/Y$ are constants.

\begin{lemma}[]\label{lem.separation_var}
    $f(x) = g(y)$ implies that $f(x) = g(y) = \textit{const}$,
\end{lemma}
\begin{proof}
    $f(x) = g(y)$ $\Rightarrow$ $f'(x) = \partial_x (g(y)) = 0$ $\Rightarrow$ $f(x) = \textit{const}$.
\end{proof}

Let $\lambda$ be a constant and we write
\begin{equation*}
    X^{\prime \prime}+\lambda X=0, \quad Y^{\prime \prime}-\lambda Y=0 .
\end{equation*}

We call $\lambda$ the \underline{separation constant}. At this moment $\lambda$ is arbitrary. Thus the PDE was reduced to two ODEs.

\subsubsection{Solving separated solutions}

If $\lambda=0$, then two ODEs have the following linearly independent solutions.
\begin{equation}\label{eq.separated_basis_0}
X=1, x, \quad Y=1, y.
\end{equation}

If $\lambda \neq 0$, then two ODEs have the following linearly independent solutions.
\begin{equation}\label{eq.separated_basis}
X=e^{\sqrt{-\lambda} x}, e^{-\sqrt{-\lambda} x}, \quad Y=e^{\sqrt{\lambda} y}, e^{-\sqrt{\lambda} y} .
\end{equation}

In either case, the solution is given by superpositions:
\begin{equation}\label{eq.basis_separated}
u=
\left\{\begin{aligned}
&\left(A_1 x+A_2\right)\left(B_1 y+B_2\right), && \lambda=0 
\\
&\left(A_1 e^{\sqrt{-\lambda} x}+A_2 e^{-\sqrt{-\lambda x}}\right)\left(B_1 e^{\sqrt{\lambda} y}+B_2 e^{-\sqrt{\lambda} y}\right), && \lambda \neq 0
\end{aligned}\right.
\end{equation}
where $A_1, A_2, B_1, B_2$ are constants. 

For $\lambda>0$, by writing $\lambda=k^2(k>0)$ we have
\begin{equation}\label{eq.basis_+}
u(x, y)=\left(A_1 e^{i k x}+A_2 e^{-i k x}\right)\left(B_1 e^{k y}+B_2 e^{-k y}\right),
\end{equation}
and for $\lambda<0$, by writing $\lambda=-l^2(l>0)$ we have
\begin{equation}\label{eq.basis_-}
u(x, y)=\left(A_1 e^{l x}+A_2 e^{-l x}\right)\left(B_1 e^{i l y}+B_2 e^{-i l y}\right) .
\end{equation}

Therefore, we get 

\begin{equation}\label{eq.basis_separated_+-}
u=
\left\{\begin{aligned}
&\left(A_1 x+A_2\right)\left(B_1 y+B_2\right), && 
\\
&\left(A_1 e^{i k x}+A_2 e^{-i k x}\right)\left(B_1 e^{k y}+B_2 e^{-k y}\right), && 
\\
&\left(A_1 e^{l x}+A_2 e^{-l x}\right)\left(B_1 e^{i l y}+B_2 e^{-i l y}\right).
\end{aligned}\right.
\end{equation}

Instead of \eqref{eq.separated_basis}, we can also choose
\begin{equation}
X=\cos (\sqrt{\lambda} x), \sin (\sqrt{\lambda} x), \quad Y=\cosh (\sqrt{\lambda} y), \sinh (\sqrt{\lambda} y) .
\end{equation}

In this case, we have
\begin{equation}\label{eq.cos_basis_+}
    u(x, y)=\left(A_1 \cos (k x)+A_2 \sin (k x)\right)\left(B_1 \cosh (k y)+B_2 \sinh (k y)\right)
\end{equation}
\begin{equation}\label{eq.cos_basis_-}
    u(x, y)=\left(A_1 \cosh (l x)+A_2 \sinh (l x)\right)\left(B_1 \cos (l y)+B_2 \sin (l y)\right)
\end{equation}

Note that \eqref{eq.cos_basis_+} becomes \eqref{eq.basis_+} and \eqref{eq.cos_basis_-} becomes \eqref{eq.basis_-} by redefining the coefficients. We call solutions such as \eqref{eq.basis_separated} through \eqref{eq.cos_basis_-} separated solutions because they are given in the form $u(x, y)=X(x) Y(y)$.

The final result is 
\begin{equation}\label{eq.basis_separated_final}
u=
\left\{\begin{aligned}
&\left(A_1 x+A_2\right)\left(B_1 y+B_2\right), 
\\
&\left(A_1 \cos (k x)+A_2 \sin (k x)\right)\left(B_1 \cosh (k y)+B_2 \sinh (k y)\right),
\\
&\left(A_1 \cosh (l x)+A_2 \sinh (l x)\right)\left(B_1 \cos (l y)+B_2 \sin (l y)\right).
\end{aligned}\right.
\end{equation}

\subsubsection{Solving the boundary value problem}
The separation constant $\lambda$ and coefficients $A_1, A_2, B_1, B_2$ are partially determined by boundary conditions that in the region $0<x<L, 0<y<\infty$, $u(x, y)$ satisfies that
\begin{equation}\label{eq.solve_boundary_value_1}
u(0, y)=0, \quad u(L, y)=0, \quad u(x, 0)=0, \quad u(x, L)=\varphi(x).
\end{equation}

Let us only consider the first three conditions.
\begin{equation}\label{eq.solve_boundary_value_2}
u(0, y)=0, \quad u(L, y)=0, \quad u(x, 0)=0.
\end{equation}

In \eqref{eq.basis_separated_final}, we have three cases.

\begin{enumerate}
    \item $u=\left(A_1 x+A_2\right)\left(B_1 y+B_2\right)$. In this case, $u$ satisfies boundary conditions $u(0, y)=u(L, y)=0$ in \eqref{eq.solve_boundary_value_2} if and only if $u = 0$.
    \item $u = \left(A_1 \cos (k x)+A_2 \sin (k x)\right)\left(B_1 \cosh (k y)+B_2 \sinh (k y)\right)$. In this case, $u$ satisfies boundary conditions $u(0, y)=u(L, y)=0$ when $A_1=0$ and $k=n \pi / L$, where $n$ is an integer. Furthermore we find $B_1=0$ by the condition $u(x, 0)=0$. That is, the solution $A_2 B_2 \sin (k x) \sinh (k y)$ with $k=n \pi / L$ $(n=0, \pm 1, \pm 2, \ldots)$ satisfies the boundary conditions.
     
    \item $u = \left(A_1 \cosh (l x)+A_2 \sinh (l x)\right)\left(B_1 \cos (l y)+B_2 \sin (l y)\right)$. In this case, $u$ satisfies boundary conditions $u(0, y)=u(L, y)=0$ only when $A_1=A_2=0$. That is, only the solution $u=0$ satisfies the boundary conditions.
\end{enumerate}

Therefore we obtain the following separated solutions of Laplace's equation satisfying the boundary conditions.
\begin{equation}
u(x, y)=A \sin \frac{n \pi x}{L} \sinh \frac{n \pi y}{L}, \quad n=1,2, \ldots,
\end{equation}

\subsubsection{Matching with the last boundary condition}\label{sec.match_boundary}

Now we consider the last boundary condition $u(x, L)=\varphi(x)$ in \eqref{eq.solve_boundary_value_2}. For arbitrary $\varphi(x)$, our separated solution $A \sin \frac{n \pi x}{L} \sinh \frac{n \pi y}{L}$ cannot match with this boundary condition. However, we can use a linear combination of this separated solution to generate more solutions,
\begin{equation}\label{eq.linear_combination_sep}
u(x, y)=\sum_{n = 1}^{\infty} A_n \sin \frac{n \pi x}{L} \sinh \frac{n \pi y}{L}.
\end{equation}

To match with $u(x, L)=\varphi(x)$, we take $y = L$ in \eqref{eq.linear_combination_sep},
\begin{equation}\label{eq.match_boundary}
\varphi(x) = u(x, L)=\sum_{n = 1}^{\infty} A_n \sin \frac{n \pi x}{L} \sinh \frac{n \pi L}{L}.
\end{equation}

Therefore, we get 
\begin{equation}\label{eq.match_boundary_1}
\varphi(x) =\sum_{n = 1}^{\infty}  A_n \sinh n \pi \sin \frac{n \pi x}{L}.
\end{equation}

Then $A_n$ can be solved from the sine Fourier coefficients introduced in section \ref{sec.A_n_coef_sep}.

The result is 
\begin{equation}\label{eq.solve_A_n}
    A_n = \frac{2}{L\sinh n\pi}\int_{0}^{L}\varphi(x)\sin \frac{n \pi x}{L} dx
\end{equation}

Substitute into \eqref{eq.linear_combination_sep}, then we solve Laplace's equation \eqref{eq.Laplace} with boundary condition.

