\section{PDEs in spherical coordinates}

\subsection{Laplace equation in spherical coordinates}

In this section, we will discuss how to solve the Laplace equation $\Delta u = 0$ in spherical coordinates. The Laplacian operator $\Delta$ is defined by the following
\begin{equation}\label{eq.Laplacian_spherical}
    \Delta = \partial_{xx} + \partial_{yy} + \partial_{zz} \quad \textrm{(in 3D)}
\end{equation}

We consider the spherical domain $\{(x, y, z): x^2 + y^2 + z^2 \le R^2\}$, for which spherical coordinates $(r, \theta, \varphi)$ are particularly useful. The transformation from cartesian coordinates $(x, y, z)$ to spherical coordinates can simplify the domain to $\{(x, y, z): r \le R\}$.

The relationship between Cartesian coordinates $(x, y, z)$ and spherical coordinates $(r, \theta, \varphi)$ is given by
\begin{equation}\label{eq.from_cartesian_to_spherical}
    x=r \sin \theta \cos \varphi, \quad y=r \sin \theta \sin \varphi, \quad z=r \cos \theta.
\end{equation}

Additionally, we have the inverse relationships,
\begin{equation}\label{eq.from_spherical_to_cartesian}
    r = \sqrt{x^2 + y^2 + z^2}, \quad \theta = \arctan\left(\frac{z}{\sqrt{x^2 + y^2}}\right), \quad \varphi = \arctan\left(\frac{y}{x}\right).
\end{equation}

To convert PDEs from Cartesian to spherical coordinates, we need to express the partial derivatives with respect to $x$, $y$ and $z$ in terms of $r$, $\theta$ and $\varphi$.

Here is a picture of spherical coordinates.
\textbf{TODO: }


\begin{proposition}[]
The Laplacian transforms as in the spherical coordinates
\begin{equation}\label{eq.Laplace_spherical_2nd_order}
    \Delta f = f_{xx} + f_{yy} + f_{zz} = \frac{1}{r^2} \left(r^2 u_r\right)_r + \frac{1}{r^2 \sin \theta} \left(\sin \theta\, u_\theta\right)_\theta+\frac{1}{r^2 \sin ^2 \theta} u_{\varphi \varphi}.
\end{equation}
\end{proposition}
\begin{proof} Let $\rho = \sqrt{x^2 + y^2}$, then we know that we have
$$
    z=r \cos \theta, \quad \rho=r \sin \theta,
$$
which is the same as the transformation rule of the polar coordinates \eqref{eq.from_cartesian_to_spherical}.
    
By \eqref{eq.Laplace_2nd_order} to $u_{z z}+u_{\rho \rho}$, we know that
\begin{equation}\label{eq.proof_Laplace_spherical_1}
    u_{z z}+u_{\rho \rho}=u_{r r}+\frac{1}{r} u_r+\frac{1}{r^2} u_{\theta \theta}.
\end{equation}
    
    
For $x$, $y$, we also have $x=\rho \cos \varphi$, $y=\rho \sin \varphi$. Apply \eqref{eq.Laplace_2nd_order} again to $u_{x x}+u_{y y}$, we get
\begin{equation}\label{eq.proof_Laplace_spherical_2}
\begin{split}
    u_{x x}+u_{y y}+u_{z z} =& u_{\rho \rho}+\frac{1}{\rho} u_\rho + \frac{1}{\rho^2} u_{\theta \theta} + u_{z z}
    \\
    =&u_{r r}+\frac{1}{r} u_r+\frac{1}{r^2} u_{\theta \theta}+\frac{1}{\rho} u_\rho+\frac{1}{\rho^2} u_{\varphi \varphi}.
\end{split}
\end{equation}
    
We note that  
$$
    r=\sqrt{\rho^2+z^2} \quad\Rightarrow\quad r_\rho=\frac{\rho}{r} = \sin\theta, 
$$  
$$
    \tan \theta=\frac{\rho}{z} \quad\Rightarrow\quad \frac{d \tan \theta}{d \theta} \frac{\partial \theta}{\partial \rho}=\frac{1}{z} \quad\Rightarrow\quad \frac{1}{\cos ^2 \theta} \frac{\partial \theta}{\partial \rho}=\frac{1}{r \cos \theta} \quad\Rightarrow\quad \theta_\rho=\frac{\cos \theta}{r}.
$$
Therefore, we obtain that 
$$
    u_\rho = r_\rho u_r + \theta_\rho u_\theta = \sin\theta\, u_r + \frac{\cos \theta}{r} u_\theta
$$

Inserting into \eqref{eq.proof_Laplace_spherical_2}, we obtain
\begin{equation}\label{eq.proof_Laplace_spherical_3}
\begin{split}
    u_{x x}+u_{y y}+u_{z z} =& u_{r r}+\frac{1}{r} u_r+\frac{1}{r^2} u_{\theta \theta}+\frac{1}{r\sin \theta} \left(\sin\theta\, u_r + \frac{\cos \theta}{r} u_\theta\right)+\frac{1}{r^2\sin^2 \theta} u_{\varphi \varphi}
    \\ 
    =& \frac{1}{r^2} \left(r^2 u_r\right)_r + \frac{1}{r^2 \sin \theta} \left(\sin \theta\, u_\theta\right)_\theta+\frac{1}{r^2 \sin ^2 \theta} u_{\varphi \varphi}.
\end{split}
\end{equation}
which completes the proof of the proposition.
\end{proof}


This expression is crucial for solving the Laplace equation in spherical coordinates.

\subsection{Separation of Variables for the 3D Laplace equation}
Let's consider solving the equation on a spherical domain
\begin{equation} 
    \Delta u = 0, \quad \sqrt{x^2+y^2+z^2} \leq R
\end{equation}
where $\Delta$ is the Laplacian operator.

We assume that the solution is rotational symmetric with respect to the $z$-axis, i.e. it is independent of the $\varphi$ variable, $u = u(r, \theta)$. Under this assumption, the Laplace equation becomes
\begin{equation}\label{eq.laplace}
    \Delta u = \frac{1}{r^2} \left(r^2 u_r\right)_r + \frac{1}{r^2\sin\theta} \left(\sin \theta\, u_{\theta}\right)_{\theta} = 0,
\end{equation}

Let us find separated solutions of Laplace's equation $\Delta u = 0$ on the domain $0 \leq \sqrt{x^2+y^2+z^2} \leq R$. In terms of spherical coordinates, the domain becomes $0\le r \le R$, $0 \leq \theta \leq \pi$. The boundary value problem should be written as
\begin{equation}\label{eq.boundary_value_spherical_disk}
\left\{\begin{aligned}
    &\Delta u = \frac{1}{r^2} \left(r^2 u_r\right)_r + \frac{1}{r^2\sin\theta} \left(\sin \theta\, u_{\theta}\right)_{\theta} = 0, \quad && 0\leq r \leq R, \quad 0 \leq \theta \leq \pi
    \\
    &u(R, \theta)=G(\theta), && -\pi \leq \theta \leq \pi. 
\end{aligned}\right.
\end{equation}

By plugging $u(r, \theta)=R(r) \Theta(\theta)$ into $\Delta u=0$, we obtain
$$
0=\frac{1}{r^2} (r^2 u_r)_r+\frac{1}{r^2\sin\theta} \left(\sin \theta\, u_{\theta}\right)_{\theta}=\frac{1}{r^2}(r^2 R^{\prime})^{\prime} \Theta+\frac{1}{r^2\sin\theta} (\sin\theta\,\Theta^{\prime})' R
$$

Dividing by $R \Theta$ and multiplying by $r^2$, we have
$$
\frac{(r^2 R^{\prime})^{\prime}}{R} = - \frac{(\sin\theta\,\Theta^{\prime})^{\prime}}{\sin\theta\,\Theta}.
$$


By introducing the separation constant $\mu$, we have
\begin{equation}\label{eq.Laplace_separation_var_spherical}
\left\{\begin{aligned}
    &(\sin\theta\,\Theta^{\prime})^{\prime}+\mu \sin\theta\,\Theta=0, \quad \Theta(0),\, \Theta(\pi)  < \infty, 
    \\
    &(r^2 R^{\prime})^{\prime} - \mu R = 0.
\end{aligned}
\right.
\end{equation}

We can solve $R(r)$ as follows. Inserting the change of variable $r = e^s$, the equation of $R$ becomes $R''(s) + R'(s) - \mu R(s) = 0$. Denote the solutions of the characteristic equation $x^2 + x - \mu = 0$ by $x_{\pm}$. We get $x_{\pm} = \frac{-1 \pm \sqrt{1 + 4\mu}}{2}$ and the general solution is given by $R = Ae^{x_+ s} + Be^{x_- s} = Ar^{x_+} + Br^{x_-}$. To get a smooth solution, it is required that $x_{\pm}$ are integers. We can check that if $\mu = k(k+1)$, then the solutions are integers: $x_+ = k$ and $x_- = -k-1$. The finiteness of $R(0)$ implies that $B = 0$. In summary, we have 
\begin{equation}\label{eq.solve_Laplace_3D_R}
    \mu = k(k+1),\quad R(r) = Ar^{k}.
\end{equation}

Inserting $\mu = k(k+1)$ into the equation of $\theta$, we get 
\begin{equation}\label{eq.Legendre'}
    (\sin\theta\,\Theta^{\prime})^{\prime}+k(k+1) \sin\theta\,\Theta=0, \quad \Theta(0),\, \Theta(\pi)  < \infty.
\end{equation}
This is the Legendre equation. If we denote the solution of it by $P_k(\cos\theta)$, referred to as Legendre polynomial. Then we have 
\begin{equation}\label{eq.solve_Laplace_3D_theta}
    \Theta(\theta) = P_k(\cos\theta).
\end{equation}

Therefore, the separated solutions are obtained as
\begin{equation}
u(r, \theta)= A_k r^k P_k(\cos\theta)
\end{equation}

The general solution is
$$
u(r, \theta)=\sum_{k = 0}^\infty A_k r^k P_k(\cos\theta)
$$

Finally, to decide the coefficients $A_k$, we need to discuss the properties of the Legendre polynomial.

\subsection{The Legendre polynomial}

When applying separation of variables to the 3D Laplace equation, a new class of functions, known as Legendre polynomial, will emerge.

\subsubsection{The definition of Legendre polynomial}

\begin{definition}[The Legendre polynomial] The \underline{Legendre equation} is given by the follows
\begin{equation}\label{eq.Legendre_equation}
    (\sin\theta\,\Theta^{\prime})^{\prime}+k(k+1) \sin\theta\,\Theta=0.
\end{equation}
The solution $P_k(\cos\theta)$ to the above equation satisfying $\Theta(-\pi)=\Theta(\pi)$, $\Theta^{\prime}(-\pi)=\Theta^{\prime}(\pi)$ is refered to as the \underline{Legendre polynomial}. 
\end{definition}
\begin{remark}[]
    The Bessel function defined in this manner has an ambiguity in constants, as any solution of the form $C J_m(x)$ also satisfies the above equation. This ambiguity is resolved by defining $J_m(x)$ using \eqref{eq.Bessel_int_formula} below.
\end{remark}

\begin{lemma}[]
The Legendre is equivalent to the following equations.
\begin{equation}\label{eq.Legendre_equiv}
\begin{gathered}
    (\sin\theta\,\Theta^{\prime})^{\prime}+k(k+1) \sin\theta\,\Theta=0,
    \\
    \Theta^{\prime\prime} + \cot\theta\,\Theta^{\prime}+k(k+1) \Theta=0,
    \\
    ((1-s^2)y')' + k(k+1) y = 0
    \\
    (1-s^2)y'' - 2sy' + k(k+1) y = 0
\end{gathered}
\end{equation}
\end{lemma}
\begin{proof}
    The first two follows from a simple computation. The second two follows from a change of variable $s = \cos \theta$. (We have $y(s) = \Theta(\arccos s)$ and $y(\cos \theta) = \Theta(\theta)$.)
\end{proof}

\subsubsection{Formulas of the Legendre polynomial}

We can solve the Legendre polynomial using Taylor series as in the proof of the following proposition.
\begin{proposition}[]
    The Legendre polynomial is a polynomial in $s$ of degree $k$.
\end{proposition}
\begin{proof}
Assume that the Taylor series of the solution \( y(s) \) is given by $y(s) = \sum_{n=0}^{\infty} a_n s^n$.
The first derivatives can be computed as
\[
\begin{gathered}
    y'(s) = \sum_{n=0}^{\infty} n a_n s^{n-1} = \sum_{n=0}^{\infty} n a_n s^{n-1}
    \\
    (1-s^2)y'(s) = \sum_{n=0}^{\infty} n a_n s^{n-1} - \sum_{n=0}^{\infty} n a_n s^{n+1} = \sum_{n=-2}^{-1} (n+2) a_{n+2} s^{n+1} + \sum_{n=0}^{\infty} ((n+2) a_{n+2} - n a_n) s^{n+1}
\end{gathered}
\]
where we used the fact that $\sum_{n=0}^{\infty} n a_n s^{n-1} = \sum_{n=-2}^{-1} (n+2) a_{n+2} s^{n+1} + \sum_{n=0}^{\infty} (n+2) a_{n+2} s^{n+1}$.

The second order derivative can be computed as 
\[
    ((1-s^2)y'(s))' = \sum_{n=-2}^{-1} (n+2)(n+1) a_{n+2} s^n + \sum_{n=0}^{\infty} ((n+2) a_{n+2} - n a_n)(n+1) s^n
\]
    
Inserting into the Legendre equation $((1 - s^2)y')' + k(k+1) y = 0$, we get
\[
    \sum_{n=0}^{\infty} [(n+1)(n+2)a_{n+2} + (k(k+1) - n(n+1)) a_n] s^n  = 0
\]
which implies that
\begin{equation}\label{eq.proof_Legendre_is_poly_1}
    a_{n+2} = \frac{n(n+1) - k(k+1)}{(n+1)(n+2)} a_n = \frac{(n-k)(n+k+1)}{(n+1)(n+2)} a_n
\end{equation}

From the above recursion formula, we get the following observations.

\begin{enumerate}
    \item Given \( a_k \neq 0 \), we can calculate \( a_{k+2}, a_{k+4}, \ldots \) and so on, providing one set of solutions. Similarly, given \( a_{k+1} \), we can calculate \( a_{k+3}, a_{k+5}, \ldots \), providing a second set of solutions.
    \item By \eqref{eq.proof_Legendre_is_poly_1}, \( a_{k+2} = 0 \), which implies that \( a_{k+2} = a_{k+4} = \ldots = 0\). If we take $a_{k+1} = 0$, we also have \( a_{k+3} = a_{k+5} = \cdots = 0 \). In this case, the series terminates at $k$, making \( y(s) \) a polynomial.
    \item By \eqref{eq.proof_Legendre_is_poly_1}, if $a_{k+1} \neq 0$, then we have $a_{k+3} \neq 0$, $a_{k+5} \neq 0, \cdots$. A complicated analysis implies that $a_{k}\not\rightarrow 0$ and the series is not convergent. Therefore, $a_k$ must be zero.
\end{enumerate}

Combining the above analysis, the Legendre polynomial is a polynomial in $s$. Because $a_n = 0$ and $a_k\neq 0$ for $n > k$, the degree of the polynomial is equal to $k$.
\end{proof}

\begin{proposition}[Rodrigue formula] We have (up to a constant)
\begin{equation}\label{eq.Rodrigue}
    P_k(s)=\frac{1}{2^k k!}\left(\frac{d}{d s}\right)^k\left(s^2-1\right)^k, \quad k=0,1,2, \ldots .
\end{equation}
\end{proposition}
\begin{proof}
Let us expand the polynomial $\left(\frac{d}{d s}\right)^k\left(s^2-1\right)^k$ with Legendre polynomials.
$$
    \left(\frac{d}{d s}\right)^k\left(s^2-1\right)^k=\sum_{j=0}^k c_j P_j(s).
$$
As $\left(\frac{d}{d s}\right)^k\left(s^2-1\right)$ is a degree $k$ polynomial, the upper limit of the above sum is $k$.

As $P_k(s)$ is a solution to a Sturm-Liouville problem, they are orthogonal. From this, we obtain
\[
    c_j \int_{-1}^1 (P_j(s))^2\, \mathrm{d}s = \int_{-1}^1\left(\frac{d}{d s}\right)^k\left(s^2-1\right)^k\, P_j(s)\, \mathrm{d}s
\]
Using integration by parts, for $j < k$, we get
\[
\begin{split}
    c_j \int_{-1}^1 (P_j(s))^2\, \mathrm{d}s =& \left[\left(\frac{d}{d s}\right)^{k-1}\left(s^2-1\right)^k\, P_j(s)\right]_{-1}^1 - \int_{-1}^1\left(\frac{d}{d s}\right)^{k-1}\left(s^2-1\right)^k\, P'_j(s)\, \mathrm{d}s
    \\
    =& \int_{-1}^1\left(\frac{d}{d s}\right)^{k-1}\left(s^2-1\right)^k\, P'_j(s)\, \mathrm{d}s = - \int_{-1}^1\left(\frac{d}{d s}\right)^{k-2}\left(s^2-1\right)^k\, P''_j(s)\, \mathrm{d}s = \cdots
    \\
    =& (-1)^k \int_{-1}^1\left(s^2-1\right)^k\, P^{(k)}_j(s)\, \mathrm{d}s = 0
\end{split}
\]
where we used the fact that $\left(\frac{d}{d s}\right)^j\left(s^2-1\right)^k|_{s = \pm 1} = 0$ for any $j < k$ (see Lemma \ref{lem.Legendre_derivative}) and $P^{(k)}_j(s) = 0$ (the $k$-th derivative of a degree $k$ polynomial is $0$).

Therefore, we know that $c_j=0$ for $j<k$, which implies that $ \left(\frac{d}{d s}\right)^k\left(s^2-1\right)^k= c_k\, P_k(s)$. The choice of $c_k = 2^k k!$ implies \eqref{eq.Rodrigue}.
\end{proof}

The following lemma is important in the proof of the Rodrigue formula.
\begin{lemma}[]\label{lem.Legendre_derivative}
    $\left(\frac{d}{d s}\right)^j\left(s^2-1\right)^k|_{s = \pm 1} = 0$ for $j < k$.
\end{lemma}
\begin{proof}
Let us only prove the case of $s = 1$. We know that $\left(\frac{d}{d s}\right)^j\left(s^2-1\right)^k|_{s = 1}$ is proportional to the $j$-th Taylor coefficient of $\left(s^2-1\right)^k$ at $s = 1$. By $\left(s^2-1\right)^k = (s-1)^k(s+1)^k = 2^k(s-1)^k(1 + \frac{s-1}{2})^k$, the $j$-th Taylor coefficient is $0$ when $j < k$. Therefore, $\left(\frac{d}{d s}\right)^j\left(s^2-1\right)^k|_{s = 1} = 0$.
\end{proof}

The following lemma is important in the computation of Legendre polynomial
\begin{lemma}[]
    $\left(\frac{d}{d s}\right)^k\Big|_{s = 0} s^l = l!\cdot \delta_{kl}$    
\end{lemma}
\begin{proof}
    For $k > l$, we have$\left(\frac{d}{d s}\right)^k s^l = l!\cdot \delta_{kl} = 0$. For $k\le l$, $\left(\frac{d}{d s}\right)^k s^l = \frac{l!}{k!}s^{l-k}$ when $k \le l$. When $k < l$, $\left(\frac{d}{d s}\right)^k\Big|_{s = 0} s^l = \frac{l!}{{l-k}!}s^{l-k}|_{s = 0} = 0$. When $k = l$, $\left(\frac{d}{d s}\right)^l s^l = l!$.
\end{proof}


Using the Rodrigue formula, we can show that 
\begin{proposition}[Recursion formula of Legendre polynomial] We have
\begin{equation}\label{eq.Legendre_recursion}
\begin{gathered}
    n P_n(s)=(2 n-1) s P_{n-1}(s)-(n-1) P_{n-2}(s) \quad n=2,3, \cdots, 
    \\ 
    P_0(s)=1, \quad P_1(s)=s .
\end{gathered}
\end{equation}
\end{proposition}
\begin{proof}
    \textbf{TODO: }
\end{proof}

\begin{example}[] By Rodrigue formula, we can compute that $P_0(s) = 1$, $P_1(s) = s$ and $P_2(s) = \frac{1}{2^2 2!}\frac{d^2}{ds^2} (s^2 - 1)^2 = \frac{1}{2}(3s^2 - 1)$.
\end{example}


\subsubsection{The orthogonality relation}
Consider the following Sturm-Liouville eigenvalues problem,
\begin{equation}\label{eq.Legendre_SL_problem}
\left\{
\begin{aligned}
    &(\sin\theta\,\Theta^{\prime})^{\prime}+k(k+1) \sin\theta\,\Theta=0, 
    \\
    &\Theta(0),\, \Theta(\pi)  < \infty.
\end{aligned}
\right.
\end{equation}
or equivalently
\begin{equation}\label{eq.Legendre_SL_problem'}
\left\{
\begin{aligned}
    &((1-s^2)y'(s))' + k(k+1) y(s) = 0, 
    \\
    &y(-1),\, y(1)  < \infty.
\end{aligned}
\right.
\end{equation}
We know that $y(s) = P_k(s)$ is a solution to the above equation. Applying the Sturm-Liouville theorem Theorem \ref{th.SL_1}, we get the following proposition.

\begin{proposition}[]
We have the following orthogonality relations of the Legendre polynomial.
\begin{equation}\label{eq.Legendre_orthogonality}
    \int_{-1}^1 P_k(s) P_{k'}(s) d s=\int_0^\pi P_k(\cos \theta) P_{k'}(\cos \theta) \sin \theta d \theta=\frac{2}{2k+1}\delta_{kk'}.
\end{equation}
\end{proposition}
\begin{proof} 
Using the Sturm-Liouville theorem, it is straight forward to show that the integral is 0 when $k\neq k'$. Therefore, it suffices to show that $\int_{-1}^1 (P_k(s))^2 d s = \frac{2}{2k+1}$.

By Rodrigue formula, 
\begin{equation}\label{eq.proof_orthogonal_Legendre_1}
    \int_{-1}^1 (P_k(s))^2 d s = \frac{1}{2^{2k} (k!)^2}\int_{-1}^1\left(\frac{d}{d s}\right)^k\left(s^2-1\right)^k\ \left(\frac{d}{d s}\right)^k\left(s^2-1\right)^k d s
\end{equation}
Apply integration by parts for $k$ times and the boundary terms vanishes by Lemma \ref{lem.Legendre_derivative}, which implies
\begin{equation}\label{eq.proof_orthogonal_Legendre_2}
    \int_{-1}^1 (P_k(s))^2 d s = \frac{1}{2^{2k} (k!)^2}\int_{-1}^1 \left(s^2-1\right)^k\cdot \left(\frac{d}{d s}\right)^{2k}\left(s^2-1\right)^k d s = \frac{(2k)!}{2^{2k} (k!)^2}\int_{-1}^1 \left(s^2-1\right)^k d s
\end{equation}
where we applied the fact that $\left(\frac{d}{d s}\right)^{2k}\left(s^2-1\right)^k = (2k)!$, which follows from the fact that 
\begin{equation}\label{eq.proof_orthogonal_Legendre_3}
    \left(\frac{d}{d s}\right)^{2k}\left(s^2-1\right)^k = \left(\frac{d}{d s}\right)^{2k}(s^{2k} + \textrm{lower order terms}) = \left(\frac{d}{d s}\right)^{2k} s^{2k}
\end{equation}

Therefore, we get
\begin{equation}\label{eq.proof_orthogonal_Legendre_4}
    \int_{-1}^1 (P_k(s))^2 d s = \frac{(2k)!}{2^{2k} (k!)^2}\int_{-1}^1 \left(s^2-1\right)^k d s  = \frac{(2k)!}{2^{2k} (k!)^2}\int_{-1}^1 \left(s-1\right)^k \left(s+1\right)^k d s
\end{equation}

\begin{equation}\label{eq.proof_orthogonal_Legendre_5}
\begin{split}
    &\int_{-1}^1 \left(s-1\right)^k \left(s+1\right)^k d s = \frac{1}{k+1}\int_{-1}^1 \left(s-1\right)^k \frac{d}{ds}\left(s+1\right)^{k+1} d s
    \\
    =& \left[\frac{1}{k+1}\left(s-1\right)^k \frac{d}{ds}\left(s+1\right)^{k+1}\right]^1_{-1} - \frac{1}{k+1}\int_{-1}^1 \frac{d}{ds}\left(s-1\right)^k \left(s+1\right)^{k+1} d s
    \\
    =&  \frac{k}{k+1}\int_{-1}^1 \left(s-1\right)^{k-1}\left(s+1\right)^{k+1} d s = \cdots = \frac{k\cdots 1}{k+1\cdots (k+k)}\int_{-1}^1 \left(s-1\right)^{k-k}\left(s+1\right)^{k+k} d s
    \\
    =& \frac{(k!)^2}{(2k)!} \int_{-1}^1 \left(s+1\right)^{2k} d s = \frac{(k!)^2}{(2k)!}\frac{1}{2k+1} \left[\left(s+1\right)^{2k}\right]_{-1}^1 = \frac{(k!)^2}{(2k)!}\frac{2^{2k+1}}{2k+1}
\end{split}
\end{equation}
Therefore,
\begin{equation}\label{eq.proof_orthogonal_Legendre_6}
    \int_{-1}^1 (P_k(s))^2 d s = \frac{(2k)!}{2^{2k} (k!)^2}\int_{-1}^1 \left(s-1\right)^k \left(s+1\right)^k d s = \frac{(2k)!}{2^{2k} (k!)^2}\frac{(k!)^2}{(2k)!}\frac{2^{2k+1}}{2k+1} = \frac{2}{2k+1},
\end{equation}
which completes the proof.
\end{proof}

Now we introduce the notion of Fourier--Legendre series

\begin{definition}[Fourier-Legendre series]
Let us consider the expansion of a piecewise smooth function $f(x)$, $0<x<1$, in a series of the form
$$
f(\theta)=\sum_{n=0}^{\infty} A_k P_k(\cos \theta), \qquad f(s)=\sum_{n=0}^{\infty} A_k P_k(s).
$$ 
This is called a \underline{Fourier--Legendre series}. 
\end{definition}

We have the following theorem.

\begin{theorem}[]
If $f(\theta)$, defined on $0<\theta<\pi$ (or $f(s)$, defined on $-1<s<1$), is a piecewise smooth function, then $f(\theta)$ can be expanded in terms of Fourier--Bessel series
$$
    f(\theta)=\sum_{n=0}^{\infty} A_k P_k(\cos \theta),\quad 0<\theta<\pi, \qquad f(s)=\sum_{n=0}^{\infty} A_k P_k(s),\quad -1<s<1.
$$
where the coefficients $A_n$ can be computed by
\begin{equation}\label{eq.Legendre_formula_coef}
    A_k = \frac{2k+1}{2}\int_0^\pi f(\theta)P_k(\cos\theta) \sin\theta\, \mathrm{d}\theta,\quad \textrm{or}\quad A_k = \frac{2k+1}{2}\int_0^\pi f(s)P_k(s) \, \mathrm{d}s, \quad k=0,2, \cdots
\end{equation}
Moreover, the series converges for each $\theta$ or $s$ to $\frac{1}{2}[f(\theta+0)+f(\theta-0)]$ or $\frac{1}{2}[f(s+0)+f(s-0)]$ respectively.
\end{theorem}
\begin{proof}
    \textbf{TODO: } By multiplying (3.10) by $P_k(\cos\theta)$
\end{proof}

\begin{example}[]\label{ex.Fourier_Legendre_1}
Let us compute the Fourier--Legendre series of the function 
$$
f(s)=\left\{\begin{aligned}
    &1, &&s\in (0, 1),
    \\ 
    &-1,\quad &&s\in (-1, 0).
\end{aligned}\right.
$$
\end{example}
\begin{proof}[Solution] The Fourier-Legendre series is given by:
\[
f(s) = \sum_{k=0}^\infty a_k P_k(s),
\]
where the coefficients \( a_k \) are determined by \eqref{eq.Legendre_formula_coef}
\[
a_k = \frac{2k + 1}{2} \int_{-1}^1 f(s) P_k(s) \, ds.
\]

As \( f(s) \) is an odd function, we get $a_k = 0$ for $k = 2m$. Therefore, the expansion simplifies to:
\[
f(s) = \sum_{m=0}^\infty a_{2m+1} P_{2m+1}(s),
\]
To compute these coefficients, we evaluate the integral:
\[
a_{2m+1} = \frac{2(2m+1) + 1}{2} \int_{-1}^1 P_{2m+1}(s) \, ds = (4m + 3) \int_0^1 P_{2m+1}(s) \, ds.
\]

Using the Rodrigue formula, we get 
\[
\begin{split}
    \int_0^1 P_{2m+1}(s) \, ds =& \frac{1}{2^{2m+1} (2m+1)!} \int_0^1 \left(\frac{d}{d s}\right)^{2m+1}\left(s^2-1\right)^{2m+1} \, ds 
    \\
    =& \frac{1}{2^{2m+1} (2m+1)!} \left[\left(\frac{d}{d s}\right)^{2m}\left(s^2-1\right)^{2m+1}\right]_0^1 
    \\
    =& \frac{1}{2^{2m+1} (2m+1)!} \left(\frac{d}{d s}\right)^{2m}\bigg|_{s = 0}\left(s^2-1\right)^{2m+1}
\end{split}
\]

Using the Binomial formula $(a + b)^k = \sum_{l = 0}^k C_k^l a^l b^{k - l}$ and the fact that $\left(\frac{d}{d s}\right)^k\Big|_{s = 0} s^l = l!\cdot \delta_{kl}$, we get 
\[
\begin{split}
    \int_0^1 P_{2m+1}(s) \, ds
    =& \frac{1}{2^{2m+1} (2m+1)!} \left(\frac{d}{d s}\right)^{2m}\bigg|_{s = 0}\left(s^2-1\right)^{2m+1}
    \\
    =& \frac{1}{2^{2m+1} (2m+1)!} \left(\frac{d}{d s}\right)^{2m}\bigg|_{s = 0}\left(\sum_{l = 0}^{2m+1} C_{2m+1}^l (-1)^{2m+1-l}s^{2l}\right)
    \\
    =& \frac{1}{2^{2m+1} (2m+1)!} \left(\frac{d}{d s}\right)^{2m}\bigg|_{s = 0}\left(C_{2m+1}^{m} (-1)^{m+1}s^{2m}\right) 
    \\
    =& \frac{1}{2^{2m+1} (2m+1)!} C_{2m+1}^{m} (-1)^{m+1} (2m)!
    \\
    =& \frac{1}{2^{2m+1} (2m+1)} C_{2m+1}^{m} (-1)^{m+1} 
\end{split}
\]

Inserting into the equation of $a_{2m+1}$, we get 
\[
a_{2m+1} = (4m + 3) \int_0^1 P_{2m+1}(s) \, ds = \frac{4m + 3}{2^{2m+1} (2m+1)} C_{2m+1}^{m} (-1)^{m+1}.
\]

Thus, the Fourier-Legendre series of \( f(s) \) is:
\[
f(s) = \sum_{m=0}^\infty \frac{(-1)^{m+1}(4m + 3)}{2^{2m+1} (2m+1)} C_{2m+1}^{m} \, P_{2m+1}(s).
\]

\end{proof}


\subsection{The electric potential}
Let us consider electric potential satisfying a boundary value problem of the Laplace equation.
\begin{equation}\label{eq.electric_potential}
\left\{\begin{aligned}
    &\Delta u=\frac{1}{r^2} \left(r^2 u_r\right)_r + \frac{1}{r^2\sin\theta} \left(\sin \theta\, u_{\theta}\right)_{\theta} = 0, && 0 \leq r<a, \quad 0\le \theta \le \pi
    \\
    &u(a, \theta)=G(\theta), && 0\le \theta \le \pi, 
\end{aligned}\right.
\end{equation}
where
\begin{equation}\label{eq.electric_potential_boundary}
G(\theta)=\left\{\begin{aligned}
    &1, &&\theta\in (0, \pi/2),
    \\ 
    &0,\quad &&s\in (\pi/2, \pi).
\end{aligned}\right.
\end{equation}

We solve the equation in terms of Legendre polynomial. 

We first look for separated solutions in the form
$$
u(r, \theta)=R(r) \Theta(\theta).
$$
Inserting into \eqref{eq.vibrating_drumhead}, we get
$$
    \frac{(r^2 R^{\prime})^{\prime}}{R} = - \frac{(\sin\theta\,\Theta^{\prime})^{\prime}}{\sin\theta\,\Theta}.
$$
By introducing the separation constant $\mu$, we have
\begin{equation}
\left\{\begin{aligned}
    &(\sin\theta\,\Theta^{\prime})^{\prime}+\mu \sin\theta\,\Theta=0, \quad \Theta(0),\, \Theta(\pi)  < \infty, 
    \\
    &(r^2 R^{\prime})^{\prime} - \mu R = 0.
\end{aligned}
\right.
\end{equation}

We can solve $R(r)$ as follows. Inserting the change of variable $r = e^s$, the equation of $R$ becomes $R''(s) + R'(s) - \mu R(s) = 0$. Denote the solutions of the characteristic equation $x^2 + x - \mu = 0$ by $x_{\pm}$. We get $x_{\pm} = \frac{-1 \pm \sqrt{1 + 4\mu}}{2}$ and the general solution is given by $R = Ae^{x_+ s} + Be^{x_- s} = Ar^{x_+} + Br^{x_-}$. To get a smooth solution, it is required that $x_{\pm}$ are integers. We can check that if $\mu = k(k+1)$, then the solutions are integers: $x_+ = k$ and $x_- = -k-1$. The finiteness of $R(0)$ implies that $B = 0$. In summary, we have 
\begin{equation}
    \mu = k(k+1),\quad R(r) = Ar^{k}.
\end{equation}

Inserting $\mu = k(k+1)$ into the equation of $\theta$, we get 
\begin{equation}
    (\sin\theta\,\Theta^{\prime})^{\prime}+k(k+1) \sin\theta\,\Theta=0, \quad \Theta(0),\, \Theta(\pi)  < \infty.
\end{equation}
This is the Legendre equation and the solution is given by 
\begin{equation}
    \Theta(\theta) = P_k(\cos\theta).
\end{equation}

Therefore, the separated solutions are obtained as
\begin{equation}
u(r, \theta)= A_k r^k P_k(\cos\theta)
\end{equation}

The general solution is
$$
u(r, \theta)=\sum_{k = 0}^\infty A_k r^k P_k(\cos\theta)
$$

Finally, to decide the coefficients $A_k$, we apply the boundary condition $u(a, \theta)=G(\theta)$, which implies that 
$$
G(\theta)=\sum_{k = 0}^\infty A_k a^k P_k(\cos\theta)
$$

By the formula of coefficients \eqref{eq.Legendre_formula_coef}, we get 
\[
\begin{split}
    a_k =& \frac{2k + 1}{2a^k} \int_0^\pi G(\theta) P_k(\cos \theta) \, d\theta = \frac{2k + 1}{2a^k} \int_{-1}^1 G(\arccos s) P_k(s) \, ds
    \\
    =&\frac{2k + 1}{2a^k} \int_0^1 P_k(s) \, ds.
\end{split}
\]

To compute these coefficients, we evaluate the integral $\int_0^1 P_k(s) \, ds$. Using the Rodrigue formula, we get 
\[
\begin{split}
    \int_0^1 P_{k}(s) \, ds =& \frac{1}{2^{k} k!} \int_0^1 \left(\frac{d}{d s}\right)^{k}\left(s^2-1\right)^{k} \, ds 
    \\
    =& \frac{1}{2^{k} k!} \left[\left(\frac{d}{d s}\right)^{k-1}\left(s^2-1\right)^{k}\right]_0^1 
    \\
    =& \frac{1}{2^{k} k!} \left(\frac{d}{d s}\right)^{k-1}\bigg|_{s = 0}\left(s^2-1\right)^{k}
\end{split}
\]

Using the Binomial formula $(a + b)^k = \sum_{l = 0}^k C_k^l a^l b^{k - l}$ and the fact that $\left(\frac{d}{d s}\right)^k\Big|_{s = 0} s^l = l!\cdot \delta_{kl}$, we get 
\[
\begin{split}
    \int_0^1 P_{k}(s) \, ds
    =& \frac{1}{2^{k} k!} \left(\frac{d}{d s}\right)^{k-1}\bigg|_{s = 0}\left(s^2-1\right)^{k}
    \\
    =& \frac{1}{2^{k} k!} \left(\frac{d}{d s}\right)^{k-1}\bigg|_{s = 0}\left(\sum_{l = 0}^{k} C_{k}^l (-1)^{k-l}s^{2l}\right)
\end{split}
\]
If $k = 2m + 1$, then $\frac{1}{2^{k} k!} \left(\frac{d}{d s}\right)^{k-1}\bigg|_{s = 0}\left(C_{k}^{m} (-1)^{m+1}s^{2m}\right)\neq 0$ is the only non-vanishing term. If $k = 2m$, all term vanishes. Therefore, $a_{2m} = 0$. Inserting $k = 2m+1$ in the above formula, we get
\[
\begin{split}
    \int_0^1 P_{2m+1}(s) \, ds
    =& \frac{1}{2^{2m+1} (2m+1)!} \left(\frac{d}{d s}\right)^{2m}\bigg|_{s = 0}\left(C_{2m+1}^{m} (-1)^{m+1}s^{2m}\right) 
    \\
    =& \frac{1}{2^{2m+1} (2m+1)!} C_{2m+1}^{m} (-1)^{m+1} (2m)!
    \\
    =& \frac{1}{2^{2m+1} (2m+1)} C_{2m+1}^{m} (-1)^{m+1} 
\end{split}
\]


Inserting into the equation of $a_{2m+1}$, we get 
\[
a_{2m+1} = \frac{(4m + 3)}{2a^{2m+1}} \int_0^1 P_{2m+1}(s) \, ds = \frac{4m + 3}{2m+1} \frac{(-1)^{m+1}}{2^{2m+2}a^{2m+1}} C_{2m+1}^{m}.
\]

Thus, the solution $u(r, \theta)$ is given by
\[
    u(r, \theta) = \sum_{m=0}^\infty \frac{(-1)^{m+1}(4m + 3)}{2^{2m+2} (2m+1)} C_{2m+1}^{m} \left(\frac{r}{a}\right)^{2m+1} P_{2m+1}(s).
\]