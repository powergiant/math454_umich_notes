\section{PDEs in rectangular coordinates}

In this section, we will consider the separation of variables for more general equations in rectangular coordinates, possibly with variable coefficients and more general boundary conditions. 

\subsection{The heat equation}

\textbf{TODO: explain different senarios, non-homogeneous, time dependent}

\subsubsection{The homogeneous case} \textbf{TODO: compared with Sturm-Liouville theorem}



\subsubsection{The non-homogeneous case} \textbf{TODO: homogeneous has been considered previously, now non-homogeneous}

Let us consider the heat equation in the slab $0<z<L$,
\begin{equation}\label{eq.heat_and_boundary'}
    \left\{\begin{aligned} 
        &u_t=K u_{z z} + r(z), && 0<z<L, \quad t>0, 
        \\ 
        &u \cos \alpha-L u_z \sin \alpha=T_1,\quad && z=0, \quad t>0, 
        \\ 
        &u \cos \beta+L u_z \sin \beta=T_2, && z=L, \quad t>0, 
        \\
        &u=f(z), && 0<z<L, \quad t=0,
    \end{aligned}\right.
\end{equation}
where $f(z), 0<z<L$, is a piecewise smooth function, and $K>0$ and $\alpha, \beta \in[0, \pi)$ are constants. The equation is nonhomogeneous because there is an internal source $r(z)$. The boundary conditions are nonhomogeneous because $T_1$ and $T_2$ are nonzero. 

The approach to solving this equation is analogous to solving the ODE $f^{\prime \prime}+f=e^x$. First, a particular solution $\hat{f}=\frac{1}{2} e^x$ is found by assuming $\hat{f}=C e^x$ and determining $C=\frac{1}{2}$. Then, the general solution is obtained by considering $g=f-\hat{f}$, where $g$ satisfies the homogeneous equation associated with the original ODE.

For the PDE, the method is similar. We first find a particular solution by considering time-independent solutions, which leads to the following definition.


\begin{definition}
    A solution $U(z, t)$ to the first three equations of \eqref{eq.heat_and_boundary'} satisfying $\partial_t U(z, t)=0$ is called a \underline{time-independent solution} or \underline{stationary solution}. This implies that $U(z, t)=U(z)$ satisfies the following ODE,
    \begin{equation}\label{eq.time_independent_sol}
        \left\{\begin{aligned} 
            &K u_{z z} + r(z) = 0, && 0<z<L
            \\ 
            &u \cos \alpha-L u_z \sin \alpha=T_1,\quad && z=0, 
            \\ 
            &u \cos \beta+L u_z \sin \beta=T_2, && z=L
        \end{aligned}\right.
    \end{equation}
\end{definition}

The PDE can be solved using the following theorem:
\begin{theorem}\label{th.solve_nonhomogeneous}
    The solution to \eqref{eq.heat_and_boundary'} can be obtained by the following steps.
    \begin{enumerate}
        \item Solve for the time-independent solution $U(z)$ using the following ODE.
        \begin{equation}\label{eq.time_independent_sol'}
        \left\{\begin{aligned} 
            &K U_{z z}(z) + r(z) = 0, && 0<z<L
            \\ 
            &U \cos \alpha-L U_z \sin \alpha=T_1,\quad && z=0, 
            \\ 
            &U \cos \beta+L U_z \sin \beta=T_2, && z=L
        \end{aligned}\right.
    \end{equation}
    \item Define $v(z, t) = u(z, t) - U(z)$. One can compute that $v(z, t)$ satisfies the homogeneous PDE,
    \begin{equation}\label{eq.heat_homogenized}
        \left\{\begin{aligned} 
            &v_t=K v_{z z}, && 0<z<L, \quad t>0, 
            \\ 
            &v \cos \alpha-L v_z \sin \alpha=0,\quad && z=0, \quad t>0, 
            \\ 
            &v \cos \beta+L v_z \sin \beta=0, && z=L, \quad t>0, 
            \\
            &v(z, 0)=f(z)-U(z), && 0<z<L, \quad t=0,
        \end{aligned}\right.
    \end{equation}
    Then solve $v(z, t)$ from the above equation.
    \item Finally, compute $u(z, t)$ as $u(z, t) = v(z, t) + U(z)$.
    \end{enumerate}
\end{theorem}
\begin{proof}
    The only non-trivial step is showing that $v(z, t)$ satisfies \eqref{eq.heat_homogenized}, which can be verified by a straightforward computation.
\end{proof}

Let us explain the above procedure using the following examples.

\begin{example}
    Let us solve the following heat equation in a slab.
    \begin{equation}
        \left\{
        \begin{aligned}
            &u_t=K u_{z z},\quad && t>0, \quad 0<z<L \\
            &u(0, t)=T_1,\ u(L, t)=T_2 && t>0 \\
            &u(z, 0)=1,\quad && 0<z<L
        \end{aligned}
        \right.
    \end{equation}
\end{example}
\begin{proof}[Solution] We solve the equation following the steps listed in Theorem \ref{th.solve_nonhomogeneous}.

\textit{Step 1.} For this problem, \eqref{eq.time_independent_sol'} can be simplified to 
$$
U^{\prime \prime}(z)=0, \quad 0<z<L, \quad U(0)=T_1, \quad U(L)=T_2
$$
The general solution is given by $U(z)=A+Bz$, and the coefficients $A$, $B$ are determined by the boundary conditions as $A=T_1, B=(T_2-T_1)/L$. 
    
\textit{Step 2.} We then consider $v(z, t)=$ $u(z, t)-U(z)$ which satisfies
    $$
    \left\{\begin{aligned}
        & v_t=K v_{z z}, \quad t>0, \quad 0<z<L 
        \\
        & v(0, t)=v(L, t)=0, \quad t>0, 
        \\
        &v(z, 0)=1-U(z), \quad 0<z<L
        \end{aligned}\right.
    $$

By separation of variable, we write $v(z, t)=\phi(z) T(t)$, where $T^{\prime \prime}+\lambda K T=0$ and $\phi^{\prime \prime}+\lambda \phi=0$. The solutions of the first equation is $T(t)=e^{-\lambda K t}$. Combining the second equation with the Boundary conditions in the PDE, we get the Sturm-Liouville problem, $\phi^{\prime \prime}+\lambda \phi=0$, $0<z<L$, $\phi(0)=\phi(L)=0$, for which we obtain
    $$
    \phi(z)=\sin \frac{n \pi z}{L}, \quad \lambda=\left(\frac{n \pi}{L}\right)^2, \quad n=1,2, \ldots
    $$
Therefore, the separated solutions are $v_n(z, t)=\sin \frac{n \pi z}{L} e^{-(n \pi / L)^2 K t}$ and the general solution $v(z, t)$ is
    $$
    v(z, t)=\sum_{n=1}^{\infty} B_n \sin \frac{n \pi z}{L} e^{-(n \pi / L)^2 K t}
    $$
The coefficients $B_n$ are determined by the initial condition as    
    $$
    \sum_{n=1}^{\infty} B_n \sin \frac{n \pi z}{L} = v(z, 0)=1-U(z)=1-T_1-\frac{T_2-T_1}{L} z
    $$
Using the formula for the coefficients of sine Fourier series, $B_n = \frac{2}{L}\int_0^L (1-U(z))\sin (n \pi z / L) d z$. Combining with the integrals, $\int_0^L \sin (n \pi z / L) d z=L\left(1-(-1)^n\right) /(n \pi)$ and $\int_0^L z \sin (n \pi z / L) d z=L^2(-1)^{n+1} /(n \pi)$, we can compute $B_n$ and obtain
    $$
    v(z, t)=\frac{2}{\pi} \sum_{n=1}^{\infty} \frac{1-T_1-(-1)^n\left(1-T_2\right)}{n} \sin \frac{n \pi z}{L} e^{-(n \pi / L)^2 K t}
    $$ 

\textit{Step 3.} By $u(z, t) = v(z, t) + U(z)$, we get
    $$
    u(z, t)=T_1+\frac{T_2-T_1}{L} z+\frac{2}{\pi} \sum_{n=1}^{\infty} \frac{1-T_1-(-1)^n\left(1-T_2\right)}{n} \sin \frac{n \pi z}{L} e^{-(n \pi / L)^2 K t}
    $$       
This completes the solution.
\end{proof}

\begin{example}
Let us solve the following heat equation in a slab.
\begin{equation}
    \left\{
    \begin{aligned}
        &u_t=K u_{z z},\quad && t>0, \quad 0<z<L \\
        &u_z(0, t)=\Phi,\ u_z(L, t)=\Phi && t>0 \\
        &u(z, 0)=1,\quad && 0<z<L
    \end{aligned}
    \right.
\end{equation}
\end{example}
\begin{proof}[Solution] We solve the equation following the steps listed in Theorem \ref{th.solve_nonhomogeneous}.

\textit{Step 1.} For this problem, \eqref{eq.time_independent_sol'} can be simplified to 
$$
U^{\prime \prime}(z)=0, \quad 0<z<L, \quad U^{\prime}(0)=U^{\prime}(L)=\Phi
$$
The general solution is given by $U(z)=A+Bz$, and the coefficients $B$ are determined by the boundary conditions as $B=\Phi$. For arbitrary choices of $A$, $U(z)=A+\Phi z$ is a time-independent solution. For the purposes of the computation below, we only need a specific time-independent solution, so we choose $A = 0$, giving $U(z)=\Phi z$.

\textit{Step 2.} We then consider $v(z, t)=u(z, t)-$ $U(z)$ which satisfies
$$
    \left\{\begin{aligned}
        & v_t=K v_{z z}, \quad t>0, \quad 0<z<L 
        \\
        &v_z(0, t)=v_z(L, t)=0, \quad t>0, 
        \\
        &v(z, 0)=1-U(z), \quad 0<z<L
        \end{aligned}\right.
    $$

By separation of variable, we write $v(z, t)=\phi(z) T(t)$, where $T^{\prime \prime}+\lambda K T=0$ and $\phi^{\prime \prime}+\lambda \phi=0$. The solutions of the first equation is $T(t)=e^{-\lambda K t}$. Combining the second equation with the Boundary conditions in the PDE, we get the Sturm-Liouville problem, $\phi^{\prime \prime}+\lambda \phi=0$, $0<z<L$, $\phi'(0)=\phi'(L)=0$, for which we obtain
    $$
    \phi(z)=\cos \frac{n \pi z}{L}, \quad \lambda=\left(\frac{n \pi}{L}\right)^2, \quad n=0,1,2, \ldots
    $$
Therefore, the separated solutions are $v_n(z, t)=\cos \frac{n \pi z}{L} e^{-(n \pi / L)^2 K t}$ and the general solution $v(z, t)$ is
    $$
    v(z, t)=\sum_{n=1}^{\infty} A_n \cos \frac{n \pi z}{L} e^{-(n \pi / L)^2 K t}
    $$
The coefficients $A_n$ are determined by the initial condition as    
    $$
    \sum_{n=1}^{\infty} A_n \cos \frac{n \pi z}{L} = v(z, 0)=1-U(z)=1-\Phi z
    $$
    Using the formula for the coefficients of cosine Fourier series, $A_n = \frac{2}{L}\int_0^L (1-U(z))\cos (n \pi z / L) d z$. Combining with the integrals, $\int_0^L z \cos (n \pi z / L) d z=$ $\left((-1)^n-1\right) L^2 /(n \pi)^2$ and $ \int_0^L \cos ^2(n \pi z / L) d z=L / 2$ for $n \neq 0$, we can compute $A_n$ and obtain
    $$
    A_0=1-A-\frac{L \Phi}{2}, \quad A_n=2 L \Phi \frac{1-(-1)^n}{(n \pi)^2}
    $$
and
    $$
    v(z, t)=1-\frac{L}{2}\Phi+\frac{2 L \Phi}{\pi^2} \sum_{n=1}^{\infty} \frac{1-(-1)^n}{n^2} \cos \frac{n \pi z}{L} e^{-(n \pi / L)^2 K t}.
    $$


\textit{Step 3.} By $u(z, t) = v(z, t) + U(z)$, we get
    $$
    u(z, t)=1+\left(z-\frac{L}{2}\right) \Phi+\frac{2 L \Phi}{\pi^2} \sum_{n=1}^{\infty} \frac{1-(-1)^n}{n^2} \cos \frac{n \pi z}{L} e^{-(n \pi / L)^2 K t}.
    $$       
This completes the solution.
\end{proof}

\subsubsection{Physics of heat conduction}

\subsubsection{The case of time-dependent source}

\subsection{The wave equation}

\subsubsection{Physics of string vibration}

\subsubsection{The homogeneous case}

\subsubsection{The non-homogeneous case}


\subsection{The Laplace's equation}

\subsubsection{Physics of static electricity}

\subsubsection{The homogeneous case}

\subsubsection{The non-homogeneous case}